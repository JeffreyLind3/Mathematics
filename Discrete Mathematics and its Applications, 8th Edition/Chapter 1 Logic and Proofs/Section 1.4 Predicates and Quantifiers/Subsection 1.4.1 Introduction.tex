%%%%%%%%%%%%%%%%%%%%%%%%%%%%%%%%%%%%%%%%%%%%%%%%%%%%%%%%%%%%%%%%
%%%%%%%%%%%%%%%%%%%%%%%%%%% Metadata %%%%%%%%%%%%%%%%%%%%%%%%%%%
%%%%%%%%%%%%%%%%%%%%%%%%%%%%%%%%%%%%%%%%%%%%%%%%%%%%%%%%%%%%%%%%
\documentclass{Axon}

\title{Discrete Mathematics and its Applications, 8th Edition - Chapter 1 The Foundations: Logic and Proofs - Section 1.4 Predicates and Quantifiers - Subsection 1.4.1 Introduction}

\authors{
    \addauthor{Jeffrey G. Lind III}{jeffrey@jeffreylind.com}
}

\addbibresource{Bibliography.bib}
%%%%%%%%%%%%%%%%%%%%%%%%%%%%%%%%%%%%%%%%%%%%%%%%%%%%%%%%%%%%%%%%
%%%%%%%%%%%%%%%%%%%%%%%%%%%%% Paper %%%%%%%%%%%%%%%%%%%%%%%%%%%%
%%%%%%%%%%%%%%%%%%%%%%%%%%%%%%%%%%%%%%%%%%%%%%%%%%%%%%%%%%%%%%%%
\begin{document}
\maketitle
\makeauthor
%%%%%%%%%%%%%%%%%%%%%%%%%%%%%%%%%%%%%%%%%%%%%%%%%%%%%%%%%%%%%%%%
%%%%%%%%%%%%%%%%%%%%%%%%%%% Abstract %%%%%%%%%%%%%%%%%%%%%%%%%%%
%%%%%%%%%%%%%%%%%%%%%%%%%%%%%%%%%%%%%%%%%%%%%%%%%%%%%%%%%%%%%%%%
\begin{abstract}
Notes on Discrete Mathematics and its Applications, 8th Edition - Chapter 1 The Foundations: Logic and Proofs - Section 1.4 Predicates and Quantifiers - Subsection 1.4.1 Introduction \cite{Rosen}.
\end{abstract}
%%%%%%%%%%%%%%%%%%%%%%%%%%%%%%%%%%%%%%%%%%%%%%%%%%%%%%%%%%%%%%%%
%%%%%%%%%%%%%%%%%%%%%%%%%%% Section 1 %%%%%%%%%%%%%%%%%%%%%%%%%%
%%%%%%%%%%%%%%%%%%%%%%%%%%%%%%%%%%%%%%%%%%%%%%%%%%%%%%%%%%%%%%%%
\section{Introduction}
Propositional logic, studied in Sections 1.1-1.3, cannot adequately express the meaning of all statements in mathematics and in natural language. For example, suppose that we know that

\begin{center}
    "Every computer connected to the university network is functioning properly."
\end{center}

No rules of propositional logic allow us to conclude the truth of the statement

\begin{center}
    "MATH3 is functioning properly,"
\end{center}

where MATH3 is one of the computers connected to the university network. Likewise, we cannot use the rules of propositional logic to conclude from the statement

\begin{center}
    "CS2 is under attack by an intruder,"
\end{center}

where CS2 is a computer on the universe network, to conclude the truth of

\begin{center}
    "There is a computer on the university network that is under attack by an intruder."
\end{center}

In this section we will introduce a more powerful type of logic called \textbf{predicate logic}. We will see how predicate logic can be used to express the meaning of a wide range of statements in mathematics and computer science in ways that permit us to reason and explore relationships between objects. To understand predicate logic, we first need to introduce the concept of a predicate. Afterward, we will introduce the notion of quantifiers, which enable us to reason with statements that assert that a certain property holds for all objects of a certain type and with statements that assert the existence of an object with a particular property.

\printbibliography

\end{document}