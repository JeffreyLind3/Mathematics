%%%%%%%%%%%%%%%%%%%%%%%%%%%%%%%%%%%%%%%%%%%%%%%%%%%%%%%%%%%%%%%%
%%%%%%%%%%%%%%%%%%%%%%%%%%% Metadata %%%%%%%%%%%%%%%%%%%%%%%%%%%
%%%%%%%%%%%%%%%%%%%%%%%%%%%%%%%%%%%%%%%%%%%%%%%%%%%%%%%%%%%%%%%%
\documentclass{Axon}

\title{Discrete Mathematics and its Applications, 8th Edition - Chapter 1 The Foundations: Logic and Proofs - Section 1.4 Predicates and Quantifiers - Subsection 1.4.5 Quantifiers with Restricted Domains}

\authors{
    \addauthor{Jeffrey G. Lind III}{jeffrey@jeffreylind.com}
}

\addbibresource{Bibliography.bib}
%%%%%%%%%%%%%%%%%%%%%%%%%%%%%%%%%%%%%%%%%%%%%%%%%%%%%%%%%%%%%%%%
%%%%%%%%%%%%%%%%%%%%%%%%%%%%% Paper %%%%%%%%%%%%%%%%%%%%%%%%%%%%
%%%%%%%%%%%%%%%%%%%%%%%%%%%%%%%%%%%%%%%%%%%%%%%%%%%%%%%%%%%%%%%%
\begin{document}
\maketitle
\makeauthor
%%%%%%%%%%%%%%%%%%%%%%%%%%%%%%%%%%%%%%%%%%%%%%%%%%%%%%%%%%%%%%%%
%%%%%%%%%%%%%%%%%%%%%%%%%%% Abstract %%%%%%%%%%%%%%%%%%%%%%%%%%%
%%%%%%%%%%%%%%%%%%%%%%%%%%%%%%%%%%%%%%%%%%%%%%%%%%%%%%%%%%%%%%%%
\begin{abstract}
Notes on Discrete Mathematics and its Applications, 8th Edition - Chapter 1 The Foundations: Logic and Proofs - Section 1.4 Predicates and Quantifiers - Subsection 1.4.5 Quantifiers with Restricted Domains \cite{Rosen}.
\end{abstract}
%%%%%%%%%%%%%%%%%%%%%%%%%%%%%%%%%%%%%%%%%%%%%%%%%%%%%%%%%%%%%%%%
%%%%%%%%%%%%%%%%%%%%%%%%%%% Section 1 %%%%%%%%%%%%%%%%%%%%%%%%%%
%%%%%%%%%%%%%%%%%%%%%%%%%%%%%%%%%%%%%%%%%%%%%%%%%%%%%%%%%%%%%%%%
\section{Introduction}
An abbreviated notation is often used to restrict the domain of a quantifier. In this notation, a condition a variable must satisfy is included after the quantifier. This is illustrated in Example \ref{Example: 17}. We will also describe other forms of this notation involving set membership in Section 2.1.

\begin{example}\label{Example: 17}
    What do the statements \(\forall x < 0 (x^2 > 0)\), \(\forall y \neq 0 (y^3 \neq 0)\), and \(\exists z > 0 (z^2 = 2)\) mean, where the domain in each case consists of the real numbers?

    \noindent
    \textbf{Solution:}
    The statement \(\forall x < 0 (x^2 > 0)\) states that for every real number \(x\) with \(x < 0\), \(x^2 > 0\). That is, it states "The square of a negative real number is positive." This statement is the same as \(\forall x(x < 0 \to x^2 > 0)\).

    The statement \(\forall y \neq 0(y^3 \neq 0)\) states that for every real number \(y\) with \(y \neq 0\), we have \(y^3 \neq 0\). That is, it states "The cube of every nonzero real number is nonzero." This statement is equivalent to \(\forall y(y \neq 0 \to y^3 \neq 0)\).

    Finally, the statement \(\exists z > 0 (z^2 = 2)\) states that there exists a real number \(z\) with \(z > 0\) such that \(z^2 = 2\). That is, it states "There is a positive square root of \(2\)." This statement is equivalent to \(\exists z(z > 0 \land z^2 = 2)\).
\end{example}

Note that the restriction of a universal quantification is the same as the universal quantification of a conditional statement. For instance, \(\forall x < 0 (x^2 > 0)\) is another way of expressing \(\forall x (x < 0 \to x^2 > 0)\). On the other hand, the restriction of an existential quantification is the same as the existential quantification of a conjunction. For instance, \(\exists z > 0 (z^2 = 2)\) is another way of expressing \(\exists z (z > 0 \land z^2 = 2)\).

\printbibliography

\end{document}