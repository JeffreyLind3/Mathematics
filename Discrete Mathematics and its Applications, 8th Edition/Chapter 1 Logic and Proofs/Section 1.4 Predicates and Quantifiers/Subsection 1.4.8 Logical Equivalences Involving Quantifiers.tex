%%%%%%%%%%%%%%%%%%%%%%%%%%%%%%%%%%%%%%%%%%%%%%%%%%%%%%%%%%%%%%%%
%%%%%%%%%%%%%%%%%%%%%%%%%%% Metadata %%%%%%%%%%%%%%%%%%%%%%%%%%%
%%%%%%%%%%%%%%%%%%%%%%%%%%%%%%%%%%%%%%%%%%%%%%%%%%%%%%%%%%%%%%%%
\documentclass{Axon}

\title{Discrete Mathematics and its Applications, 8th Edition - Chapter 1 The Foundations: Logic and Proofs - Section 1.4 Predicates and Quantifiers - Subsection 1.4.8 Logical Equivalences Involving Quantifiers}

\authors{
    \addauthor{Jeffrey G. Lind III}{jeffrey@jeffreylind.com}
}

\addbibresource{Bibliography.bib}
%%%%%%%%%%%%%%%%%%%%%%%%%%%%%%%%%%%%%%%%%%%%%%%%%%%%%%%%%%%%%%%%
%%%%%%%%%%%%%%%%%%%%%%%%%%%%% Paper %%%%%%%%%%%%%%%%%%%%%%%%%%%%
%%%%%%%%%%%%%%%%%%%%%%%%%%%%%%%%%%%%%%%%%%%%%%%%%%%%%%%%%%%%%%%%
\begin{document}
\maketitle
\makeauthor
%%%%%%%%%%%%%%%%%%%%%%%%%%%%%%%%%%%%%%%%%%%%%%%%%%%%%%%%%%%%%%%%
%%%%%%%%%%%%%%%%%%%%%%%%%%% Abstract %%%%%%%%%%%%%%%%%%%%%%%%%%%
%%%%%%%%%%%%%%%%%%%%%%%%%%%%%%%%%%%%%%%%%%%%%%%%%%%%%%%%%%%%%%%%
\begin{abstract}
Notes on Discrete Mathematics and its Applications, 8th Edition - Chapter 1 The Foundations: Logic and Proofs - Section 1.4 Predicates and Quantifiers - Subsection 1.4.8 Logical Equivalences Involving Quantifiers \cite{Rosen}.
\end{abstract}
%%%%%%%%%%%%%%%%%%%%%%%%%%%%%%%%%%%%%%%%%%%%%%%%%%%%%%%%%%%%%%%%
%%%%%%%%%%%%%%%%%%%%%%%%%%% Section 1 %%%%%%%%%%%%%%%%%%%%%%%%%%
%%%%%%%%%%%%%%%%%%%%%%%%%%%%%%%%%%%%%%%%%%%%%%%%%%%%%%%%%%%%%%%%
\section{Introduction}
In Section 1.3 we introduced the notion of logical equivalences of compound propositions. We can extend this notion to expressions involving predicates and quantifiers.

\begin{definition}
    Statements involving predicates and quantifiers are \textit{logically equivalent} if and only if they have the same truth value no matter which predicates are substituted into these statements and which domain of discourse is used for the variables in these propositional functions. We use the notation \(S \equiv T\) to indicate that two statements \(S\) and \(T\) involving predicates and quantifiers are logically equivalent.
\end{definition}

Example \ref{Example: 19} illustrates how to show that two statements involving predicates and quantifiers are logically equivalent.

\begin{example}\label{Example: 19}
    Show that \(\forall x (P(x) \land Q(x))\) and \(\forall x P(x) \land \forall x Q(x)\) are logically equivalent (where the same domain is used throughout). This logical equivalence shows that we can distribute a universal quantifier over a conjunction. Furthermore, we can also distribute an existential quantifier over a disjunction. However, we cannot distribute a universal quantifier over a disjunction, nor can we distribute an existential quantifier over a conjunction. (See Exercises 52 and 53.)

    \noindent
    \textbf{Solution:}
    To show that these statements are logically equivalent, we must show that they always take the same truth value, no matter what the predicates \(P\) and \(Q\) are, and no matter which domain of discourse is used. Suppose we have particular predicates \(P\) and \(Q\), with a common domain. We can show that \(\forall x (P(x) \land Q(x))\) and \(\forall x P(x) \land \forall x Q(x)\) are logically equivalent by doing two things. First, we show that if \(\forall x (P(x) \land Q(x))\) is true, then \(\forall x P(x) \land \forall x Q(x)\) is true. Second, we show that if \(\forall x P(x) \land \forall x Q(x)\) is true, then \(\forall x (P(x) \land Q(x))\) is true.

    So, suppose that \(\forall x (P(x) \land Q(x))\) is true. This means that if \(a\) is in the domain, then \(P(a) \land Q(a)\) is true. Hence, \(P(a)\) is true and \(Q(a)\) is true. Because \(P(a)\) is true and \(Q(a)\) is true for every element \(a\) in the domain, we can conclude that \(\forall x P(x)\) and \(\forall x Q(x)\) are both true. This means that \(\forall x P(x) \land \forall x Q(x)\) is true.
\end{example}

\printbibliography

\end{document}