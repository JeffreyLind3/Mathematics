%%%%%%%%%%%%%%%%%%%%%%%%%%%%%%%%%%%%%%%%%%%%%%%%%%%%%%%%%%%%%%%%
%%%%%%%%%%%%%%%%%%%%%%%%%%% Metadata %%%%%%%%%%%%%%%%%%%%%%%%%%%
%%%%%%%%%%%%%%%%%%%%%%%%%%%%%%%%%%%%%%%%%%%%%%%%%%%%%%%%%%%%%%%%
\documentclass{Axon}

\title{Discrete Mathematics and its Applications, 8th Edition - Chapter 1 The Foundations: Logic and Proofs - Section 1.4 Predicates and Quantifiers - Subsection 1.4.3 Quantifiers}

\authors{
    \addauthor{Jeffrey G. Lind III}{jeffrey@jeffreylind.com}
}

\addbibresource{Bibliography.bib}
%%%%%%%%%%%%%%%%%%%%%%%%%%%%%%%%%%%%%%%%%%%%%%%%%%%%%%%%%%%%%%%%
%%%%%%%%%%%%%%%%%%%%%%%%%%%%% Paper %%%%%%%%%%%%%%%%%%%%%%%%%%%%
%%%%%%%%%%%%%%%%%%%%%%%%%%%%%%%%%%%%%%%%%%%%%%%%%%%%%%%%%%%%%%%%
\begin{document}
\maketitle
\makeauthor
%%%%%%%%%%%%%%%%%%%%%%%%%%%%%%%%%%%%%%%%%%%%%%%%%%%%%%%%%%%%%%%%
%%%%%%%%%%%%%%%%%%%%%%%%%%% Abstract %%%%%%%%%%%%%%%%%%%%%%%%%%%
%%%%%%%%%%%%%%%%%%%%%%%%%%%%%%%%%%%%%%%%%%%%%%%%%%%%%%%%%%%%%%%%
\begin{abstract}
Notes on Discrete Mathematics and its Applications, 8th Edition - Chapter 1 The Foundations: Logic and Proofs - Section 1.4 Predicates and Quantifiers - Subsection 1.4.3 Quantifiers \cite{Rosen}.
\end{abstract}
%%%%%%%%%%%%%%%%%%%%%%%%%%%%%%%%%%%%%%%%%%%%%%%%%%%%%%%%%%%%%%%%
%%%%%%%%%%%%%%%%%%%%%%%%%%% Section 1 %%%%%%%%%%%%%%%%%%%%%%%%%%
%%%%%%%%%%%%%%%%%%%%%%%%%%%%%%%%%%%%%%%%%%%%%%%%%%%%%%%%%%%%%%%%
\section{Introduction}
When the variables in a propositional function are assigned truth values, the resulting statement becomes a proposition with a certain truth value. However, there is another important way, called \textbf{quantification}, to create a proposition from a propositional function. Quantification expresses the extent to which a predicate is true over a range of elements. In English, the words \textit{all}, \textit{some}, \textit{many}, \textit{none}, and \textit{few} are used in quantification. We will focus on two types of quantification here: universal quantification, which tells us that a predicate is true for every element under consideration, and existential quantification, which tells us that there is one or more element under consideration for which the predicate is true. The area of logic that deals with predicates and quantifiers is called the \textbf{predicate calculus}.

\section{The Universal Quantifier}
Many mathematical statements assert that a property is true for all values of a variable in a particular domain, called the \textbf{domain of discourse} (or the \textbf{universe of discourse}, often just referred to as the \textbf{domain}. Such a statement is expressed using universal quantification. The universal quantification of \(P(x)\) for a particular domain is the proposition that asserts that \(P(x)\) is true for all values of \(x\) in this domain. Note that the domain specifies the possible values of the variable \(x\). The meaning of the universal quantification of \(P(x)\) changes when we change the domain. The domain must always be specified when a universal quantifier is used; without it, the universal quantification of a statement is not defined.

\begin{definition}
    The \textit{universal quantification} of \(P(x)\) is the statement
    \begin{center}
        "\(P(x)\) for all values of \(X\) in the domain."
    \end{center}
    The notation \(\forall x P(x)\) denotes the universal quantification of \(P(x)\). Here \(\forall\) is called the \textbf{universal quantifier}. We read \(\forall x P(x)\) as "for all \(x P(x)\)" or "for every \(x P(x)\)." An element for which \(P(x)\) is false is called a \textbf{counterexample} to \(\forall x P(x)\).
\end{definition}

The meaning of the universal quantifier is summarized in the first row of Table \ref{Table: 1}. We illustrate the use of the universal quantifier in Examples \ref{Example: 8}-\ref{Example: 12} and 15.

\begin{example}\label{Example: 8}
    Let \(P(x)\) be the statement "\(x + 1 > x\)." What is the truth value of the quantification \(\forall x P(x)\), where the domain consists of all real numbers?

    \noindent
    \textbf{Solution:}
    Because \(P(x)\) is true for all real numbers \(x\), the quantification
    \begin{center}
        \(\forall x P(x)\)
    \end{center}
    is true.
\end{example}

\textbf{\textit{Remark:}} Generally, an implicit assumption is made that all domains of discourse for quantifiers are nonempty. Note that if the domain is empty, then \(\forall x P(x)\) is true for any propositional function \(P(x)\) because there are no elements \(x\) in the domain for which \(P(x)\) is false.

Besides "for all" and "for every," universal quantification can be expressed in many other ways, including "all of," "for each," "given any," "for arbitrary," "for each," and "for any."

\textbf{\textit{Remark:}} It is best to avoid using "for any \(x\)" because it is often ambiguous as to whether "any" means "every" or "some." In some cases, "any" is unambiguous, such as when it is used in negatives: "There is not any reason to avoid studying."

A statement \(\forall x P(x)\) is false, where \(P(x)\) is a propositional function, if and only if \(P(x)\) is not always true when \(x\) is in the domain. One way to show that \(P(x)\) is not always true when \(x\) is in the domain is to find a counterexample to the statement \(\forall x P(x)\). Note that a single counterexample is all we need to establish that \(\forall x P(x)\) is false. Example \ref{Example: 9} illustrates how counterexamples are used.

\begin{table}[h]
    \centering
    \begin{tabular}{c|c|c}
        \textbf{\textit{Statement}} & \textbf{\textit{When True?}}                  & \textbf{\textit{When False?}}                  \\
        \(\forall x P(x)\)          & \(P(x)\) is true for every \(x\)              & There is an \(x\) for which \(P(x)\) is false. \\
        \(\exists x P(x)\)          & There is an \(x\) for which \(P(x)\) is true. & \(P(x)\) is false for every \(x\).
    \end{tabular}
    \caption{Quantifiers.}
    \label{Table: 1}
\end{table}

\begin{example}\label{Example: 9}
    Let \(Q(x)\) be the statement "\(x < 2\)." What is the truth value of the quantification \(\forall Q(x)\), where the domain consists of all real numbers?

    \noindent
    \textbf{Solution:}
    \(Q(x)\) is not true for every real number \(x\), because, for instance, \(Q(3)\) is false. That is, \(x = 3\) is a counterexample for the statement \(\forall x Q(x)\). Thus,
    \begin{center}
        \(\forall x Q(x)\)
    \end{center}
    is false.
\end{example}

\begin{example}
    Suppose that \(P(x)\) is "\(x^2 > 0\)." To show that the statement \(\forall x P(x)\) is false where the universe of discourse consists of all integers, we give a counterexample. We see that \(x = 0\) is a counterexample because \(x^2 = 0\) when \(x = 0\), so that \(x^2\) is not greater than \(0\) when \(x = 0\).
\end{example}

Looking for counterexamples to universally quantified statements is an important activity in the study of mathematics, as we will see in subsequent sections of this book.

\begin{example}
    What does the statement \(\forall x N(x)\) mean if \(N(x)\) is "Computer \(x\) is connected to the network" and the domain consists of all computers on campus?

    \noindent
    \textbf{Solution:}
    The statement \(\forall x N(x)\) means that for every computer \(x\) on campus, that computer \(x\) is connected to the network. This statement can be expressed in English as "Every computer on campus is connected to the network."
\end{example}

As we have pointed out, specifying the domain is mandatory when quantifiers are used. The truth value of a quantified statement often depends on which elements are in this domain, as Example \ref{Example: 12} shows.

\begin{example}\label{Example: 12}
    What is the truth value of \(\forall x (x^2 \ge x)\) if the domain consists of all real numbers? What is the truth value of this statement if the domain consists of all integers?

    \noindent
    \textbf{Solution:}
    The universal quantification \(\forall x(x^2 \ge x)\), where the domain consists of all real numbers, is false. For example, \(\left(\frac{1}{2}\right)^2 \not\ge \frac{1}{2}\). Note that \(x^2 \ge x\) if and only if \(x^2 - x = x(x - 1) \ge 0\). Consequently, \(x^2 \ge x\) if and only if \(x \le 0\) or \(x \ge 1\). It follows that \(\forall x (x^2 \geq x)\) is false if the domain consists of all real numbers (because the inequality is false for all real numbers \(x\) with \(0 < x < 1\)). However, if the domain consists of the integers, \(\forall x(x^2 \ge x)\) is true, because there are no integers \(x\) with \(0 < x < 1\).
\end{example}

\printbibliography

\end{document}