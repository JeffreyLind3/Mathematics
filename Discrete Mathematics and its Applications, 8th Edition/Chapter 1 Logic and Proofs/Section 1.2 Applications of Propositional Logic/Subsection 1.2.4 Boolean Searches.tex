%%%%%%%%%%%%%%%%%%%%%%%%%%%%%%%%%%%%%%%%%%%%%%%%%%%%%%%%%%%%%%%%
%%%%%%%%%%%%%%%%%%%%%%%%%%% Metadata %%%%%%%%%%%%%%%%%%%%%%%%%%%
%%%%%%%%%%%%%%%%%%%%%%%%%%%%%%%%%%%%%%%%%%%%%%%%%%%%%%%%%%%%%%%%
\documentclass{Axon}

\title{Discrete Mathematics and its Applications, 8th Edition - Chapter 1 The Foundations: Logic and Proofs - Section 1.2 Applications of Propositional Logic - Subsection 1.2.4 Boolean Searches}

\authors{
    \addauthor{Jeffrey G. Lind III}{jeffrey@jeffreylind.dev}
}

\addbibresource{Bibliography.bib}
%%%%%%%%%%%%%%%%%%%%%%%%%%%%%%%%%%%%%%%%%%%%%%%%%%%%%%%%%%%%%%%%
%%%%%%%%%%%%%%%%%%%%%%%%%%%%% Paper %%%%%%%%%%%%%%%%%%%%%%%%%%%%
%%%%%%%%%%%%%%%%%%%%%%%%%%%%%%%%%%%%%%%%%%%%%%%%%%%%%%%%%%%%%%%%
\begin{document}
\maketitle
\makeauthor
%%%%%%%%%%%%%%%%%%%%%%%%%%%%%%%%%%%%%%%%%%%%%%%%%%%%%%%%%%%%%%%%
%%%%%%%%%%%%%%%%%%%%%%%%%%% Abstract %%%%%%%%%%%%%%%%%%%%%%%%%%%
%%%%%%%%%%%%%%%%%%%%%%%%%%%%%%%%%%%%%%%%%%%%%%%%%%%%%%%%%%%%%%%%
\begin{abstract}
Notes on Discrete Mathematics and its Applications, 8th Edition - Chapter 1 The Foundations: Logic and Proofs - Section 1.2 Applications of Propositional Logic - Subsection 1.2.4 Boolean Searches \cite{Rosen}.
\end{abstract}
%%%%%%%%%%%%%%%%%%%%%%%%%%%%%%%%%%%%%%%%%%%%%%%%%%%%%%%%%%%%%%%%
%%%%%%%%%%%%%%%%%%%%%%%%%%% Section 1 %%%%%%%%%%%%%%%%%%%%%%%%%%
%%%%%%%%%%%%%%%%%%%%%%%%%%%%%%%%%%%%%%%%%%%%%%%%%%%%%%%%%%%%%%%%
\section{Introduction}
Logical connectives are used extensively in searches of large collections of information, such as indexes of Web pages. Because these searches employ techniques from propositional logic, they are called \textbf{Boolean searches}.

In Boolean searches, the connective \(AND\) is used to match records that contain both of two search terms, the connective \(OR\) is used to match one or both of two search terms, and the connective \(NOT\) (sometimes written as \(AND \ NOT\) is used to exclude a particular search term. Careful planning of how logical connectives are used is often required when Boolean searches are used to locate information of potential interest. Example \ref{Example: 6} illustrates how Boolean searches are carried out.

\begin{example}\label{Example: 6}
    \textbf{Web Page Searching} Most Web search engines support Boolean searching techniques, which is useful for finding Web pages about particular subjects. For instance, using Boolean searching to find Web pages about universities in New Mexico, we can look for pages matching NEW \(AND\) MEXICO \(AND\) UNIVERSITIES. The results of this search will include those pages that contain the three words NEW, MEXICO, and UNIVERSITIES. This will include all of the pages of interest, together with others such as a page about new universities in Mexico. (Note that Google, and many other search engines, do require the use of "AND" because such search engines use all search terms by default.) Most search engines support the use of quotation marks to search for specific phrases. So, it may be more effective to search for pages matching "NEW MEXICO" \(AND\) UNIVERSITIES.

    Next, to find pages that deal with universities in New Mexico or Arizona, we can search for pages matching (NEW \(AND\) MEXICO \(OR\) ARIZONA) \(AND\) UNIVERSITIES. (\textit{Note:} Here the \(AND\) operator takes precedence over the \(OR\) operator. Also, in Google, the terms used for this search would be NEW MEXICO \(OR\) ARIZONA.) The results of this search will include all pages that contain the word UNIVERSITIES and either both the words NEW and MEXICO or the word ARIZONA. Again, pages besides those of interest will be listed. Finally, to find Web pages that deal with universities in Mexico (and not New Mexico), we might first look for pages matching MEXICO \(AND\) UNIVERSITIES, but because the results of this search will include pages about universities in New Mexico, as well as universities in Mexico, it might be better to search for pages matching (MEXICO \(AND\) UNIVERSITIES) \(NOT\) NEW. The results of this search include pages that contain both the words MEXICO and UNIVERSITIES but do not contain the word NEW. (In Google, and many other search engines, the word "NOT" is replaced by the symbol "-". In Google, the terms used for this last search would be MEXICO UNIVERSITIES - NEW.)
\end{example}

\printbibliography

\end{document}