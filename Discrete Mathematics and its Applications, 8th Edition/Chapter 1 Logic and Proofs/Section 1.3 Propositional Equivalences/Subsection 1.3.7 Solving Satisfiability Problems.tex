%%%%%%%%%%%%%%%%%%%%%%%%%%%%%%%%%%%%%%%%%%%%%%%%%%%%%%%%%%%%%%%%
%%%%%%%%%%%%%%%%%%%%%%%%%%% Metadata %%%%%%%%%%%%%%%%%%%%%%%%%%%
%%%%%%%%%%%%%%%%%%%%%%%%%%%%%%%%%%%%%%%%%%%%%%%%%%%%%%%%%%%%%%%%
\documentclass{Axon}

\title{Discrete Mathematics and its Applications, 8th Edition - Chapter 1 The Foundations: Logic and Proofs - Section 1.3 Propositional Equivalences - Subsection 1.3.6 Applications of Satisfiability}

\authors{
    \addauthor{Jeffrey G. Lind III}{jeffrey@jeffreylind.dev}
}

\addbibresource{Bibliography.bib}
%%%%%%%%%%%%%%%%%%%%%%%%%%%%%%%%%%%%%%%%%%%%%%%%%%%%%%%%%%%%%%%%
%%%%%%%%%%%%%%%%%%%%%%%%%%%%% Paper %%%%%%%%%%%%%%%%%%%%%%%%%%%%
%%%%%%%%%%%%%%%%%%%%%%%%%%%%%%%%%%%%%%%%%%%%%%%%%%%%%%%%%%%%%%%%
\begin{document}
\maketitle
\makeauthor
%%%%%%%%%%%%%%%%%%%%%%%%%%%%%%%%%%%%%%%%%%%%%%%%%%%%%%%%%%%%%%%%
%%%%%%%%%%%%%%%%%%%%%%%%%%% Abstract %%%%%%%%%%%%%%%%%%%%%%%%%%%
%%%%%%%%%%%%%%%%%%%%%%%%%%%%%%%%%%%%%%%%%%%%%%%%%%%%%%%%%%%%%%%%
\begin{abstract}
Notes on Discrete Mathematics and its Applications, 8th Edition - Chapter 1 The Foundations: Logic and Proofs - Section 1.3 Propositional Equivalences - Subsection 1.3.7 Solving Satisfiability Problems \cite{Rosen}.
\end{abstract}
%%%%%%%%%%%%%%%%%%%%%%%%%%%%%%%%%%%%%%%%%%%%%%%%%%%%%%%%%%%%%%%%
%%%%%%%%%%%%%%%%%%%%%%%%%%% Section 1 %%%%%%%%%%%%%%%%%%%%%%%%%%
%%%%%%%%%%%%%%%%%%%%%%%%%%%%%%%%%%%%%%%%%%%%%%%%%%%%%%%%%%%%%%%%
\section{Introduction}
A truth table can be used to determine whether a compound proposition is satisfiable, or equivalently, whether its negation is a tautology (see Exercise 64). This can be done by hand for a compound proposition with a small number of variables, but when the number of variables grows, this becomes impractical. For instance, there are \(2^{20} = 1,048,576\) rows in the truth table for a compound proposition with \(20\) variables. Thus, you need a computer to help you determine, in this way, whether a compound proposition in \(20\) variables is satisfiable.

When many applications are modeled, questions concerning the satisfiability of compound propositions with hundreds, thousands, or millions of variables arise. Note, for example, that when there are \(1000\) variables, checking every one of the \(2^{1000}\) (a number with more than \(300\) decimal digits) possible combinations of truth values of the variables in a compound proposition cannot be done by a computer in even trillions of years. No procedure is known that a computer can follow to determine in a reasonable amount of time whether an arbitrary compound proposition in such a large number of variables is satisfiable. However, progress has been made developing methods for solving the satisfiability problem for the particular types of compound propositions that arise in practical applications, such as for the solution of Sudoku puzzles. Many computer programs have been developed for solving satisfiability problems which have practical use. In our discussion of the subject of algorithms in Chapter 3, we will discuss this question further. In particular, we will explain the important role of the propositional satisfiability problem plays in the study of the complexity of algorithms.

\printbibliography

\end{document}