%%%%%%%%%%%%%%%%%%%%%%%%%%%%%%%%%%%%%%%%%%%%%%%%%%%%%%%%%%%%%%%%
%%%%%%%%%%%%%%%%%%%%%%%%%%% Metadata %%%%%%%%%%%%%%%%%%%%%%%%%%%
%%%%%%%%%%%%%%%%%%%%%%%%%%%%%%%%%%%%%%%%%%%%%%%%%%%%%%%%%%%%%%%%
\documentclass{Axon}

\title{Discrete Mathematics and its Applications, 8th Edition - Chapter 1 The Foundations: Logic and Proofs - Section 1.3 Propositional Equivalences - Exercises}

\authors{
    \addauthor{Jeffrey G. Lind III}{jeffrey@jeffreylind.dev}
}

\addbibresource{Bibliography.bib}
%%%%%%%%%%%%%%%%%%%%%%%%%%%%%%%%%%%%%%%%%%%%%%%%%%%%%%%%%%%%%%%%
%%%%%%%%%%%%%%%%%%%%%%%%%%%%% Paper %%%%%%%%%%%%%%%%%%%%%%%%%%%%
%%%%%%%%%%%%%%%%%%%%%%%%%%%%%%%%%%%%%%%%%%%%%%%%%%%%%%%%%%%%%%%%
\begin{document}
\maketitle
\makeauthor
%%%%%%%%%%%%%%%%%%%%%%%%%%%%%%%%%%%%%%%%%%%%%%%%%%%%%%%%%%%%%%%%
%%%%%%%%%%%%%%%%%%%%%%%%%%% Abstract %%%%%%%%%%%%%%%%%%%%%%%%%%%
%%%%%%%%%%%%%%%%%%%%%%%%%%%%%%%%%%%%%%%%%%%%%%%%%%%%%%%%%%%%%%%%
\begin{abstract}
Notes on Discrete Mathematics and its Applications, 8th Edition - Chapter 1 The Foundations: Logic and Proofs - Section 1.3 Propositional Equivalences - Exercises \cite{Rosen}.
\end{abstract}
%%%%%%%%%%%%%%%%%%%%%%%%%%%%%%%%%%%%%%%%%%%%%%%%%%%%%%%%%%%%%%%%
%%%%%%%%%%%%%%%%%%%%%%%%%%% Section 1 %%%%%%%%%%%%%%%%%%%%%%%%%%
%%%%%%%%%%%%%%%%%%%%%%%%%%%%%%%%%%%%%%%%%%%%%%%%%%%%%%%%%%%%%%%%
\section*{Exercise 1}
Use truth tables to verify these equivalences.

\begin{enumerate}
    \item[\textbf{a)}] \(p \land \textbf{T} \equiv p\)
    \item[\textbf{b)}] \(p \lor \textbf{F} \equiv p\)
    \item[\textbf{c)}] \(p \land \textbf{F} \equiv \textbf{F}\)
    \item[\textbf{d)}] \(p \lor \textbf{T} \equiv \textbf{T}\)
    \item[\textbf{e)}] \(p \lor p \equiv p\)
    \item[\textbf{f)}] \(p \land p \equiv p\)
\end{enumerate}

\noindent
\textbf{Solution:}
\begin{table}[h]
    \centering
    \begin{tabular}{c|c|c}
        \(p\) & \textbf{T} & \(p \land \textbf{T}\) \\
        T     & T          & T                      \\
        F     & T          & F
    \end{tabular}
    \caption{Truth Table for 1(a)}
\end{table}

\begin{table}[h]
    \centering
    \begin{tabular}{c|c|c}
        \(p\) & \textbf{F} & \(p \lor \textbf{F}\) \\
        T     & F          & T                     \\
        F     & F          & F
    \end{tabular}
    \caption{Truth Table for 1(b)}
\end{table}

\begin{table}[h]
    \centering
    \begin{tabular}{c|c|c}
        \(p\) & \textbf{F} & \(p \land \textbf{F}\) \\
        T     & F          & F                      \\
        F     & F          & F
    \end{tabular}
    \caption{Truth Table for 1(c)}
\end{table}

\begin{table}[ht]
    \centering
    \begin{tabular}{c|c|c}
        \(p\) & \textbf{T} & \(p \lor \textbf{T}\) \\
        T     & T          & T                     \\
        F     & T          & T
    \end{tabular}
    \caption{Truth Table for 1(d)}
\end{table}

\begin{table}[ht]
    \centering
    \begin{tabular}{c|c}
        \(p\) & \(p \lor p\) \\
        T     & T            \\
        F     & F
    \end{tabular}
    \caption{Truth Table for 1(e)}
\end{table}

\begin{table}[ht]
    \centering
    \begin{tabular}{c|c}
        \(p\) & \(p \land p\) \\
        T     & T             \\
        F     & F
    \end{tabular}
    \caption{Truth Table for 1(f)}
\end{table}

\section*{Exercise 2}
Show that \(\lnot(\lnot p)\) and \(p\) are logically equivalent

\noindent
\textbf{Solution:}
\begin{table}[ht]
    \centering
    \begin{tabular}{c|c|c}
        \(p\) & \(\lnot p\) & \(\lnot(\lnot p)\) \\
        T     & F           & T                  \\
        F     & T           & F
    \end{tabular}
    \caption{Truth Table for 2}
\end{table}

\section*{Exercise 3}
Use truth tables to verify the commutative laws

\begin{enumerate}
    \item[\textbf{a)}] \(p \lor q \equiv q \lor p\)
    \item[\textbf{b)}] \(p \land q \equiv q \land p\)
\end{enumerate}

\noindent
\textbf{Solution:}

\begin{table}[ht]
    \centering
    \begin{tabular}{c|c|c|c}
        \(p\) & \(q\) & \(p \lor q\) & \(q \lor p\) \\
        T     & T     & T            & T            \\
        T     & F     & T            & T            \\
        F     & T     & T            & T            \\
        F     & F     & F            & F
    \end{tabular}
    \caption{Truth Table for 3(a)}
\end{table}

\begin{table}[ht]
    \centering
    \begin{tabular}{c|c|c|c}
        \(p\) & \(q\) & \(p \land q\) & \(q \land p\) \\
        T     & T     & T             & T             \\
        T     & F     & F             & F             \\
        F     & T     & F             & F             \\
        F     & F     & F             & F
    \end{tabular}
    \caption{Truth Table for 3(b)}
\end{table}

\section*{Exercise 4}
Use truth tables to verify the associative laws

\begin{enumerate}
    \item[\textbf{a)}] \((p \lor q) \lor r \equiv p \lor (q \lor r)\).
    \item[\textbf{b)}] \((p \land q) \land r \equiv p \land (q \land r)\)
\end{enumerate}

\noindent
\textbf{Solution:}

\begin{table}[ht]
    \centering
    \begin{tabular}{c|c|c|c|c|c|c}
        \(p\) & \(q\) & \(r\) & \(p \lor q\) & \((p \lor q) \lor r\) & \(q \lor r\) & \(p \lor (q \lor r)\) \\
        T     & T     & T     & T            & T                     & T            & T                     \\
        T     & T     & F     & T            & T                     & T            & T                     \\
        T     & F     & T     & T            & T                     & T            & T                     \\
        T     & F     & F     & T            & T                     & F            & T                     \\
        F     & T     & T     & T            & T                     & T            & T                     \\
        F     & T     & F     & T            & T                     & T            & T                     \\
        F     & F     & T     & F            & T                     & T            & T                     \\
        F     & F     & F     & F            & F                     & F            & F
    \end{tabular}
    \caption{Truth Table for 4(a)}
\end{table}

\begin{table}[ht]
    \centering
    \begin{tabular}{c|c|c|c|c|c|c}
        \(p\) & \(q\) & \(r\) & \(p \land q\) & \((p \land q) \land r\) & \(q \land r\) & \(p \land (q \land r)\) \\
        T     & T     & T     & T             & T                       & T             & T                       \\
        T     & T     & F     & T             & F                       & F             & F                       \\
        T     & F     & T     & F             & F                       & F             & F                       \\
        T     & F     & F     & F             & F                       & F             & F                       \\
        F     & T     & T     & F             & F                       & T             & F                       \\
        F     & T     & F     & F             & F                       & F             & F                       \\
        F     & F     & T     & F             & F                       & F             & F                       \\
        F     & F     & F     & F             & F                       & F             & F
    \end{tabular}
    \caption{Truth Table for 4(b)}
\end{table}

\section*{Exercise 5}
Use a truth table to verify the distributive law
\begin{equation}
    p \land (q \lor r) \equiv (p \land q) \lor (p \land r)
\end{equation}

\noindent
\textbf{Solution:}

\begin{table}[ht]
    \centering
    \begin{tabular}{c|c|c|c|c|c|c|c}
        \(p\) & \(q\) & \(r\) & \(q \lor r\) & \(p \land (q \lor r)\) & \(p \land q\) & \(p \land r\) & \((p \land q) \lor (p \land r)\) \\
        T     & T     & T     & T            & T                      & T             & T             & T                                \\
        T     & T     & F     & T            & T                      & T             & F             & T                                \\
        T     & F     & T     & T            & T                      & F             & T             & T                                \\
        T     & F     & F     & F            & F                      & F             & F             & F                                \\
        F     & T     & T     & T            & F                      & F             & F             & F                                \\
        F     & T     & F     & T            & F                      & F             & F             & F                                \\
        F     & F     & T     & T            & F                      & F             & F             & F                                \\
        F     & F     & F     & F            & F                      & F             & F             & F
    \end{tabular}
    \caption{Truth Table for 5}
\end{table}

\section*{Exercise 6}
Use a truth table to verify the first De Morgan law
\begin{equation}
    \lnot (p \land q) \equiv \lnot p \lor \lnot q
\end{equation}

\noindent
\textbf{Solution:}

\begin{table}[ht]
    \centering
    \begin{tabular}{c|c|c|c|c|c|c}
        \(p\) & \(q\) & \(p \land q\) & \(\lnot (p \land q)\) & \(\lnot p\) & \(\lnot q\) & \(\lnot p \lor \lnot q\) \\
        T     & T     & T             & F                     & F           & F           & F                        \\
        T     & F     & F             & T                     & F           & T           & T                        \\
        F     & T     & F             & T                     & T           & F           & T                        \\
        F     & F     & F             & T                     & T           & T           & T
    \end{tabular}
    \caption{Truth Table for 6}
\end{table}

\section*{Exercise 7}
Use De Morgan's laws to find the negation of each of the following statements.
\begin{enumerate}
    \item[\textbf{a)}] Jan is rich and happy.
    \item[\textbf{b)}] Carlos will bicycle or run tomorrow
    \item[\textbf{c)}] Mei walks or takes the bus to class.
    \item[\textbf{d)}] Ibrahim is smart and hard working.
\end{enumerate}

\noindent
\textbf{Solution:}
\begin{enumerate}
    \item[\textbf{a)}] Jan is not rich or not happy.
    \item[\textbf{b)}] Carlos will not bicycle and will not run tomorrow.
    \item[\textbf{c)}] Mei does not walk and does not take the bus to class.
    \item[\textbf{d)}] Ibrahim is not smart or not hard working.
\end{enumerate}

\section*{Exercise 8}
Use De Morgan's laws to find the negation of each of the following statements.
\begin{enumerate}
    \item[\textbf{a)}] Kwame will take a job in industry or go to graduate school.
    \item[\textbf{b)}] Yoshiko knows Java and calculus.
    \item[\textbf{c)}] James is young and strong.
    \item[\textbf{d)}] Rita will move to Oregon or Washington.
\end{enumerate}

\noindent
\textbf{Solution:}
\begin{enumerate}
    \item[\textbf{a)}] Kwame will not take a job in industry and Kwame will not go to graduate school.
    \item[\textbf{b)}] Yoshiko does not know Java or Yoshiko does not know calculus.
    \item[\textbf{c)}] James is not young or James is not strong.
    \item[\textbf{d)}] Rita will not move to Oregon and Rita will not move to Washington.
\end{enumerate}

\section*{Exercise 9}
For each of these compound propositions, use the conditional-disjunction equivalence (Example 3) to find an equivalent compound proposition that does not involve conditionals.
\begin{enumerate}
    \item[\textbf{a)}] \(p \to \lnot q\)
    \item[\textbf{b)}] \((p \to q) \to r\)
    \item[\textbf{c)}] \((\lnot q \to p) \to (p \to \lnot q)\)
\end{enumerate}

\noindent
\textbf{Solution:}
The conditional-disjunction equivalence states that for propositional variables \(p\) and \(q\), \(p \to q \equiv \lnot p \lor q\). Therefore,
\begin{enumerate}
    \item[\textbf{a)}] \(p \to \lnot q \equiv \lnot p \lor \lnot q\)
    \item[\textbf{b)}] \((p \to q) \to r \equiv \lnot(\lnot p \lor q) \lor r \equiv (p \land \lnot q) \lor r\)
    \item[\textbf{c)}] \((\lnot q \to p) \to (p \to \lnot q) \equiv (\lnot q \land \lnot p) \lor (\lnot p \lor \lnot q) \equiv \lnot p \lor \lnot q\)
\end{enumerate}

\section*{Exercise 10}
For each of these compound propositions, use the conditional-disjunction equivalence (Example 3) to find an equivalent compound proposition that does not involve conditionals.
\begin{enumerate}
    \item[\textbf{a)}] \(\lnot p \to \lnot q\)
    \item[\textbf{b)}] \((p \lor q) \to \lnot p\)
    \item[\textbf{c)}] \((p \to \lnot q) \to (\lnot p \to q)\)
\end{enumerate}

\noindent
\textbf{Solution:}
The conditional-disjunction equivalence states that for propositional variables \(p\) and \(q\), \(p \to q \equiv \lnot p \lor q\). Therefore,
\begin{enumerate}
    \item[\textbf{a)}] \(\lnot p \to \lnot q \equiv p \lor \lnot q\)
    \item[\textbf{b)}] \((p \lor q) \to \lnot p \equiv \lnot(p \lor q) \lor \lnot p \equiv \lnot p\)
    \item[\textbf{c)}] \((p \to \lnot q) \to (\lnot p \to q) \equiv \lnot(p \to \lnot q) \lor (\lnot p \to q) \equiv \lnot(\lnot p \lor \lnot q) \lor (p \lor q) \equiv p \lor q\)
\end{enumerate}

\section*{Exercise 11}
Show that each of these conditional statements is a tautology by using truth tables.
\begin{enumerate}
    \item[\textbf{a)}] \((p \land q) \to p\)
    \item[\textbf{b)}] \(p \to (p \lor q)\)
    \item[\textbf{c)}] \(\lnot p \to (p \to q)\)
    \item[\textbf{d)}] \((p \land q) \to (p \to q)\)
    \item[\textbf{e)}] \(\lnot(p \to q) \to p\)
    \item[\textbf{f)}] \(\lnot(p \to q) \to \lnot q\)
\end{enumerate}

\noindent
\textbf{Solution:}

\begin{table}[ht]
    \centering
    \begin{tabular}{c|c|c|c}
        \(p\) & \(q\) & \(p \land q\) & \((p \land q) \to p\) \\
        T     & T     & T             & T                     \\
        T     & F     & F             & T                     \\
        F     & T     & F             & T                     \\
        F     & F     & F             & T
    \end{tabular}
    \caption{Truth Table for 11(a)}
\end{table}

\begin{table}[ht]
    \centering
    \begin{tabular}{c|c|c|c}
        \(p\) & \(q\) & \(p \lor q\) & \(p \to (p \lor q)\) \\
        T     & T     & T            & T                     \\
        T     & F     & T            & T                    \\
        F     & T     & T            & T                    \\
        F     & F     & F            & T
    \end{tabular}
    \caption{Truth Table for 11(b)}
\end{table}

\begin{table}[ht]
    \centering
    \begin{tabular}{c|c|c|c|c}
        \(p\) & \(q\) & \(\lnot p\) & \(p \to q\) & \(\lnot p \to (p \to q)\) \\
        T     & T     & F           & T           & T                         \\
        T     & F     & F           & F           & T                         \\
        F     & T     & T           & T           & T                         \\
        F     & F     & T           & T           & T
    \end{tabular}
    \caption{Truth Table for 11(c)}
\end{table}

\begin{table}[ht]
    \centering
    \begin{tabular}{c|c|c|c|c}
        \(p\) & \(q\) & \(p \land q\) & \(p \to q\) & \((p \land q) \to (p \to q)\) \\
        T     & T     & T             & T           & T                             \\
        T     & F     & F             & F           & T                             \\
        F     & T     & F             & T           & T                             \\
        F     & F     & F             & T           & T
    \end{tabular}
    \caption{Truth Table for 11(d)}
\end{table}

\begin{table}[ht]
    \centering
    \begin{tabular}{c|c|c|c|c}
        \(p\) & \(q\) & \(p \to q\) & \(\lnot (p \to q)\) & \(\lnot (p \to q) \to p\) \\
        T     & T     & T           & F                   & T                         \\
        T     & F     & F           & T                   & T                         \\
        F     & T     & T           & F                   & T                         \\
        F     & F     & T           & F                   & T
    \end{tabular}
    \caption{Truth Table for 11(e)}
\end{table}

\begin{table}[ht]
    \centering
    \begin{tabular}{c|c|c|c|c|c}
        \(p\) & \(q\) & \(p \to q\) & \(\lnot(p \to q)\) & \(\lnot q\) & \(\lnot(p \to q) \to \lnot q\) \\
        T     & T     & T           & F                  & F           & T                              \\
        T     & F     & F           & T                  & T           & T                              \\
        F     & T     & T           & F                  & F           & T                              \\
        F     & F     & T           & F                  & T           & T
    \end{tabular}
    \caption{Truth Table for 11(f)}
\end{table}

\section*{Exercise 12}
Show that each of these conditional statements is a tautology by using truth tables.
\begin{enumerate}
    \item[\textbf{a)}] \([\lnot p \land (p \lor q)] \to q\)
    \item[\textbf{b)}] \([(p \to q) \land (q \to r)] \to (p \to r)\)
    \item[\textbf{c)}] \([p \land (p \to q)] \to q\)
    \item[\textbf{d)}] \([(p \lor q) \land (p \to r) \land (q \to r)] \to r\)
\end{enumerate}

\noindent
\textbf{Solution:}
\begin{table}[ht]
    \centering
    \begin{tabular}{c|c|c|c|c|c}
        \(p\) & \(q\) & \(\lnot p\) & \(p \lor q\) & \(\lnot p \land (p \lor q)\) & \([\lnot p \land (p \lor q)] \to q\) \\
        T     & T     & F           & T            & F                            & T                                    \\
        T     & F     & F           & T            & F                            & T                                    \\
        F     & T     & T           & T            & T                            & T                                    \\
        F     & F     & T           & F            & F                            & T
    \end{tabular}
    \caption{Truth Table for 12(a)}
\end{table}

\begin{table}[ht]
    \centering
    \begin{tabular}{c|c|c|c|c|c|c|c}
        \(p\) & \(q\) & \(r\) & \(p \to q\) & \(q \to r\) & \((p \to q) \land (q \to r)\) & \(p \to r\) & \([(p \to q) \land (q \to r)] \to (p \to r)\) \\
        T     & T     & T     & T           & T           & T                             & T           & T                                             \\
        T     & T     & F     & T           & F           & F                             & F           & T                                             \\
        T     & F     & T     & F           & T           & F                             & T           & T                                             \\
        T     & F     & F     & F           & T           & F                             & F           & T                                             \\
        F     & T     & T     & T           & T           & T                             & T           & T                                             \\
        F     & T     & F     & T           & F           & F                             & T           & T                                             \\
        F     & F     & T     & T           & T           & T                             & T           & T                                             \\
        F     & F     & F     & T           & T           & T                             & T           & T
    \end{tabular}
    \caption{Truth Table for 12(b)}
\end{table}

\begin{table}[ht]
    \centering
    \begin{tabular}{c|c|c|c|c}
        \(p\) & \(q\) & \(p \to q\) & \(p \land (p \to q)\) & \([p \land (p \to q)] \to q\) \\
        T     & T     & T           & T                     & T                             \\
        T     & F     & F           & F                     & T                             \\
        F     & T     & T           & F                     & T                             \\
        F     & F     & T           & F                     & T
    \end{tabular}
    \caption{Truth Table for 12(c)}
\end{table}

\begin{table}[ht]
    \centering
    \begin{tabular}{c|c|c|c|c|c|c|c}
        \(p\) & \(q\) & \(r\) & \(p \lor q\) & \(p \to r\) & \(q \to r\) & \((p \lor q) \land (p \to r) \land (q \to r)\) & \([(p \lor q) \land (p \to r) \land (q \to r)] \to r\) \\
        T     & T     & T     & T            & T           & T           & T                                              & T                                                      \\
        T     & T     & F     & T            & F           & F           & F                                              & T                                                      \\
        T     & F     & T     & T            & T           & T           & T                                              & T                                                      \\
        T     & F     & F     & T            & F           & T           & F                                              & T                                                      \\
        F     & T     & T     & T            & T           & T           & T                                              & T                                                      \\
        F     & T     & F     & F            & T           & F           & F                                              & T                                                      \\
        F     & F     & T     & F            & T           & T           & F                                              & T                                                      \\
        F     & F     & F     & F            & T           & T           & F                                              & T
    \end{tabular}
    \caption{Truth Table for 12(d)}
\end{table}

\section*{Exercise 13}
Show that each conditional statement in Exercise 11 is a tautology using the fact that a conditional statement is false exactly when the hypothesis is true and the conclusion is false. (Do not use truth tables.)

\noindent
\textbf{Solution:}
\begin{enumerate}
    \item[\textbf{a)}] Suppose \(p \land q\) is true and \(p\) is false. This is a contradiction, because for \(p \land q\) to be true, \(p\) must be true. Therefore, \((p \land q) \to p\) is a tautology.
    \item[\textbf{b)}] Suppose that \(p\) is true and \(p \lor q\) is false. This is a contradiction, because for \(p \lor q\) to be false, \(p\) must not be true and \(q\) must not be true. Therefore, \(p \to (p \lor q)\) is a tautology.
    \item[\textbf{c)}] Suppose that \(\lnot p\) is true and \(p \to q\) is false. This is a contradiction, because for \(p \to q\) to be false, \(p\) must be true and \(q\) must be false. Therefore, \(\lnot p \to (p \to q)\) is a tautology.
    \item[\textbf{d)}] Suppose that \(p \land q\) is true and \(p \to q\) is false. This is a contradiction, because for \(p \to q\) to be false, \(q\) must be false and \(p\) must be true, but for \(p \land q\) to be true, \(p\) and \(q\) must be true. Therefore, \((p \land q) \to (p \to q)\) is a tautology.
    \item[\textbf{e)}] Suppose that \(\lnot(p \to q)\) is true and \(p\) is false. This is a contradiction, because for \(\lnot(p \to q)\) to be true, \(p \to q\) must be false. For \(p \to q\) to be false, \(p\) must be true and \(q\) must be false. Therefore, \(\lnot(p \to q) \to p\) is a tautology.
    \item[\textbf{f)}] Suppose that \(\lnot(p \to q)\) is true and \(\lnot q\) is false. This is a contradiction, because for \(\lnot(p \to q\) to be true, \(p \to q\) must be false. For \(p \to q\) to be false, \(p\) must be true and \(q\) must be false. For \(\lnot q\) to be false, \(q\) must be true. Therefore, \(\lnot(p \to q) \to \lnot q\) is a tautology.
\end{enumerate}

\section*{Exercise 14}
Show that each conditional statement in Exercise 12 is a tautology using the fact that a conditional statement is false exactly when the hypothesis is true and the conclusion is false. (Do not use truth tables.)

\noindent
\textbf{Solution:}
\begin{enumerate}
    \item[\textbf{a)}] Suppose that \(\lnot p \land (p \lor q)\) is true and \(q\) is false. This is a contradiction, because then \(p \lor q\) and \(\lnot p\) would both have to be true, implying that \(p\) is false and \(q\) is true. Therefore, \([\lnot p \land (p \lor q)] \to q\) is a tautology.
    \item[\textbf{b)}] Suppose that \((p \to q) \land (q \to r)\) is true and \(p \to r\) is false. This is a contradiction, because for \(p \to r\) to be false, \(p\) must be true and \(r\) must be false. For \((p \to q) \land (q \to r)\) to be true, \(p \to q\) and \(q \to r\) must be true, but this requires differing truth values for \(q\). Therefore, \([(p \to q) \land (q \to r)] \to (p \to r)\) is a tautology.
    \item[\textbf{c)}] Suppose that \(p \land (p \to q)\) is true and \(q\) is false. This is a contradiction, because for \(p \land (p \to q)\) to be true, \(p\) must be true and \(q\) must not be false. Therefore, \([p \land (p \to q)] \to q\) is a tautology.
    \item[\textbf{d)}] Suppose that \((p \lor q) \land (p \to r) \land (q \to r)\) is true and \(r\) is false. This is a contradiction, because for \((p \lor q) \land (p \to r) \land (q \to r)\) to be true, \(p \lor q\), \(p \to r\), and \(q \to r\) must all be true. For \(p \lor q\) to be true, \(p\) or \(q\) must be true. However, for both \(p \to r\) and \(q \to r\) to be true, \(p\) and \(q\) must both be false since \(r\) is false. Therefore,  \([(p \lor q) \land (p \to r) \land (q \to r)] \to r\) is a tautology.
\end{enumerate}

\section*{Exercise 15}
Show that each conditional statement in Exercise 11 is a tautology by applying a chain of logical identities as in Example 8. (Do not use truth tables.)

\noindent
\textbf{Solution:}
\begin{enumerate}
    \item[\textbf{a)}] 
    \begin{gather*}
        (p \land q) \to p \equiv \lnot(p \land q) \lor p \ \textbf{by the conditional-disjunction equivalence} \\
        \equiv (\lnot p \lor \lnot q) \lor p \ \textbf{by a De Morgan's law} \\
        \equiv \lnot p \lor \lnot q \lor p \ \textbf{by an associative law} \\
        \equiv p \lor \lnot p \lor \lnot q \ \textbf{by a commutative law} \\
        \equiv \textbf{T} \lor \lnot q \ \textbf{by a negation law} \\
        \equiv \textbf{T} \ \textbf{by a domination law}
    \end{gather*}
    
    \item[\textbf{b)}]
    \begin{gather*}
        p \to (p \lor q) \\
        \equiv \lnot p \lor (p \lor q) \ \textbf{by the conditional-disjunction equivalence} \\
        \equiv (\lnot p \lor p) \lor q \ \textbf{by an associative law} \\
        \equiv (p \lor \lnot p) \lor q \ \textbf{by a commutative law} \\
        \equiv \textbf{T} \lor q \ \textbf{by a negation law} \\
        \equiv \textbf{T} \ \textbf{by a domination law}
    \end{gather*}
    
    \item[\textbf{c)}]
    \begin{gather*}
        \lnot p \to (p \to q) \\
        \equiv p \lor (p \to q) \ \textbf{by the conditional-disjunction equivalence} \\
        \equiv p \lor (\lnot p \lor q) \ \textbf{by the conditional-disjunction equivalence} \\
        \equiv (p \lor \lnot p) \lor q \ \textbf{by an associative law} \\
        \equiv \textbf{T} \lor q \ \textbf{by a negation law} \\
        \equiv \textbf{T} \ \textbf{by a domination law}
    \end{gather*}
    
    \item[\textbf{d)}] 
    \begin{gather*}
        (p \land q) \to (p \to q) \\
        \equiv \lnot (p \land q) \lor (p \to q) \ \textbf{by the conditional-disjunction equivalence} \\
        \equiv \lnot(p \land q) \lor (\lnot p \lor q) \ \textbf{by the conditional-disjunction equivalence} \\
        \equiv (\lnot p \lor \lnot q) \lor (\lnot p \lor q) \ \textbf{by a De Morgan's law} \\
        \equiv (\lnot p \lor \lnot p) \lor (\lnot q \lor q) \ \textbf{by an associative law} \\
        \equiv (\lnot p \lor \lnot p) \lor (q \lor \lnot q) \ \textbf{by a commutative law} \\
        \equiv \lnot p \lor (q \lor \lnot q) \ \textbf{by a idempotent law} \\
        \equiv \lnot p \lor \textbf{T} \ \textbf{by a negation law} \\
        \equiv \textbf{T} \ \textbf{by a domination law}
    \end{gather*}

    
    \item[\textbf{e)}]
    \begin{gather*}
        \lnot(p \to q) \to p \\
        \equiv (p \to q) \lor p \ \textbf{by the conditional-disjunction equivalence} \\
        \equiv (\lnot p \lor q) \lor p \ \textbf{by the conditional-disjunction equivalence} \\
        \equiv (\lnot p \lor p) \lor q \ \textbf{by an associative law} \\
        \equiv \textbf{T} \lor q \ \textbf{by a negation law} \\
        \equiv \textbf{T} \ \textbf{by a domination law}
    \end{gather*}
    
    \item[\textbf{f)}]
    \begin{gather*}
        \lnot(p \to q) \to \lnot q \\
        \equiv (p \to q) \lor \lnot q \ \textbf{by the conditional-disjunction equivalence} \\
        \equiv (\lnot p \lor q) \lor \lnot q  \ \textbf{by the conditional-disjunction equivalence} \\
        \equiv \lnot p \lor (q \lor \lnot q) \ \textbf{by an associative law} \\
        \equiv \lnot p \lor \textbf{T} \ \textbf{by a negation law} \\
        \equiv \textbf{T} \ \textbf{by a domination law}
    \end{gather*}
\end{enumerate}

\section*{Exercise 16}
Show that each conditional statement in Exercise 12 is a tautology by applying a chain of logical identities as in Example 8. (Do not use truth tables.)

\noindent
\textbf{Solution:}
\begin{enumerate}
    \item[\textbf{a)}]
    \begin{gather*}
        [\lnot p \land (p \lor q)] \to q \\
        \equiv \lnot [\lnot p \land (p \lor q)] \lor q \ \textbf{by the conditional-disjunction equivalence} \\
        \equiv [p \lor \lnot(p \lor q)] \lor q \ \textbf{by a De Morgan's law} \\
        \equiv (p \lor q) \lor \lnot(p \lor q) \ \textbf{by commutative and associative laws} \\
        \equiv \textbf{T} \ \textbf{by a negation law}
    \end{gather*}
    
    \item[\textbf{b)}]
    \begin{gather*}
        [(p \to q) \land (q \to r)] \to (p \to r) \\
        \equiv \lnot[(p \to q) \land (q \to r)] \lor (p \to r) \ \textbf{by the conditional-disjunction equivalence} \\
        \equiv [\lnot(p \to q) \lor \lnot(q \to r)] \lor (p \to r) \ \textbf{by a De Morgan's law} \\
        \equiv [(p \land \lnot q) \lor (q \land \lnot r)] \lor (p \to r) \ \textbf{by the conditional-disjunction equivalence and a De Morgan's law} \\
        \equiv (p \land \lnot q) \lor (q \land \lnot r) \lor (\lnot p \lor q) \ \textbf{by the conditional-disjunction equivalence and an associative law} \\
        \equiv q \lor (q \land \lnot r) \lor (p \land \lnot q) \lor \lnot p \ \textbf{by a commutative law and an associative law} \\
        \equiv (\lnot p \lor q) \lor (p \land \lnot q) \ \textbf{by an absorption law} \\
        \equiv \lnot(p \land \lnot q) \lor (p \land \lnot q) \ \textbf{by a De Morgan's law} \\
        \equiv \textbf{T} \ \textbf{by a negation law}
    \end{gather*}
    
    \item[\textbf{c)}]
    \begin{gather*}
        [p \land (p \to q)] \to q \\
        \equiv \lnot[p \land (p \to q)] \lor q \ \textbf{by the conditional-disjunction equivalence} \\
        \equiv \lnot[p \land (\lnot p \lor q)] \lor q \ \textbf{by the conditional-disjunction equivalence} \\
        \equiv [\lnot p \lor \lnot(\lnot p \lor q)] \lor q \ \textbf{by a De Morgan's law} \\
        \equiv \lnot p \lor \lnot(\lnot p \lor q)] \lor q \ \textbf{by an associative law} \\
        \equiv \lnot p \lor q \lor \lnot(\lnot p \lor q) \ \textbf{by a commutative law} \\
        \equiv (\lnot p \lor q) \lor \lnot(\lnot p \lor q) \ \textbf{by an associative law} \\
        \equiv \textbf{T} \ \textbf{by a negation law}
    \end{gather*}
    
    \item[\textbf{d)}]
    \begin{gather*}
        [(p \lor q) \land (p \to r) \land (q \to r)] \to r \\
        \equiv [(p \lor q) \land (\lnot p \lor r) \land (\lnot q \lor r)] \to r \ \textbf{by the conditional-disjunction equivalence} \\
        \equiv [(p \lor q) \land (r \lor (\lnot p \land \lnot q))] \to r \ \textbf{by a distributive law} \\
        \equiv \lnot[(p \lor q) \land (r \lor (\lnot p \land \lnot q))] \lor r \ \textbf{by the conditional-disjunction equivalence} \\
        \equiv \lnot[(p \lor q) \land (r \lor \lnot(p \lor q))] \lor r \ \textbf{by a De Morgan's law} \\
        \equiv [\lnot(p \lor q) \lor \lnot(r \lor \lnot(p \lor q))] \lor r \ \textbf{by a De Morgan's law} \\
        \equiv \lnot(p \lor q) \lor \lnot(r \lor \lnot(p \lor q)) \lor r \ \textbf{by an associative law} \\
        \equiv r \lor \lnot(p \lor q) \lor \lnot(r \lor \lnot(p \lor q)) \ \textbf{by a commutative law} \\
        \equiv (r \lor \lnot(p \lor q)) \lor \lnot(r \lor \lnot(p \lor q)) \ \textbf{by an associative law} \\
        \equiv \textbf{T} \ \textbf{by a negation law}
    \end{gather*}
\end{enumerate}

\section*{Exercise 17}
Use truth tables to verify the absorption laws.
\begin{enumerate}
    \item[\textbf{a)}] \(p \lor (p \land q) \equiv p\)
    \item[\textbf{b)}] \(p \land (p \lor q) \equiv p\)
\end{enumerate}

\noindent
\textbf{Solution:}
\begin{table}[ht]
    \centering
    \begin{tabular}{c|c|c|c}
        \(p\) & \(q\) & \(p \land q\) & \(p \lor (p \land q)\) \\
        T     & T     & T             & T                      \\
        T     & F     & F             & T                      \\
        F     & T     & F             & F                      \\
        F     & F     & F             & F
    \end{tabular}
    \caption{Truth Table for 17(a)}
\end{table}

\begin{table}[ht]
    \centering
    \begin{tabular}{c|c|c|c}
        \(p\) & \(q\) & \(p \lor q\) & \(p \land (p \lor q)\) \\
        T     & T     & T            & T                      \\
        T     & F     & T            & T                      \\
        F     & T     & T            & F                      \\
        F     & F     & F            & F
    \end{tabular}
    \caption{Truth Table for 17(b)}
\end{table}

\section*{Exercise 18}
Determine whether \((\lnot p \land(p \to q)) \to \lnot q\) is a tautology.

\noindent
\textbf{Solution:}
\begin{table}[ht]
    \centering
    \begin{tabular}{c|c|c|c|c|c|c}
        \(p\) & \(q\) & \(p \to q\) & \(\lnot p\) & \(\lnot p \land (p \to q)\) & \(\lnot q\) & \((\lnot p \land (p \to q)) \to \lnot q\) \\
        T     & T     & T           & F           & F                           & F           & T                                         \\
        T     & F     & F           & F           & F                           & T           & T                                         \\
        F     & T     & T           & T           & T                           & F           & F                                         \\
        F     & F     & T           & T           & T                           & T           & T
    \end{tabular}
    \caption{Truth Table for 18}
    \label{Truth Table for 18}
\end{table}

As seen in Table \ref{Truth Table for 18}, \((\lnot p \land(p \to q)) \to \lnot q\) is not a tautology.

\section*{Exercise 19}
Determine whether \((\lnot q \land (p \to q)) \to \lnot p\) is a tautology.

\noindent
\textbf{Solution:}
\begin{table}[ht]
    \centering
    \begin{tabular}{c|c|c|c|c|c|c}
        \(p\) & \(q\) & \(\lnot q\) & \(p \to q\) & \(\lnot q \land (p \to q)\) & \(\lnot p\) & \((\lnot q \land (p \to q)) \to \lnot p\) \\
        T     & T     & F           & T           & F                           & F           & T                                         \\
        T     & F     & T           & F           & F                           & F           & T                                         \\
        F     & T     & F           & T           & F                           & T           & T                                         \\
        F     & F     & T           & T           & T                           & T           & T
    \end{tabular}
    \caption{Truth Table for 19}
    \label{Truth Table for 19}
\end{table}

As seen in Table \ref{Truth Table for 19}, \((\lnot q \land (p \to q)) \to \lnot p\) is a tautology.

Each of Exercises 20-32 asks you to show that two compound propositions are logically equivalent. To do this, either show that both sides are true, or that both sides are false, for exactly the same combinations of truth values of the propositional variables in these expressions (whichever is easier).

\section*{Exercise 20}
Show that \(p \leftrightarrow q\) and \((p \land q) \lor (\lnot p \land \lnot q)\) are logically equivalent.

\noindent
\textbf{Solution:}
\begin{table}[ht]
    \centering
    \begin{tabular}{c|c|c|c|c|c|c|c}
        \(p\) & \(q\) & \(p \leftrightarrow q\) & \(p \land q\) & \(\lnot p\) & \(\lnot q\) & \(\lnot p \land \lnot q\) & \((p \land q) \lor (\lnot p \land \lnot q)\) \\
        T     & T     & T                       & T             & F           & F           & F                         & T                                            \\
        T     & F     & F                       & F             & F           & T           & F                         & F                                            \\
        F     & T     & F                       & F             & T           & F           & F                         & F                                            \\
        F     & F     & T                       & F             & T           & T           & T                         & T
    \end{tabular}
    \caption{Truth Table for 20}
\end{table}

\section*{Exercise 21}
Show that \(\lnot(p \leftrightarrow q)\) and \(p \leftrightarrow \lnot q\) are logically equivalent.

\noindent
\textbf{Solution:}
\begin{table}[ht]
    \centering
    \begin{tabular}{c|c|c|c|c}
        \(p\) & \(q\) & \(\lnot(p \leftrightarrow q)\) & \(\lnot q\) & \(p \leftrightarrow \lnot q\) \\
        T     & T     & F                              & F           & F                             \\
        T     & F     & T                              & T           & T                             \\
        F     & T     & T                              & F           & T                             \\
        F     & F     & F                              & T           & F
    \end{tabular}
    \caption{Truth Table for 21}
\end{table}

\section*{Exercise 22}
Show that \(p \to q\) and \(\lnot q \to \lnot p\) are logically equivalent.

\noindent
\textbf{Solution:}
\begin{table}[ht]
    \centering
    \begin{tabular}{c|c|c|c|c|c}
        \(p\) & \(q\) & \(p \to q\) & \(\lnot q\) & \(\lnot p\) & \(\lnot q \to \lnot p\) \\
        T     & T     & T           & F           & F           & T                       \\
        T     & F     & F           & T           & F           & F                       \\
        F     & T     & T           & F           & T           & T                       \\
        F     & F     & T           & T           & T           & T
    \end{tabular}
    \caption{Truth Table for 22}
\end{table}

\section*{Exercise 23}
Show that \(\lnot p \leftrightarrow q\) and \(p \leftrightarrow \lnot q\) are logically equivalent.

\noindent
\textbf{Solution:}
\begin{table}[ht]
    \centering
    \begin{tabular}{c|c|c|c|c|c}
        \(p\) & \(q\) & \(\lnot p\) & \(\lnot p \leftrightarrow q\) & \(\lnot q\) & \(p \leftrightarrow \lnot q\) \\
        T     & T     & F           & F                             & F           & F                             \\
        T     & F     & F           & T                             & T           & T                             \\
        F     & T     & T           & T                             & F           & T                             \\
        F     & F     & T           & F                             & T           & F
    \end{tabular}
    \caption{Truth Table for 23}
\end{table}

\section*{Exercise 24}
Show that \(\lnot(p \oplus q)\) and \(p \leftrightarrow q\) are logically equivalent.

\noindent
\textbf{Solution:}
\begin{table}[ht]
    \centering
    \begin{tabular}{c|c|c|c|c}
        \(p\) & \(q\) & \(p \oplus q\) & \(\lnot(p \oplus q)\) & \(p \leftrightarrow q\) \\
        T     & T     & F              & T                     & T                       \\
        T     & F     & T              & F                     & F                       \\
        F     & T     & T              & F                     & F                       \\
        F     & F     & F              & T                     & T
    \end{tabular}
    \caption{Truth Table for 24}
\end{table}

\section*{Exercise 25}
Show that \(\lnot(p \leftrightarrow q)\) and \(\lnot p \leftrightarrow q\) are logically equivalent.

\noindent
\textbf{Solution:}
\begin{table}[ht]
    \centering
    \begin{tabular}{c|c|c|c|c}
        \(p\) & \(q\) & \(\lnot(p \leftrightarrow q)\) & \(\lnot p\) & \(\lnot p \leftrightarrow q\) \\
        T     & T     & F                              & F           & F                             \\
        T     & F     & T                              & F           & T                             \\
        F     & T     & T                              & T           & T                             \\
        F     & F     & F                              & T           & F
    \end{tabular}
    \caption{Truth Table for 25}
\end{table}

\section*{Exercise 26}
Show that \((p \to q) \land (p \to r)\) and \(p \to (q \land r)\) are logically equivalent.

\noindent
\textbf{Solution:}
\begin{table}[ht]
    \centering
    \begin{tabular}{c|c|c|c|c|c|c|c}
        \(p\) & \(q\) & \(r\) & \(p \to q\) & \(p \to r\) & \((p \to q) \land (p \to r)\) & \(q \land r\) & \(p \to (q \land r)\) \\
        T     & T     & T     & T           & T           & T                             & T             & T                     \\
        T     & T     & F     & T           & F           & F                             & F             & F                     \\
        T     & F     & T     & F           & T           & F                             & F             & F                     \\
        T     & F     & F     & F           & F           & F                             & F             & F                     \\
        F     & T     & T     & T           & T           & T                             & T             & T                     \\
        F     & T     & F     & T           & T           & T                             & F             & T                     \\
        F     & F     & T     & T           & T           & T                             & F             & T                     \\
        F     & F     & F     & T           & T           & T                             & F             & T                     \\
    \end{tabular}
    \caption{Truth Table for 26}
\end{table}

\printbibliography

\end{document}