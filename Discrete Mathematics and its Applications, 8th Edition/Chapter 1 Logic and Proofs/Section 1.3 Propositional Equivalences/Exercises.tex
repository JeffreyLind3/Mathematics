%%%%%%%%%%%%%%%%%%%%%%%%%%%%%%%%%%%%%%%%%%%%%%%%%%%%%%%%%%%%%%%%
%%%%%%%%%%%%%%%%%%%%%%%%%%% Metadata %%%%%%%%%%%%%%%%%%%%%%%%%%%
%%%%%%%%%%%%%%%%%%%%%%%%%%%%%%%%%%%%%%%%%%%%%%%%%%%%%%%%%%%%%%%%
\documentclass{Axon}

\title{Discrete Mathematics and its Applications, 8th Edition - Chapter 1 The Foundations: Logic and Proofs - Section 1.3 Propositional Equivalences - Exercises}

\authors{
    \addauthor{Jeffrey G. Lind III}{jeffrey@jeffreylind.dev}
}

\addbibresource{Bibliography.bib}
%%%%%%%%%%%%%%%%%%%%%%%%%%%%%%%%%%%%%%%%%%%%%%%%%%%%%%%%%%%%%%%%
%%%%%%%%%%%%%%%%%%%%%%%%%%%%% Paper %%%%%%%%%%%%%%%%%%%%%%%%%%%%
%%%%%%%%%%%%%%%%%%%%%%%%%%%%%%%%%%%%%%%%%%%%%%%%%%%%%%%%%%%%%%%%
\begin{document}
\maketitle
\makeauthor
%%%%%%%%%%%%%%%%%%%%%%%%%%%%%%%%%%%%%%%%%%%%%%%%%%%%%%%%%%%%%%%%
%%%%%%%%%%%%%%%%%%%%%%%%%%% Abstract %%%%%%%%%%%%%%%%%%%%%%%%%%%
%%%%%%%%%%%%%%%%%%%%%%%%%%%%%%%%%%%%%%%%%%%%%%%%%%%%%%%%%%%%%%%%
\begin{abstract}
Notes on Discrete Mathematics and its Applications, 8th Edition - Chapter 1 The Foundations: Logic and Proofs - Section 1.3 Propositional Equivalences - Exercises \cite{Rosen}.
\end{abstract}
%%%%%%%%%%%%%%%%%%%%%%%%%%%%%%%%%%%%%%%%%%%%%%%%%%%%%%%%%%%%%%%%
%%%%%%%%%%%%%%%%%%%%%%%%%%% Section 1 %%%%%%%%%%%%%%%%%%%%%%%%%%
%%%%%%%%%%%%%%%%%%%%%%%%%%%%%%%%%%%%%%%%%%%%%%%%%%%%%%%%%%%%%%%%
\section*{Exercise 1}
Use truth tables to verify these equivalences.

\begin{enumerate}
    \item[\textbf{a)}] \(p \land \textbf{T} \equiv p\)
    \item[\textbf{b)}] \(p \lor \textbf{F} \equiv p\)
    \item[\textbf{c)}] \(p \land \textbf{F} \equiv \textbf{F}\)
    \item[\textbf{d)}] \(p \lor \textbf{T} \equiv \textbf{T}\)
    \item[\textbf{e)}] \(p \lor p \equiv p\)
    \item[\textbf{f)}] \(p \land p \equiv p\)
\end{enumerate}

\noindent
\textbf{Solution:}
\begin{table}[h]
    \centering
    \begin{tabular}{c|c|c}
        \(p\) & \textbf{T} & \(p \land \textbf{T}\) \\
        T     & T          & T                      \\
        F     & T          & F
    \end{tabular}
    \caption{Truth Table for 1(a)}
\end{table}

\begin{table}[h]
    \centering
    \begin{tabular}{c|c|c}
        \(p\) & \textbf{F} & \(p \lor \textbf{F}\) \\
        T     & F          & T                     \\
        F     & F          & F
    \end{tabular}
    \caption{Truth Table for 1(b)}
\end{table}

\begin{table}[h]
    \centering
    \begin{tabular}{c|c|c}
        \(p\) & \textbf{F} & \(p \land \textbf{F}\) \\
        T     & F          & F                      \\
        F     & F          & F
    \end{tabular}
    \caption{Truth Table for 1(c)}
\end{table}

\begin{table}[ht]
    \centering
    \begin{tabular}{c|c|c}
        \(p\) & \textbf{T} & \(p \lor \textbf{T}\) \\
        T     & T          & T                     \\
        F     & T          & T
    \end{tabular}
    \caption{Truth Table for 1(d)}
\end{table}

\begin{table}[ht]
    \centering
    \begin{tabular}{c|c}
        \(p\) & \(p \lor p\) \\
        T     & T            \\
        F     & F
    \end{tabular}
    \caption{Truth Table for 1(e)}
\end{table}

\begin{table}[ht]
    \centering
    \begin{tabular}{c|c}
        \(p\) & \(p \land p\) \\
        T     & T             \\
        F     & F
    \end{tabular}
    \caption{Truth Table for 1(f)}
\end{table}

\section*{Exercise 2}
Show that \(\lnot(\lnot p)\) and \(p\) are logically equivalent

\noindent
\textbf{Solution:}
\begin{table}[ht]
    \centering
    \begin{tabular}{c|c|c}
        \(p\) & \(\lnot p\) & \(\lnot(\lnot p)\) \\
        T     & F           & T                  \\
        F     & T           & F
    \end{tabular}
    \caption{Truth Table for 2}
\end{table}

\section*{Exercise 3}
Use truth tables to verify the commutative laws

\begin{enumerate}
    \item[\textbf{a)}] \(p \lor q \equiv q \lor p\)
    \item[\textbf{b)}] \(p \land q \equiv q \land p\)
\end{enumerate}

\noindent
\textbf{Solution:}

\begin{table}[ht]
    \centering
    \begin{tabular}{c|c|c|c}
        \(p\) & \(q\) & \(p \lor q\) & \(q \lor p\) \\
        T     & T     & T            & T            \\
        T     & F     & T            & T            \\
        F     & T     & T            & T            \\
        F     & F     & F            & F
    \end{tabular}
    \caption{Truth Table for 3(a)}
\end{table}

\begin{table}[ht]
    \centering
    \begin{tabular}{c|c|c|c}
        \(p\) & \(q\) & \(p \land q\) & \(q \land p\) \\
        T     & T     & T             & T             \\
        T     & F     & F             & F             \\
        F     & T     & F             & F             \\
        F     & F     & F             & F
    \end{tabular}
    \caption{Truth Table for 3(b)}
\end{table}

\section*{Exercise 4}
Use truth tables to verify the associative laws

\begin{enumerate}
    \item[\textbf{a)}] \((p \lor q) \lor r \equiv p \lor (q \lor r)\).
    \item[\textbf{b)}] \((p \land q) \land r \equiv p \land (q \land r)\)
\end{enumerate}

\noindent
\textbf{Solution:}

\begin{table}[ht]
    \centering
    \begin{tabular}{c|c|c|c|c|c|c}
        \(p\) & \(q\) & \(r\) & \(p \lor q\) & \((p \lor q) \lor r\) & \(q \lor r\) & \(p \lor (q \lor r)\) \\
        T     & T     & T     & T            & T                     & T            & T                     \\
        T     & T     & F     & T            & T                     & T            & T                     \\
        T     & F     & T     & T            & T                     & T            & T                     \\
        T     & F     & F     & T            & T                     & F            & T                     \\
        F     & T     & T     & T            & T                     & T            & T                     \\
        F     & T     & F     & T            & T                     & T            & T                     \\
        F     & F     & T     & F            & T                     & T            & T                     \\
        F     & F     & F     & F            & F                     & F            & F
    \end{tabular}
    \caption{Truth Table for 4(a)}
\end{table}

\begin{table}[ht]
    \centering
    \begin{tabular}{c|c|c|c|c|c|c}
        \(p\) & \(q\) & \(r\) & \(p \land q\) & \((p \land q) \land r\) & \(q \land r\) & \(p \land (q \land r)\) \\
        T     & T     & T     & T             & T                       & T             & T                       \\
        T     & T     & F     & T             & F                       & F             & F                       \\
        T     & F     & T     & F             & F                       & F             & F                       \\
        T     & F     & F     & F             & F                       & F             & F                       \\
        F     & T     & T     & F             & F                       & T             & F                       \\
        F     & T     & F     & F             & F                       & F             & F                       \\
        F     & F     & T     & F             & F                       & F             & F                       \\
        F     & F     & F     & F             & F                       & F             & F
    \end{tabular}
    \caption{Truth Table for 4(b)}
\end{table}

\section*{Exercise 5}
Use a truth table to verify the distributive law
\begin{equation}
    p \land (q \lor r) \equiv (p \land q) \lor (p \land r)
\end{equation}

\noindent
\textbf{Solution:}

\begin{table}[ht]
    \centering
    \begin{tabular}{c|c|c|c|c|c|c|c}
        \(p\) & \(q\) & \(r\) & \(q \lor r\) & \(p \land (q \lor r)\) & \(p \land q\) & \(p \land r\) & \((p \land q) \lor (p \land r)\) \\
        T     & T     & T     & T            & T                      & T             & T             & T                                \\
        T     & T     & F     & T            & T                      & T             & F             & T                                \\
        T     & F     & T     & T            & T                      & F             & T             & T                                \\
        T     & F     & F     & F            & F                      & F             & F             & F                                \\
        F     & T     & T     & T            & F                      & F             & F             & F                                \\
        F     & T     & F     & T            & F                      & F             & F             & F                                \\
        F     & F     & T     & T            & F                      & F             & F             & F                                \\
        F     & F     & F     & F            & F                      & F             & F             & F
    \end{tabular}
    \caption{Truth Table for 5}
\end{table}

\section*{Exercise 6}
Use a truth table to verify the first De Morgan law
\begin{equation}
    \lnot (p \land q) \equiv \lnot p \lor \lnot q
\end{equation}

\noindent
\textbf{Solution:}

\begin{table}[ht]
    \centering
    \begin{tabular}{c|c|c|c|c|c|c}
        \(p\) & \(q\) & \(p \land q\) & \(\lnot (p \land q)\) & \(\lnot p\) & \(\lnot q\) & \(\lnot p \lor \lnot q\) \\
        T     & T     & T             & F                     & F           & F           & F                        \\
        T     & F     & F             & T                     & F           & T           & T                        \\
        F     & T     & F             & T                     & T           & F           & T                        \\
        F     & F     & F             & T                     & T           & T           & T
    \end{tabular}
    \caption{Truth Table for 6}
\end{table}

\section*{Exercise 7}
Use De Morgan's laws to find the negation of each of the following statements.
\begin{enumerate}
    \item[\textbf{a)}] Jan is rich and happy.
    \item[\textbf{b)}] Carlos will bicycle or run tomorrow
    \item[\textbf{c)}] Mei walks or takes the bus to class.
    \item[\textbf{d)}] Ibrahim is smart and hard working.
\end{enumerate}

\noindent
\textbf{Solution:}
\begin{enumerate}
    \item[\textbf{a)}] Jan is not rich or not happy.
    \item[\textbf{b)}] Carlos will not bicycle and will not run tomorrow.
    \item[\textbf{c)}] Mei does not walk and does not take the bus to class.
    \item[\textbf{d)}] Ibrahim is not smart or not hard working.
\end{enumerate}

\section*{Exercise 8}
Use De Morgan's laws to find the negation of each of the following statements.
\begin{enumerate}
    \item[\textbf{a)}] Kwame will take a job in industry or go to graduate school.
    \item[\textbf{b)}] Yoshiko knows Java and calculus.
    \item[\textbf{c)}] James is young and strong.
    \item[\textbf{d)}] Rita will move to Oregon or Washington.
\end{enumerate}

\noindent
\textbf{Solution:}
\begin{enumerate}
    \item[\textbf{a)}] Kwame will not take a job in industry and Kwame will not go to graduate school.
    \item[\textbf{b)}] Yoshiko does not know Java or Yoshiko does not know calculus.
    \item[\textbf{c)}] James is not young or James is not strong.
    \item[\textbf{d)}] Rita will not move to Oregon and Rita will not move to Washington.
\end{enumerate}

\section*{Exercise 9}
For each of these compound propositions, use the conditional-disjunction equivalence (Example 3) to find an equivalent compound proposition that does not involve conditionals.
\begin{enumerate}
    \item[\textbf{a)}] \(p \to \lnot q\)
    \item[\textbf{b)}] \((p \to q) \to r\)
    \item[\textbf{c)}] \((\lnot q \to p) \to (p \to \lnot q)\)
\end{enumerate}

\noindent
\textbf{Solution:}
The conditional-disjunction equivalence states that for propositional variables \(p\) and \(q\), \(p \to q \equiv \lnot p \lor q\). Therefore,
\begin{enumerate}
    \item[\textbf{a)}] \(p \to \lnot q \equiv \lnot p \lor \lnot q\)
    \item[\textbf{b)}] \((p \to q) \to r \equiv \lnot(\lnot p \lor q) \lor r \equiv (p \land \lnot q) \lor r\)
    \item[\textbf{c)}] \((\lnot q \to p) \to (p \to \lnot q) \equiv (\lnot q \land \lnot p) \lor (\lnot p \lor \lnot q) \equiv \lnot p \lor \lnot q\)
\end{enumerate}

\section*{Exercise 10}
For each of these compound propositions, use the conditional-disjunction equivalence (Example 3) to find an equivalent compound proposition that does not involve conditionals.
\begin{enumerate}
    \item[\textbf{a)}] \(\lnot p \to \lnot q\)
    \item[\textbf{b)}] \((p \lor q) \to \lnot p\)
    \item[\textbf{c)}] \((p \to \lnot q) \to (\lnot p \to q)\)
\end{enumerate}

\noindent
\textbf{Solution:}
The conditional-disjunction equivalence states that for propositional variables \(p\) and \(q\), \(p \to q \equiv \lnot p \lor q\). Therefore,
\begin{enumerate}
    \item[\textbf{a)}] \(\lnot p \to \lnot q \equiv p \lor \lnot q\)
    \item[\textbf{b)}] \((p \lor q) \to \lnot p \equiv \lnot(p \lor q) \lor \lnot p \equiv \lnot p\)
    \item[\textbf{c)}] \((p \to \lnot q) \to (\lnot p \to q) \equiv \lnot(p \to \lnot q) \lor (\lnot p \to q) \equiv \lnot(\lnot p \lor \lnot q) \lor (p \lor q) \equiv p \lor q\)
\end{enumerate}

\section*{Exercise 11}
Show that each of these conditional statements is a tautology by using truth tables.
\begin{enumerate}
    \item[\textbf{a)}] \((p \land q) \to p\)
    \item[\textbf{b)}] \(p \to (p \lor q)\)
    \item[\textbf{c)}] \(\lnot p \to (p \to q)\)
    \item[\textbf{d)}] \((p \land q) \to (p \to q)\)
    \item[\textbf{e)}] \(\lnot(p \to q) \to p\)
    \item[\textbf{f)}] \(\lnot(p \to q) \to \lnot q\)
\end{enumerate}

\noindent
\textbf{Solution:}

\begin{table}[ht]
    \centering
    \begin{tabular}{c|c|c|c}
        \(p\) & \(q\) & \(p \land q\) & \((p \land q) \to p\) \\
        T     & T     & T             & T                     \\
        T     & F     & F             & T                     \\
        F     & T     & F             & T                     \\
        F     & F     & F             & T
    \end{tabular}
    \caption{Truth Table for 11(a)}
\end{table}

\begin{table}[ht]
    \centering
    \begin{tabular}{c|c|c|c}
        \(p\) & \(q\) & \(p \lor q\) & \(p \to (p \lor q)\) \\
        T     & T     & T            & T                     \\
        T     & F     & T            & T                    \\
        F     & T     & T            & T                    \\
        F     & F     & F            & T
    \end{tabular}
    \caption{Truth Table for 11(b)}
\end{table}

\begin{table}[ht]
    \centering
    \begin{tabular}{c|c|c|c|c}
        \(p\) & \(q\) & \(\lnot p\) & \(p \to q\) & \(\lnot p \to (p \to q)\) \\
        T     & T     & F           & T           & T                         \\
        T     & F     & F           & F           & T                         \\
        F     & T     & T           & T           & T                         \\
        F     & F     & T           & T           & T
    \end{tabular}
    \caption{Truth Table for 11(c)}
\end{table}

\begin{table}[ht]
    \centering
    \begin{tabular}{c|c|c|c|c}
        \(p\) & \(q\) & \(p \land q\) & \(p \to q\) & \((p \land q) \to (p \to q)\) \\
        T     & T     & T             & T           & T                             \\
        T     & F     & F             & F           & T                             \\
        F     & T     & F             & T           & T                             \\
        F     & F     & F             & T           & T
    \end{tabular}
    \caption{Truth Table for 11(d)}
\end{table}

\begin{table}[ht]
    \centering
    \begin{tabular}{c|c|c|c|c}
        \(p\) & \(q\) & \(p \to q\) & \(\lnot (p \to q)\) & \(\lnot (p \to q) \to p\) \\
        T     & T     & T           & F                   & T                         \\
        T     & F     & F           & T                   & T                         \\
        F     & T     & T           & F                   & T                         \\
        F     & F     & T           & F                   & T
    \end{tabular}
    \caption{Truth Table for 11(e)}
\end{table}

\begin{table}[ht]
    \centering
    \begin{tabular}{c|c|c|c|c|c}
        \(p\) & \(q\) & \(p \to q\) & \(\lnot(p \to q)\) & \(\lnot q\) & \(\lnot(p \to q) \to \lnot q\) \\
        T     & T     & T           & F                  & F           & T                              \\
        T     & F     & F           & T                  & T           & T                              \\
        F     & T     & T           & F                  & F           & T                              \\
        F     & F     & T           & F                  & T           & T
    \end{tabular}
    \caption{Truth Table for 11(f)}
\end{table}

\section*{Exercise 12}
Show that each of these conditional statements is a tautology by using truth tables.
\begin{enumerate}
    \item[\textbf{a)}] \([\lnot p \land (p \lor q)] \to q\)
    \item[\textbf{b)}] \([(p \to q) \land (q \to r)] \to (p \to r)\)
    \item[\textbf{c)}] \([p \land (p \to q)] \to q\)
    \item[\textbf{d)}] \([(p \lor q) \land (p \to r) \land (q \to r)] \to r\)
\end{enumerate}

\noindent
\textbf{Solution:}
\begin{table}[ht]
    \centering
    \begin{tabular}{c|c|c|c|c|c}
        \(p\) & \(q\) & \(\lnot p\) & \(p \lor q\) & \(\lnot p \land (p \lor q)\) & \([\lnot p \land (p \lor q)] \to q\) \\
        T     & T     & F           & T            & F                            & T                                    \\
        T     & F     & F           & T            & F                            & T                                    \\
        F     & T     & T           & T            & T                            & T                                    \\
        F     & F     & T           & F            & F                            & T
    \end{tabular}
    \caption{Truth Table for 12(a)}
\end{table}

\begin{table}[ht]
    \centering
    \begin{tabular}{c|c|c|c|c|c|c|c}
        \(p\) & \(q\) & \(r\) & \(p \to q\) & \(q \to r\) & \((p \to q) \land (q \to r)\) & \(p \to r\) & \([(p \to q) \land (q \to r)] \to (p \to r)\) \\
        T     & T     & T     & T           & T           & T                             & T           & T                                             \\
        T     & T     & F     & T           & F           & F                             & F           & T                                             \\
        T     & F     & T     & F           & T           & F                             & T           & T                                             \\
        T     & F     & F     & F           & T           & F                             & F           & T                                             \\
        F     & T     & T     & T           & T           & T                             & T           & T                                             \\
        F     & T     & F     & T           & F           & F                             & T           & T                                             \\
        F     & F     & T     & T           & T           & T                             & T           & T                                             \\
        F     & F     & F     & T           & T           & T                             & T           & T
    \end{tabular}
    \caption{Truth Table for 12(b)}
\end{table}

\begin{table}[ht]
    \centering
    \begin{tabular}{c|c|c|c|c}
        \(p\) & \(q\) & \(p \to q\) & \(p \land (p \to q)\) & \([p \land (p \to q)] \to q\) \\
        T     & T     & T           & T                     & T                             \\
        T     & F     & F           & F                     & T                             \\
        F     & T     & T           & F                     & T                             \\
        F     & F     & T           & F                     & T
    \end{tabular}
    \caption{Truth Table for 12(c)}
\end{table}

\begin{table}[ht]
    \centering
    \begin{tabular}{c|c|c|c|c|c|c|c}
        \(p\) & \(q\) & \(r\) & \(p \lor q\) & \(p \to r\) & \(q \to r\) & \((p \lor q) \land (p \to r) \land (q \to r)\) & \([(p \lor q) \land (p \to r) \land (q \to r)] \to r\) \\
        T     & T     & T     & T            & T           & T           & T                                              & T                                                      \\
        T     & T     & F     & T            & F           & F           & F                                              & T                                                      \\
        T     & F     & T     & T            & T           & T           & T                                              & T                                                      \\
        T     & F     & F     & T            & F           & T           & F                                              & T                                                      \\
        F     & T     & T     & T            & T           & T           & T                                              & T                                                      \\
        F     & T     & F     & F            & T           & F           & F                                              & T                                                      \\
        F     & F     & T     & F            & T           & T           & F                                              & T                                                      \\
        F     & F     & F     & F            & T           & T           & F                                              & T
    \end{tabular}
    \caption{Truth Table for 12(d)}
\end{table}

\section*{Exercise 13}
Show that each conditional statement in Exercise 11 is a tautology using the fact that a conditional statement is false exactly when the hypothesis is true and the conclusion is false. (Do not use truth tables.)

\noindent
\textbf{Solution:}
\begin{enumerate}
    \item[\textbf{a)}] Suppose \(p \land q\) is true and \(p\) is false. This is a contradiction, because for \(p \land q\) to be true, \(p\) must be true. Therefore, \((p \land q) \to p\) is a tautology.
    \item[\textbf{b)}] Suppose that \(p\) is true and \(p \lor q\) is false. This is a contradiction, because for \(p \lor q\) to be false, \(p\) must not be true and \(q\) must not be true. Therefore, \(p \to (p \lor q)\) is a tautology.
    \item[\textbf{c)}] Suppose that \(\lnot p\) is true and \(p \to q\) is false. This is a contradiction, because for \(p \to q\) to be false, \(p\) must be true and \(q\) must be false. Therefore, \(\lnot p \to (p \to q)\) is a tautology.
    \item[\textbf{d)}] Suppose that \(p \land q\) is true and \(p \to q\) is false. This is a contradiction, because for \(p \to q\) to be false, \(q\) must be false and \(p\) must be true, but for \(p \land q\) to be true, \(p\) and \(q\) must be true. Therefore, \((p \land q) \to (p \to q)\) is a tautology.
    \item[\textbf{e)}] Suppose that \(\lnot(p \to q)\) is true and \(p\) is false. This is a contradiction, because for \(\lnot(p \to q)\) to be true, \(p \to q\) must be false. For \(p \to q\) to be false, \(p\) must be true and \(q\) must be false. Therefore, \(\lnot(p \to q) \to p\) is a tautology.
    \item[\textbf{f)}] Suppose that \(\lnot(p \to q)\) is true and \(\lnot q\) is false. This is a contradiction, because for \(\lnot(p \to q\) to be true, \(p \to q\) must be false. For \(p \to q\) to be false, \(p\) must be true and \(q\) must be false. For \(\lnot q\) to be false, \(q\) must be true. Therefore, \(\lnot(p \to q) \to \lnot q\) is a tautology.
\end{enumerate}

\section*{Exercise 14}
Show that each conditional statement in Exercise 12 is a tautology using the fact that a conditional statement is false exactly when the hypothesis is true and the conclusion is false. (Do not use truth tables.)

\noindent
\textbf{Solution:}
\begin{enumerate}
    \item[\textbf{a)}] Suppose that \(\lnot p \land (p \lor q)\) is true and \(q\) is false. This is a contradiction, because then \(p \lor q\) and \(\lnot p\) would both have to be true, implying that \(p\) is false and \(q\) is true. Therefore, \([\lnot p \land (p \lor q)] \to q\) is a tautology.
    \item[\textbf{b)}] Suppose that \((p \to q) \land (q \to r)\) is true and \(p \to r\) is false. This is a contradiction, because for \(p \to r\) to be false, \(p\) must be true and \(r\) must be false. For \((p \to q) \land (q \to r)\) to be true, \(p \to q\) and \(q \to r\) must be true, but this requires differing truth values for \(q\). Therefore, \([(p \to q) \land (q \to r)] \to (p \to r)\) is a tautology.
    \item[\textbf{c)}] Suppose that \(p \land (p \to q)\) is true and \(q\) is false. This is a contradiction, because for \(p \land (p \to q)\) to be true, \(p\) must be true and \(q\) must not be false. Therefore, \([p \land (p \to q)] \to q\) is a tautology.
    \item[\textbf{d)}] Suppose that \((p \lor q) \land (p \to r) \land (q \to r)\) is true and \(r\) is false. This is a contradiction, because for \((p \lor q) \land (p \to r) \land (q \to r)\) to be true, \(p \lor q\), \(p \to r\), and \(q \to r\) must all be true. For \(p \lor q\) to be true, \(p\) or \(q\) must be true. However, for both \(p \to r\) and \(q \to r\) to be true, \(p\) and \(q\) must both be false since \(r\) is false. Therefore,  \([(p \lor q) \land (p \to r) \land (q \to r)] \to r\) is a tautology.
\end{enumerate}

\section*{Exercise 15}
Show that each conditional statement in Exercise 11 is a tautology by applying a chain of logical identities as in Example 8. (Do not use truth tables.)

\noindent
\textbf{Solution:}
\begin{enumerate}
    \item[\textbf{a)}] 
    \begin{gather*}
        (p \land q) \to p \equiv \lnot(p \land q) \lor p \ \textbf{by the conditional-disjunction equivalence} \\
        \equiv (\lnot p \lor \lnot q) \lor p \ \textbf{by a De Morgan's law} \\
        \equiv \lnot p \lor \lnot q \lor p \ \textbf{by an associative law} \\
        \equiv p \lor \lnot p \lor \lnot q \ \textbf{by a commutative law} \\
        \equiv \textbf{T} \lor \lnot q \ \textbf{by a negation law} \\
        \equiv \textbf{T} \ \textbf{by a domination law}
    \end{gather*}
    
    \item[\textbf{b)}]
    \begin{gather*}
        p \to (p \lor q) \\
        \equiv \lnot p \lor (p \lor q) \ \textbf{by the conditional-disjunction equivalence} \\
        \equiv (\lnot p \lor p) \lor q \ \textbf{by an associative law} \\
        \equiv (p \lor \lnot p) \lor q \ \textbf{by a commutative law} \\
        \equiv \textbf{T} \lor q \ \textbf{by a negation law} \\
        \equiv \textbf{T} \ \textbf{by a domination law}
    \end{gather*}
    
    \item[\textbf{c)}]
    \begin{gather*}
        \lnot p \to (p \to q) \\
        \equiv p \lor (p \to q) \ \textbf{by the conditional-disjunction equivalence} \\
        \equiv p \lor (\lnot p \lor q) \ \textbf{by the conditional-disjunction equivalence} \\
        \equiv (p \lor \lnot p) \lor q \ \textbf{by an associative law} \\
        \equiv \textbf{T} \lor q \ \textbf{by a negation law} \\
        \equiv \textbf{T} \ \textbf{by a domination law}
    \end{gather*}
    
    \item[\textbf{d)}] 
    \begin{gather*}
        (p \land q) \to (p \to q) \\
        \equiv \lnot (p \land q) \lor (p \to q) \ \textbf{by the conditional-disjunction equivalence} \\
        \equiv \lnot(p \land q) \lor (\lnot p \lor q) \ \textbf{by the conditional-disjunction equivalence} \\
        \equiv (\lnot p \lor \lnot q) \lor (\lnot p \lor q) \ \textbf{by a De Morgan's law} \\
        \equiv (\lnot p \lor \lnot p) \lor (\lnot q \lor q) \ \textbf{by an associative law} \\
        \equiv (\lnot p \lor \lnot p) \lor (q \lor \lnot q) \ \textbf{by a commutative law} \\
        \equiv \lnot p \lor (q \lor \lnot q) \ \textbf{by a idempotent law} \\
        \equiv \lnot p \lor \textbf{T} \ \textbf{by a negation law} \\
        \equiv \textbf{T} \ \textbf{by a domination law}
    \end{gather*}

    
    \item[\textbf{e)}]
    \begin{gather*}
        \lnot(p \to q) \to p \\
        \equiv (p \to q) \lor p \ \textbf{by the conditional-disjunction equivalence} \\
        \equiv (\lnot p \lor q) \lor p \ \textbf{by the conditional-disjunction equivalence} \\
        \equiv (\lnot p \lor p) \lor q \ \textbf{by an associative law} \\
        \equiv \textbf{T} \lor q \ \textbf{by a negation law} \\
        \equiv \textbf{T} \ \textbf{by a domination law}
    \end{gather*}
    
    \item[\textbf{f)}]
    \begin{gather*}
        \lnot(p \to q) \to \lnot q \\
        \equiv (p \to q) \lor \lnot q \ \textbf{by the conditional-disjunction equivalence} \\
        \equiv (\lnot p \lor q) \lor \lnot q  \ \textbf{by the conditional-disjunction equivalence} \\
        \equiv \lnot p \lor (q \lor \lnot q) \ \textbf{by an associative law} \\
        \equiv \lnot p \lor \textbf{T} \ \textbf{by a negation law} \\
        \equiv \textbf{T} \ \textbf{by a domination law}
    \end{gather*}
\end{enumerate}

\section*{Exercise 16}
Show that each conditional statement in Exercise 12 is a tautology by applying a chain of logical identities as in Example 8. (Do not use truth tables.)

\noindent
\textbf{Solution:}
\begin{enumerate}
    \item[\textbf{a)}]
    \begin{gather*}
        [\lnot p \land (p \lor q)] \to q \\
        \equiv \lnot [\lnot p \land (p \lor q)] \lor q \ \textbf{by the conditional-disjunction equivalence} \\
        \equiv [p \lor \lnot(p \lor q)] \lor q \ \textbf{by a De Morgan's law} \\
        \equiv (p \lor q) \lor \lnot(p \lor q) \ \textbf{by commutative and associative laws} \\
        \equiv \textbf{T} \ \textbf{by a negation law}
    \end{gather*}
    
    \item[\textbf{b)}]
    \begin{gather*}
        [(p \to q) \land (q \to r)] \to (p \to r) \\
        \equiv \lnot[(p \to q) \land (q \to r)] \lor (p \to r) \ \textbf{by the conditional-disjunction equivalence} \\
        \equiv [\lnot(p \to q) \lor \lnot(q \to r)] \lor (p \to r) \ \textbf{by a De Morgan's law} \\
        \equiv [(p \land \lnot q) \lor (q \land \lnot r)] \lor (p \to r) \ \textbf{by the conditional-disjunction equivalence and a De Morgan's law} \\
        \equiv (p \land \lnot q) \lor (q \land \lnot r) \lor (\lnot p \lor q) \ \textbf{by the conditional-disjunction equivalence and an associative law} \\
        \equiv q \lor (q \land \lnot r) \lor (p \land \lnot q) \lor \lnot p \ \textbf{by a commutative law and an associative law} \\
        \equiv (\lnot p \lor q) \lor (p \land \lnot q) \ \textbf{by an absorption law} \\
        \equiv \lnot(p \land \lnot q) \lor (p \land \lnot q) \ \textbf{by a De Morgan's law} \\
        \equiv \textbf{T} \ \textbf{by a negation law}
    \end{gather*}
    
    \item[\textbf{c)}]
    \begin{gather*}
        [p \land (p \to q)] \to q \\
        \equiv \lnot[p \land (p \to q)] \lor q \ \textbf{by the conditional-disjunction equivalence} \\
        \equiv \lnot[p \land (\lnot p \lor q)] \lor q \ \textbf{by the conditional-disjunction equivalence} \\
        \equiv [\lnot p \lor \lnot(\lnot p \lor q)] \lor q \ \textbf{by a De Morgan's law} \\
        \equiv \lnot p \lor \lnot(\lnot p \lor q)] \lor q \ \textbf{by an associative law} \\
        \equiv \lnot p \lor q \lor \lnot(\lnot p \lor q) \ \textbf{by a commutative law} \\
        \equiv (\lnot p \lor q) \lor \lnot(\lnot p \lor q) \ \textbf{by an associative law} \\
        \equiv \textbf{T} \ \textbf{by a negation law}
    \end{gather*}
    
    \item[\textbf{d)}]
    \begin{gather*}
        [(p \lor q) \land (p \to r) \land (q \to r)] \to r \\
        \equiv [(p \lor q) \land (\lnot p \lor r) \land (\lnot q \lor r)] \to r \ \textbf{by the conditional-disjunction equivalence} \\
        \equiv [(p \lor q) \land (r \lor (\lnot p \land \lnot q))] \to r \ \textbf{by a distributive law} \\
        \equiv \lnot[(p \lor q) \land (r \lor (\lnot p \land \lnot q))] \lor r \ \textbf{by the conditional-disjunction equivalence} \\
        \equiv \lnot[(p \lor q) \land (r \lor \lnot(p \lor q))] \lor r \ \textbf{by a De Morgan's law} \\
        \equiv [\lnot(p \lor q) \lor \lnot(r \lor \lnot(p \lor q))] \lor r \ \textbf{by a De Morgan's law} \\
        \equiv \lnot(p \lor q) \lor \lnot(r \lor \lnot(p \lor q)) \lor r \ \textbf{by an associative law} \\
        \equiv r \lor \lnot(p \lor q) \lor \lnot(r \lor \lnot(p \lor q)) \ \textbf{by a commutative law} \\
        \equiv (r \lor \lnot(p \lor q)) \lor \lnot(r \lor \lnot(p \lor q)) \ \textbf{by an associative law} \\
        \equiv \textbf{T} \ \textbf{by a negation law}
    \end{gather*}
\end{enumerate}

\section*{Exercise 17}
Use truth tables to verify the absorption laws.
\begin{enumerate}
    \item[\textbf{a)}] \(p \lor (p \land q) \equiv p\)
    \item[\textbf{b)}] \(p \land (p \lor q) \equiv p\)
\end{enumerate}

\noindent
\textbf{Solution:}
\begin{table}[ht]
    \centering
    \begin{tabular}{c|c|c|c}
        \(p\) & \(q\) & \(p \land q\) & \(p \lor (p \land q)\) \\
        T     & T     & T             & T                      \\
        T     & F     & F             & T                      \\
        F     & T     & F             & F                      \\
        F     & F     & F             & F
    \end{tabular}
    \caption{Truth Table for 17(a)}
\end{table}

\begin{table}[ht]
    \centering
    \begin{tabular}{c|c|c|c}
        \(p\) & \(q\) & \(p \lor q\) & \(p \land (p \lor q)\) \\
        T     & T     & T            & T                      \\
        T     & F     & T            & T                      \\
        F     & T     & T            & F                      \\
        F     & F     & F            & F
    \end{tabular}
    \caption{Truth Table for 17(b)}
\end{table}

\section*{Exercise 18}
Determine whether \((\lnot p \land(p \to q)) \to \lnot q\) is a tautology.

\noindent
\textbf{Solution:}
\begin{table}[ht]
    \centering
    \begin{tabular}{c|c|c|c|c|c|c}
        \(p\) & \(q\) & \(p \to q\) & \(\lnot p\) & \(\lnot p \land (p \to q)\) & \(\lnot q\) & \((\lnot p \land (p \to q)) \to \lnot q\) \\
        T     & T     & T           & F           & F                           & F           & T                                         \\
        T     & F     & F           & F           & F                           & T           & T                                         \\
        F     & T     & T           & T           & T                           & F           & F                                         \\
        F     & F     & T           & T           & T                           & T           & T
    \end{tabular}
    \caption{Truth Table for 18}
    \label{Truth Table for 18}
\end{table}

As seen in Table \ref{Truth Table for 18}, \((\lnot p \land(p \to q)) \to \lnot q\) is not a tautology.

\section*{Exercise 19}
Determine whether \((\lnot q \land (p \to q)) \to \lnot p\) is a tautology.

\noindent
\textbf{Solution:}
\begin{table}[ht]
    \centering
    \begin{tabular}{c|c|c|c|c|c|c}
        \(p\) & \(q\) & \(\lnot q\) & \(p \to q\) & \(\lnot q \land (p \to q)\) & \(\lnot p\) & \((\lnot q \land (p \to q)) \to \lnot p\) \\
        T     & T     & F           & T           & F                           & F           & T                                         \\
        T     & F     & T           & F           & F                           & F           & T                                         \\
        F     & T     & F           & T           & F                           & T           & T                                         \\
        F     & F     & T           & T           & T                           & T           & T
    \end{tabular}
    \caption{Truth Table for 19}
    \label{Truth Table for 19}
\end{table}

As seen in Table \ref{Truth Table for 19}, \((\lnot q \land (p \to q)) \to \lnot p\) is a tautology.

Each of Exercises 20-32 asks you to show that two compound propositions are logically equivalent. To do this, either show that both sides are true, or that both sides are false, for exactly the same combinations of truth values of the propositional variables in these expressions (whichever is easier).

\section*{Exercise 20}
Show that \(p \leftrightarrow q\) and \((p \land q) \lor (\lnot p \land \lnot q)\) are logically equivalent.

\noindent
\textbf{Solution:}
\begin{table}[ht]
    \centering
    \begin{tabular}{c|c|c|c|c|c|c|c}
        \(p\) & \(q\) & \(p \leftrightarrow q\) & \(p \land q\) & \(\lnot p\) & \(\lnot q\) & \(\lnot p \land \lnot q\) & \((p \land q) \lor (\lnot p \land \lnot q)\) \\
        T     & T     & T                       & T             & F           & F           & F                         & T                                            \\
        T     & F     & F                       & F             & F           & T           & F                         & F                                            \\
        F     & T     & F                       & F             & T           & F           & F                         & F                                            \\
        F     & F     & T                       & F             & T           & T           & T                         & T
    \end{tabular}
    \caption{Truth Table for 20}
\end{table}

\section*{Exercise 21}
Show that \(\lnot(p \leftrightarrow q)\) and \(p \leftrightarrow \lnot q\) are logically equivalent.

\noindent
\textbf{Solution:}
\begin{table}[ht]
    \centering
    \begin{tabular}{c|c|c|c|c}
        \(p\) & \(q\) & \(\lnot(p \leftrightarrow q)\) & \(\lnot q\) & \(p \leftrightarrow \lnot q\) \\
        T     & T     & F                              & F           & F                             \\
        T     & F     & T                              & T           & T                             \\
        F     & T     & T                              & F           & T                             \\
        F     & F     & F                              & T           & F
    \end{tabular}
    \caption{Truth Table for 21}
\end{table}

\section*{Exercise 22}
Show that \(p \to q\) and \(\lnot q \to \lnot p\) are logically equivalent.

\noindent
\textbf{Solution:}
\begin{table}[ht]
    \centering
    \begin{tabular}{c|c|c|c|c|c}
        \(p\) & \(q\) & \(p \to q\) & \(\lnot q\) & \(\lnot p\) & \(\lnot q \to \lnot p\) \\
        T     & T     & T           & F           & F           & T                       \\
        T     & F     & F           & T           & F           & F                       \\
        F     & T     & T           & F           & T           & T                       \\
        F     & F     & T           & T           & T           & T
    \end{tabular}
    \caption{Truth Table for 22}
\end{table}

\section*{Exercise 23}
Show that \(\lnot p \leftrightarrow q\) and \(p \leftrightarrow \lnot q\) are logically equivalent.

\noindent
\textbf{Solution:}
\begin{table}[ht]
    \centering
    \begin{tabular}{c|c|c|c|c|c}
        \(p\) & \(q\) & \(\lnot p\) & \(\lnot p \leftrightarrow q\) & \(\lnot q\) & \(p \leftrightarrow \lnot q\) \\
        T     & T     & F           & F                             & F           & F                             \\
        T     & F     & F           & T                             & T           & T                             \\
        F     & T     & T           & T                             & F           & T                             \\
        F     & F     & T           & F                             & T           & F
    \end{tabular}
    \caption{Truth Table for 23}
\end{table}

\section*{Exercise 24}
Show that \(\lnot(p \oplus q)\) and \(p \leftrightarrow q\) are logically equivalent.

\noindent
\textbf{Solution:}
\begin{table}[ht]
    \centering
    \begin{tabular}{c|c|c|c|c}
        \(p\) & \(q\) & \(p \oplus q\) & \(\lnot(p \oplus q)\) & \(p \leftrightarrow q\) \\
        T     & T     & F              & T                     & T                       \\
        T     & F     & T              & F                     & F                       \\
        F     & T     & T              & F                     & F                       \\
        F     & F     & F              & T                     & T
    \end{tabular}
    \caption{Truth Table for 24}
\end{table}

\section*{Exercise 25}
Show that \(\lnot(p \leftrightarrow q)\) and \(\lnot p \leftrightarrow q\) are logically equivalent.

\noindent
\textbf{Solution:}
\begin{table}[ht]
    \centering
    \begin{tabular}{c|c|c|c|c}
        \(p\) & \(q\) & \(\lnot(p \leftrightarrow q)\) & \(\lnot p\) & \(\lnot p \leftrightarrow q\) \\
        T     & T     & F                              & F           & F                             \\
        T     & F     & T                              & F           & T                             \\
        F     & T     & T                              & T           & T                             \\
        F     & F     & F                              & T           & F
    \end{tabular}
    \caption{Truth Table for 25}
\end{table}

\section*{Exercise 26}
Show that \((p \to q) \land (p \to r)\) and \(p \to (q \land r)\) are logically equivalent.

\noindent
\textbf{Solution:}
\begin{table}[ht]
    \centering
    \begin{tabular}{c|c|c|c|c|c|c|c}
        \(p\) & \(q\) & \(r\) & \(p \to q\) & \(p \to r\) & \((p \to q) \land (p \to r)\) & \(q \land r\) & \(p \to (q \land r)\) \\
        T     & T     & T     & T           & T           & T                             & T             & T                     \\
        T     & T     & F     & T           & F           & F                             & F             & F                     \\
        T     & F     & T     & F           & T           & F                             & F             & F                     \\
        T     & F     & F     & F           & F           & F                             & F             & F                     \\
        F     & T     & T     & T           & T           & T                             & T             & T                     \\
        F     & T     & F     & T           & T           & T                             & F             & T                     \\
        F     & F     & T     & T           & T           & T                             & F             & T                     \\
        F     & F     & F     & T           & T           & T                             & F             & T                     \\
    \end{tabular}
    \caption{Truth Table for 26}
\end{table}

\section*{Exercise 27}
Show that \((p \to r) \land (q \to r)\) and \((p \lor q) \to r\) are logically equivalent.

\noindent
\textbf{Solution:}
\begin{table}[ht]
    \centering
    \begin{tabular}{c|c|c|c|c|c|c|c}
        \(p\) & \(q\) & \(r\) & \(p \to r\) & \(q \to r\) & \((p \to r) \land (q \to r)\) & \(p \lor q\) & \((p \lor q) \to r\) \\
        T     & T     & T     & T           & T           & T                             & T            & T                    \\
        T     & T     & F     & F           & F           & F                             & T            & F                    \\
        T     & F     & T     & T           & T           & T                             & T            & T                    \\
        T     & F     & F     & F           & T           & F                             & T            & F                    \\
        F     & T     & T     & T           & T           & T                             & T            & T                    \\
        F     & T     & F     & T           & F           & F                             & T            & F                    \\
        F     & F     & T     & T           & T           & T                             & F            & T                    \\
        F     & F     & F     & T           & T           & T                             & F            & T
    \end{tabular}
    \caption{Truth Table for 27}
\end{table}

\section*{Exercise 28}
Show that \((p \to q) \lor (p \to r)\) and \(p \to (q \lor r)\) are logically equivalent.

\noindent
\textbf{Solution:}
\begin{table}[ht]
    \centering
    \begin{tabular}{c|c|c|c|c|c|c|c}
        \(p\) & \(q\) & \(r\) & \(p \to q\) & \(p \to r\) & \((p \to q) \lor (p \to r)\) & \(q \lor r\) & \(p \to (q \lor r)\) \\
        T     & T     & T     & T           & T           & T                            & T            & T                    \\
        T     & T     & F     & T           & F           & T                            & T            & T                    \\
        T     & F     & T     & F           & T           & T                            & T            & T                    \\
        T     & F     & F     & F           & F           & F                            & F            & F                    \\
        F     & T     & T     & T           & T           & T                            & T            & T                    \\
        F     & T     & F     & T           & T           & T                            & T            & T                    \\
        F     & F     & T     & T           & T           & T                            & T            & T                    \\
        F     & F     & F     & T           & T           & T                            & F            & T
    \end{tabular}
    \caption{Truth Table for 28}
\end{table}

\section*{Exercise 29}
Show that \((p \to r) \lor (q \to r)\) and \((p \land q) \to r\) are logically equivalent.

\noindent
\textbf{Solution:}
\begin{table}[ht]
    \centering
    \begin{tabular}{c|c|c|c|c|c|c|c}
        \(p\) & \(q\) & \(r\) & \(p \to r\) & \(q \to r\) & \((p \to r) \lor (q \to r)\) & \(p \land q\) & \((p \land q) \to r\) \\
        T     & T     & T     & T           & T           & T                            & T             & T                     \\
        T     & T     & F     & F           & F           & F                            & T             & F                     \\
        T     & F     & T     & T           & T           & T                            & F             & T                     \\
        T     & F     & F     & F           & T           & T                            & F             & T                     \\
        F     & T     & T     & T           & T           & T                            & F             & T                     \\
        F     & T     & F     & T           & F           & T                            & F             & T                     \\
        F     & F     & T     & T           & T           & T                            & F             & T                     \\
        F     & F     & F     & T           & T           & T                            & F             & T
    \end{tabular}
    \caption{Truth Table for 29}
\end{table}

\section*{Exercise 30}
Show that \(\lnot p \to (q \to r)\) and \(q \to (p \lor r)\) are logically equivalent.

\noindent
\textbf{Solution:}
\begin{table}[ht]
    \centering
    \begin{tabular}{c|c|c|c|c|c|c|c}
        \(p\) & \(q\) & \(r\) & \(\lnot p\) & \(q \to r\) & \(\lnot p \to (q \to r)\) & \(p \lor r\) & \(q \to (p \lor r)\) \\
        T     & T     & T     & F           & T           & T                         & T            & T                    \\
        T     & T     & F     & F           & F           & T                         & T            & T                    \\
        T     & F     & T     & F           & T           & T                         & T            & T                    \\
        T     & F     & F     & F           & T           & T                         & T            & T                    \\
        F     & T     & T     & T           & T           & T                         & T            & T                    \\
        F     & T     & F     & T           & F           & F                         & F            & F                    \\
        F     & F     & T     & T           & T           & T                         & T            & T                    \\
        F     & F     & F     & T           & T           & T                         & F            & T
    \end{tabular}
    \caption{Truth Table for 30}
\end{table}

\section*{Exercise 31}
Show that \(p \leftrightarrow q\) and \((p \to q) \land (q \to p)\) are logically equivalent.

\noindent
\textbf{Solution:}
\begin{table}[ht]
    \centering
    \begin{tabular}{c|c|c|c|c|c}
        \(p\) & \(q\) & \(p \leftrightarrow q\) & \(p \to q\) & \(q \to p\) & \((p \to q) \land (q \to p)\) \\
        T     & T     & T                       & T           & T           & T                             \\
        T     & F     & F                       & F           & T           & F                             \\
        F     & T     & F                       & T           & F           & F                             \\
        F     & F     & T                       & T           & T           & T
    \end{tabular}
    \caption{Truth Table for 31}
\end{table}

\section*{Exercise 32}
Show that \(p \leftrightarrow q\) and \(\lnot p \leftrightarrow \lnot q\) are logically equivalent.

\noindent
\textbf{Solution:}
\begin{table}[ht]
    \centering
    \begin{tabular}{c|c|c|c|c|c}
        \(p\) & \(q\) & \(p \leftrightarrow q\) & \(\lnot p\) & \(\lnot q\) & \(\lnot p \leftrightarrow \lnot q\) \\
        T     & T     & T                       & F           & F           & T                                   \\
        T     & F     & F                       & F           & T           & F                                   \\
        F     & T     & F                       & T           & F           & F                                   \\
        F     & F     & T                       & T           & T           & T
    \end{tabular}
    \caption{Truth Table for 32}
\end{table}

\section*{Exercise 33}
Show that \((p \to q) \land (q \to r) \to (p  \to r)\) is a tautology.

\noindent
\textbf{Solution:}
\begin{table}[ht]
    \centering
    \begin{tabular}{c|c|c|c|c|c|c|c}
        \(p\) & \(q\) & \(r\) & \(p \to q\) & \(q \to r\) & \(p \to r\) & \((p \to q) \land (q \to r)\) & \((p \to q) \land (q \to r) \to (p \to r)\) \\
        T     & T     & T     & T           & T           & T           & T                             & T                                           \\
        T     & T     & F     & T           & F           & F           & F                             & T                                           \\
        T     & F     & T     & F           & T           & T           & F                             & T                                           \\
        T     & F     & F     & F           & T           & F           & F                             & T                                           \\
        F     & T     & T     & T           & T           & T           & T                             & T                                           \\
        F     & T     & F     & T           & F           & T           & F                             & T                                           \\
        F     & F     & T     & T           & T           & T           & T                             & T                                           \\
        F     & F     & F     & T           & T           & T           & T                             & T
    \end{tabular}
    \caption{Truth Table for 33}
\end{table}

\section*{Exercise 34}
Show that \((p \lor q) \land (\lnot p \lor r) \to (q \lor r)\) is a tautology.

\noindent
\textbf{Solution:}
\begin{table}[ht]
    \centering
    \begin{tabular}{c|c|c|c|c|c|c|c|c}
        \(p\) & \(q\) & \(r\) & \(p \lor q\) & \(\lnot p\) & \(\lnot p \lor r\) & \(q \lor r\) & \((p \lor q) \land (\lnot p \lor r)\) & \((p \lor q) \land (\lnot p \lor r) \to (q \lor r)\) \\
        T     & T     & T     & T            & F           & T                  & T            & T                                     & T                                                    \\
        T     & T     & F     & T            & F           & F                  & T            & F                                     & T                                                    \\
        T     & F     & T     & T            & F           & T                  & T            & T                                     & T                                                    \\
        T     & F     & F     & T            & F           & F                  & F            & F                                     & T                                                    \\
        F     & T     & T     & T            & T           & T                  & T            & T                                     & T                                                    \\
        F     & T     & F     & T            & T           & T                  & T            & T                                     & T                                                    \\
        F     & F     & T     & F            & T           & T                  & T            & F                                     & T                                                    \\
        F     & F     & F     & F            & T           & T                  & F            & F                                     & T
    \end{tabular}
    \caption{Truth Table for 34}
\end{table}

\section*{Exercise 35}
Show that \((p \to q) \to r\) and \(p \to (q \to r)\) are not logically equivalent.

\noindent
\textbf{Solution:}
\begin{table}[ht]
    \centering
    \begin{tabular}{c|c|c|c|c|c|c}
        \(p\) & \(q\) & \(r\) & \(p \to q\) & \((p \to q) \to r\) & \(q \to r\) & \(p \to (q \to r)\) \\
        T     & T     & T     & T           & T                   & T           & T                   \\
        T     & T     & F     & T           & F                   & F           & F                   \\
        T     & F     & T     & F           & T                   & T           & T                   \\
        T     & F     & F     & F           & T                   & T           & T                   \\
        F     & T     & T     & T           & T                   & T           & T                   \\
        F     & T     & F     & T           & F                   & F           & T                   \\
        F     & F     & T     & T           & T                   & T           & T                   \\
        F     & F     & F     & T           & F                   & T           & T
    \end{tabular}
    \caption{Truth Table for 35}
\end{table}

\section*{Exercise 36}
Show that \((p \land q) \to r\) and \((p \to r) \land (q \to r)\) are not logically equivalent.

\noindent
\textbf{Solution:}
\begin{table}[ht]
    \centering
    \begin{tabular}{c|c|c|c|c|c|c|c}
        \(p\) & \(q\) & \(r\) & \(p \land q\) & \((p \land q) \to r\) & \(p \to r\) & \(q \to r\) & \((p \to r) \land (q \to r)\) \\
        T     & T     & T     & T             & T                     & T           & T           & F                             \\
        T     & T     & F     & T             & F                     & F           & F           & F                             \\
        T     & F     & T     & F             & T                     & T           & T           & T                             \\
        T     & F     & F     & F             & T                     & F           & T           & F                             \\
        F     & T     & T     & F             & T                     & T           & T           & T                             \\
        F     & T     & F     & F             & T                     & T           & F           & F                             \\
        F     & F     & T     & F             & T                     & T           & T           & T                             \\
        F     & F     & F     & F             & T                     & T           & T           & T
    \end{tabular}
    \caption{Truth Table for 36}
\end{table}

\section*{Exercise 37}
Show that \((p \to q) \to (r \to s)\) and \((p \to r) \to (q \to s)\) are not logically equivalent.

\noindent
\textbf{Solution:}
First,
\begin{gather*}
    (p \to q) \to (r \to s)                            \\
    \equiv \lnot(\lnot p \lor q) \lor (\lnot r \lor s) \\
    \equiv (p \land \lnot q) \lor (\lnot r \lor s).    \\
\end{gather*}

Second,
\begin{gather*}
    (p \to r) \to (q \to s)                             \\
    \equiv \lnot (\lnot p \lor r) \lor (\lnot q \lor s) \\
    \equiv (p \land \lnot r) \lor (\lnot q \lor s).     \\
\end{gather*}

Now, we need to show that \((p \land \lnot q) \lor (\lnot r \lor s) \equiv (p \land \lnot r) \lor (\lnot q \lor s)\).
\begin{gather*}
    (p \land \lnot q) \lor (\lnot r \lor s) \equiv (p \land \lnot r) \lor (\lnot q \lor s)                             \\
    \text{implies } \lnot r \lor (p \land \lnot q) \equiv \lnot q \lor (p \land \lnot r)                               \\
    \text{implies } (\lnot r \lor p) \land (\lnot r \lor \lnot q( \equiv (\lnot q \lor p) \land (\lnot q \lor \lnot r) \\
    \text{implies } \lnot r \lor p \equiv \lnot q \lor p                                                               \\
    \text{implies } \lnot r \equiv \lnot q                                                                             \\
\end{gather*}
which does not hold when \(q \equiv \text{T}\) and \(r \equiv \text{F}\).

The \textbf{dual} of a compound proposition that contains only the logical operators \(\lor\), \(\land\), and \(\lnot\) is the compound proposition obtained by replacing each \(\lor\) by \(\land\), each \(\land\) by \(\lor\), each \textbf{T} by \textbf{F}, and each \textbf{F} by \textbf{T}. The dual of \(s\) is denoted by \(s^*\).

\section*{Exercise 38}
Find the dual of each of these compound propositions.
\begin{enumerate}
    \item[\textbf{a)}] \(p \lor \lnot q\)
    \item[\textbf{b)}] \(p \land (q \lor (r \land \textbf{T}))\)
    \item[\textbf{c)}] \((p \land \lnot q) \lor (q \land \textbf{F})\)
\end{enumerate}

\noindent
\textbf{Solution:}
\begin{enumerate}
    \item[\textbf{a)}] \(p \land \lnot q\)
    \item[\textbf{b)}] \(p \lor (q \land (r \lor \textbf{F}))\)
    \item[\textbf{c)}] \((p \lor \lnot q) \land (q \lor \textbf{T})\)
\end{enumerate}

\section*{Exercise 39}
\begin{enumerate}
    \item[\textbf{a)}] \(p \land \lnot q \land \lnot r\)
    \item[\textbf{b)}] \((p \land q \land r) \lor s\)
    \item[\textbf{c)}] \((p \lor \textbf{F}) \land (q \lor \textbf{T})\)
\end{enumerate}

\noindent
\textbf{Solution:}
\begin{enumerate}
    \item[\textbf{a)}] \(p \lor \lnot q \lor \lnot r\)
    \item[\textbf{b)}] \((p \lor q \lor r) \land s\)
    \item[\textbf{c)}] \((p \land \textbf{T}) \lor (q \land \textbf{F})\)
\end{enumerate}

\section*{Exercise 40}
When does \(s^* = s\), where \(s\) is a compound proposition?

\noindent
\textbf{Solution:}
\(s^* = s\) whenever \(s\) contains no conjunctions, disjunctions, tautologies, or contradictions. Therefore, \(s^* = s\) whenever \(s\) contains only one propositional variable.

\section*{Exercise 41}
Show that \((s^*)^* = s\) when \(s\) is a compound proposition.

\noindent
\textbf{Solution:}
\((s^*)^*\) is the equivalent to switching the conjunctions to disjunctions back to conjunctions, disjunctions to conjunctions back to disjunctions, and so forth. Therefore, it performs each operation twice, thus returning the compound proposition back to the original state.

\section*{Exercise 42}
Show that the logical equivalences in Table 6, except for the double negation law, come in pairs, where each pair contains compound propositions that are duals of each other.

\noindent
\textbf{Solution:}
Each law in the table (except for the double negation law) is made to show two propositions which are both duals of each other.

\section*{Exercise 43}
Why are the duals of two equivalent compound propositions also equivalent, where these compound propositions contain only the operators \(\land\), \(\lor\), and \(\lnot\)?

\noindent
\textbf{Solution:}
As discussed in Exercise 42, all of the logical equivalences in Table 6, except for the double negation law, come in pairs, where each pair contains compound propositions that are duals of each other. When two compound propositions are logically equivalent and you take the dual of both compound propositions, they remain equivalent due to the dual symmetry presented in Table 6.

\section*{Exercise 44}
Find a compound proposition involving the propositional variables \(p\), \(q\), and \(r\) that is true when \(p\) and \(q\) are true and \(r\) is false, but is false otherwise. [\textit{Hint:} Use a conjunction of each propositional variable or its negation]

\noindent
\textbf{Solution:}
\(p \land q \land \lnot r\)

\section*{Exercise 45}
Find a compound proposition involving the propositional variables \(p\), \(q\), and \(r\) that is true when exactly two of \(p\), \(q\), and \(r\) are true and is false otherwise. [\textit{Hint:} Form a disjunction of conjunctions. Include a conjunction for each combination of values for which the compound proposition is true. Each conjunction should include each of the three propositional variables or its negations.]

\noindent
\textbf{Solution:}
\((p \land q \land \lnot r) \lor (p \land \lnot q \land r) \lor (\lnot p \land q \land r)\)

\section*{Exercise 46}
Suppose that a truth table in \(n\) propositional variables is specified. Show that a compound proposition with this truth table can be formed by taking the disjunction of conjunctions of the variables or their negations, with one conjunction included for each combination of values for which the compound proposition is true. The resulting compound proposition is said to be in \textbf{disjunctive normal form}.

\noindent
\textbf{Solution:}
The problem description practically solves the problem in itself. Start with a compound proposition with the disjunction of all \(2^n\) possible conjunctions of \(n\) propositional variables or their negations. Each row of the truth table corresponds to a conjunction within the top-level compound proposition, where you simply take the conjunction of the propositional variable or their negations (e.g. if \(p \equiv \textbf{T}\) and \(q \equiv \textbf{F}\), form a conjunction \(p \land \lnot q\)). Finally, eliminate all conjunctions from the top-level compound proposition that do not satisfy any row of the truth table.

A collection of logical operators is called \textbf{functionally complete} if every compound proposition is logically equivalent to a compound proposition involving only these logical operators.

\section*{Exercise 47}
Show that \(\lnot\), \(\land\), and \(\lor\) form a functionally complete collection of logical operators. [\textit{Hint:} Use the fact that every compound proposition is logically equivalent to one in disjunctive normal form, as shown in Exercise 46.]

\noindent
\textbf{Solution:}
From Exercise 46, every compound proposition is logically equivalent to one in disjunctive normal form. Thus, since disjunctive normal form only utilizes the operators \(\lnot\), \(\land\), and \(\lor\), \(\lnot\), \(\land\), and \(\lor\) are a functionally complete collection of logical operators.

\section*{Exercise 48}
Show that \(\lnot\) and \(\land\) form a functionally complete collection of logical operators. [\textit{Hint:} First use a De Morgan law to show that \(p \lor q\) is logically equivalent to \(\lnot(\lnot p \land \lnot q)\).]

\noindent
\textbf{Solution:}
By a De Morgan law, \(p \lor q \equiv \lnot( \lnot p \land \lnot q)\). Thus, any compound proposition in disjunctive normal form can be transformed to a compound proposition utilizing only \(\lnot\) and \(\land\) operators. Since all compound propositions can be represented in disjunctive normal form, all compound propositions can be represented with the operators \(\lnot\) and \(\land\).

\section*{Exercise 49}
Show that \(\lnot\) and \(\lor\) form a functionally complete collection of logical operators.

\noindent
\textbf{Solution:}
Consider a compound proposition in disjunctive normal form. The inner conjunction compound propositions can be transformed to be represented by \(\lnot\) and \(\lor\) by a De Morgan law, and thus \(\lnot\) and \(\lor\) are a functionally complete collection of logical operators.

We now present a group of exercises that involve the logical operators \textit{NAND} and \textit{NOR}. The proposition \(p \ \textit{NAND} \ q\) is true when either \(p\) or \(q\), or both, are false; and it is false when both \(p\) and \(q\) are true. The proposition \(p \ \textit{NOR} \ q\) is true when both \(p\) and \(q\) are false, and it is false otherwise. The propositions \(p \ \textit{NAND} \ q\) and \(p \ \textit{NOR} \ q\) are denoted by \(p \mathbin{\mid} q\) and \(p \downarrow q\), respectively. (The operators \(\mathbin{\mid}\) and \(\downarrow\) are called the \textbf{Sheffer stroke} and the \textbf{Peirce arrow} after H.M. Sheffer and C.S. Peirce, respectively.)

\section*{Exercise 50}
Construct a truth table for the logical operator \textit{NAND}.

\noindent
\textbf{Solution:}
\begin{table}[ht]
    \centering
    \begin{tabular}{c|c|c}
        \(p\) & \(q\) & \(p \ \textit{NAND} \ q\) \\
        T     & T     & F                         \\
        T     & F     & T                         \\
        F     & T     & T                         \\
        F     & F     & T
    \end{tabular}
    \caption{Truth Table for 50}
\end{table}

\section*{Exercise 51}
Show that \(p \mathbin{\mid} q\) is logically equivalent to \(\lnot (p \land q)\).

\noindent
\textbf{Solution:}
The truth table from Exercise 50 matches the truth table of \(\lnot (p \land q)\) and is thus logically equivalent.

\section*{Exercise 52}
Construct a truth table for the logical operator \textit{NOR}.

\noindent
\textbf{Solution:}
\begin{table}[ht]
    \centering
    \begin{tabular}{c|c|c}
        \(p\) & \(q\) & \(p \ \textit{NOR} \ q\) \\
        T     & T     & F                         \\
        T     & F     & F                         \\
        F     & T     & F                         \\
        F     & F     & T
    \end{tabular}
    \caption{Truth Table for 51}
\end{table}

\section*{Exercise 53}
Show that \(p  \downarrow q\) is logically equivalent to \(\lnot (p \lor q)\).

\noindent
\textbf{Solution:}
The truth table from Exercise 52 matches the truth table of \(\lnot (p \lor q)\) and is thus logically equivalent.

\section*{Exercise 54}
In this exercise we will show that \(\left\{ \downarrow \right\}\) is a functionally complete collection of logical operators.
\begin{enumerate}
    \item[\textbf{a)}] Show that \(p \downarrow q\) is logically equivalent to \(\lnot p\).
    \item[\textbf{b)}] Show that \((p \downarrow q) \downarrow (p \downarrow q)\) is logically equivalent to \(p \lor q\).
    \item[\textbf{c)}] Conclude from parts (a) and (b), and Exercise 49, that \(\left\{\downarrow\right\}\) is a functionally complete collection of logical operators.
\end{enumerate}

\noindent
\textbf{Solution:}
\begin{enumerate}
    \item[\textbf{a)}]
    \begin{table}[ht]
    \centering
    \begin{tabular}{c|c|c}
        \(p\) & \(\lnot p\) & \(p \downarrow p\) \\
        T     & F           & F                  \\
        F     & T           & T
    \end{tabular}
    \caption{Truth Table for 54(a)}
    \end{table}

    \item[\textbf{b)}]
    \begin{table}[ht]
    \centering
    \begin{tabular}{c|c|c|c|c}
        \(p\) & \(q\) & \(p \lor q\) & \(p \downarrow q\) & \((p \downarrow q) \downarrow (p \downarrow q)\) \\
        T     & T     & T            & F                  & T                                                \\
        T     & F     & T            & F                  & T                                                \\
        F     & T     & T            & F                  & T                                                \\
        F     & F     & F            & T                  & F
    \end{tabular}
    \caption{Truth Table for 54(b)}
    \end{table}
    
    \item[\textbf{c)}]
    Exercise 49 showed that \(\lnot\) and \(\lor\) form a functionally complete collection of logical operators. Part (a) of this exercise showed that for any proposition \(p\), its negation can be represented using \(\downarrow\). Part (b) of this exercise showed that for any proposition \(p\), the disjunction can be represented with \(\downarrow\). Therefore, \(\left\{\downarrow\right\}\) forms a functionally complete collection of logical operators.
\end{enumerate}

\section*{Exercise 55}
Find a compound proposition logically equivalent to \(p \to q\) using only the logical operator \(\downarrow\).

\noindent
\textbf{Solution}
First, note that \(p \to q \equiv \lnot p \lor q\). From Exercise 54(a) we see that \(\lnot \phi \equiv \phi \downarrow \phi\) for a compound proposition \(\phi\), so \(\lnot p \lor q \equiv \left(p \downarrow p\right) \lor q\). From Exercise 54(b) we also see that \(\phi \lor \lambda \equiv \left(\phi \downarrow \lambda \right) \downarrow \left(\phi \downarrow \lambda\right)\). Therefore, a compound proposition logically equivalent to \(p \to q\) using only the logical operator \(\downarrow\) is \(\left(\left(p \downarrow p\right) \downarrow q\right) \downarrow \left(\left(\left(p \downarrow p\right)\right) \downarrow q \right)\).

\section*{Exercise 56}
Show that \(\left\{\mathbin{\mid}\right\}\) is a functionally complete collection of logical operators.

\noindent
\textbf{Solution:}
From Exercise 48, it is known that \(\lnot\) and \(\land\) form a functionally complete collection of logical operators.

Table \ref{Truth Table for 56 (NOT)} shows that for all propositions \(\phi\), \(\lnot \phi \equiv \phi \mathbin{\mid} \phi\).

\begin{table}[ht]
\centering
\begin{tabular}{c|c|c}
    \(p\) & \(\lnot p\) & \(p \mathbin{\mid} p\) \\
    T     & F           & F                      \\
    F     & T           & T
\end{tabular}
\caption{Truth Table for 56 (NOT)}
\label{Truth Table for 56 (NOT)}
\end{table}

Table \ref{Truth Table for 56 (NAND)} shows that for all propositions \(\phi\) and \(\lambda\), \(\phi \land \lambda \equiv \left(\phi \mathbin{\mid} \lambda\right) \mathbin{\mid} \left(\phi \mathbin{\mid} \lambda\right)\).

\begin{table}[ht]
\centering
\begin{tabular}{c|c|c|c|c}
    \(p\) & \(q\) & \(p \land q\) & \(p \mathbin{\mid} q\) & \(\left(\phi \mathbin{\mid} \lambda\right) \mathbin{\mid} \left(\phi \mathbin{\mid} \lambda\right)\) \\
    T     & T     & T             & F                      & T                                                                                                    \\
    T     & F     & F             & T                      & F                                                                                                    \\
    F     & T     & F             & T                      & F                                                                                                    \\
    F     & F     & F             & T                      & F
\end{tabular}
\caption{Truth Table for 56 (NAND)}
\label{Truth Table for 56 (NAND)}
\end{table}

Therefore, \(\left\{\mathbin{\mid}\right\}\) is a functionally complete collection of logical operators.

\section*{Exercise 57}
Show that \(p \mathbin{\mid} q\) and \(q \mathbin{\mid} p\) are equivalent.

\noindent
\textbf{Solution:}
\begin{table}[ht]
\centering
\begin{tabular}{c|c|c|c}
    \(p\) & \(q\) & \(p \mathbin{\mid} q\) & \(q \mathbin{\mid} p\) \\
    T     & T     & F                      & F                      \\
    T     & F     & T                      & T                      \\
    F     & T     & T                      & T                      \\
    F     & F     & T                      & T
\end{tabular}
\caption{Truth Table for 57}
\end{table}

\section*{Exercise 58}
Show that \(p \mathbin{\mid} \left(q \mathbin{\mid} r\right)\) and \(\left(p \mathbin{\mid} q\right) \mathbin{\mid} r\) are not equivalent, so that the logical operator \(\mathbin{\mid}\) is not associative.

\noindent
\textbf{Solution:}
\begin{table}[ht]
\centering
\begin{tabular}{c|c|c|c|c|c|c}
    \(p\) & \(q\) & \(r\) & \(p \mathbin{\mid} q\) & \(\left( p \mathbin{\mid} q\right) \mathbin{\mid} r\) & \(q \mathbin{\mid} r\) & \(p \mathbin{\mid} \left(q \mathbin{\mid} r\right) \)\\
    T     & T     & T     & F                      & T                                                     & F                      & T                                                    \\
    T     & T     & F     & F                      & T                                                     & T                      & F                                                    \\
    T     & F     & T     & T                      & F                                                     & T                      & F                                                    \\
    T     & F     & F     & T                      & T                                                     & T                      & F                                                    \\
    F     & T     & T     & T                      & F                                                     & F                      & T                                                    \\
    F     & T     & F     & T                      & T                                                     & T                      & T                                                    \\
    F     & F     & T     & T                      & F                                                     & T                      & T                                                    \\
    F     & F     & F     & T                      & T                                                     & T                      & T
\end{tabular}
\caption{Truth Table for 58}
\end{table}

\section*{Exercise 59}
How many different truth tables of compound propositions are there that involve the propositional variables \(p\) and \(q\)?

\noindent
\textbf{Solution:}
16

\section*{Exercise 60}
Show that if \(p\), \(q\), and \(r\) are compound propositions such that \(p\) and \(q\) are logically equivalent and \(q\) and \(r\) are logically equivalent, then \(p\) and \(r\) are logically equivalent.

\noindent
\textbf{Solution:}
If \(p \equiv q\), then the truth table columns for \(p\) and \(q\) are equivalent. If \(q \equiv r\), then the truth table columns for \(q\) and \(r\) are equivalent. Therefore, since this necessarily implies that the truth table columns for \(p\) and \(r\) are the same, then \(p \equiv r\).

\section*{Exercise 61}
The following sentence is taken from the specification of a telephone system: "If the directory database is opened, then the monitor is put in a closed state, if the system is not in its initial state." This specification is hard to understand because it involves two conditional statements. Find an equivalent, easier-to-understand specification that involved disjunctions and negations but not conditional statements.

\noindent
\textbf{Solution:}
Let \(p\) be the proposition "The directory database is opened," \(q\) be the proposition "The monitor is put in a closed state," and \(r\) be the proposition "The system is not in its initial state."

Right now, the statement is in the form \(\left(p \land r\right) \to q\). Then, \(\left(p \land r\right) \to q \equiv \lnot\left(p \land r\right) \lor q \equiv \lnot p \lor \lnot r \lor q\). Finally, this translates into natural language as "The directory database is not opened or the system is in its initial state or the monitor is put in a closed state.

\section*{Exercise 62}
How many of the disjunctions \(p \lor \lnot q\), \(\lnot p \lor q\), \(q \lor r\), \(q \lor \lnot r\), and \(\lnot q \lor \lnot r\) can be made simultaneously true by an assignment of truth values to \(p\), \(q\), and \(r\)?

\noindent
\textbf{Solution:}
All \(5\) of these statements can be made true with \(p \equiv \textbf{T}\), \(q \equiv \textbf{T}\), and \(r \equiv \textbf{F}\).

\section*{Exercise 63}
How many of the disjunctions \(p \lor \lnot q \lor s\), \(\lnot p \lor \lnot r \lor s\), \(\lnot p \lor \lnot r \lor \lnot s\), \(\lnot p \lor q \lor \lnot s\), \(q \lor r \lor \lnot s\), \(q \lor \lnot r \lor \lnot s\), \(\lnot p \lor \lnot q \lor \lnot s\), \(p \lor r \lor s\), and \(p \lor r \lor \lnot s\) can be made simultaneously true by an assignment of truth values to \(p\), \(q\), \(r\), and \(s\)?

\noindent
\textbf{Solution:}
One solution that maximizes the number of true disjunctions is \(p = 0\), \(q = 1\), \(r = 1\), \(s = 1\), which makes all \(9\) disjunctions true.

\section*{Exercise 64}
Show that the negation of an unsatisfiable compound proposition is a tautology and the negation of a compound proposition that is a tautology is unsatisfiable.

\noindent
\textbf{Solution:}
Consider the truth table for an unsatisfiable compound proposition. The compound proposition's column is all filled with \textbf{F}. Negating this gives a compound proposition with a column filled with \textbf{T}, which is the definition of a tautology. This same argument applies for the second point of showing that the negation of a tautology is unsatisfiable.

\section*{Exercise 65}
Determine whether each of these compound propositions is satisfiable.
\begin{enumerate}
    \item[\textbf{a)}] \((p \lor \lnot q) \land (\lnot p \lor q) \land (\lnot p \lor \lnot q)\)
    \item[\textbf{b)}] \((p \to q) \land (p \to \lnot q) \land (\lnot p \to q) \land (\lnot p \to \lnot q)\)
    \item[\textbf{c)}] \((p \leftrightarrow q) \land (\lnot p \leftrightarrow q)\)
\end{enumerate}

\noindent
\textbf{Solution:}
\begin{enumerate}
    \item[\textbf{a)}] This is satisfiable with \(p = 0\) and \(q = 0\).
    \item[\textbf{b)}] This is unsatisfiable as there will always be a conditional with the form \(\textbf{T} \to \textbf{F}\).
    \item[\textbf{c)}] This is unsatisfiable as there will be no combination of \(p\) and \(q\) to get both biconditionals to be of form \(\textbf{T} \leftrightarrow \textbf{T}\) or \(\textbf{F} \leftrightarrow \textbf{F}\)
\end{enumerate}

\section*{Exercise 66}
Determine whether each of these compound propositions is satisfiable
\begin{enumerate}
    \item[\textbf{a)}] \((p \lor q \lor \lnot r) \land (p \lor \lnot q \lor \lnot s) \land (p \lor \lnot r \lor \lnot s) \land (\lnot p \lor \lnot q \lor \lnot s) ]\land (p \lor q \lor \lnot s)\)
    \item[\textbf{b)}] \((\lnot p \lor \lnot q \lor r) \land (\lnot p \lor q \lor \lnot s) \land (p \lor \lnot q \lor \lnot s) \land (\lnot p \lor \lnot r \lor \lnot s) \land (p \lor q \lor \lnot r) \land (p \lor \lnot r \lor \lnot s)\)
    \item[\textbf{c)}] \((p \lor q \lor r) \land (p \lor \lnot q \lor \lnot s) \land (q \lor \lnot r \lor s) \land (\lnot p \lor r \lor s) \land (\lnot p \lor q \lor \lnot s) \land (p \lor \lnot q \lor \lnot r) \land(\lnot p \lor \lnot q \lor s) \land (\lnot p \lor \lnot r \lor \lnot s)\)
\end{enumerate}

\noindent
\textbf{Solution:}
\begin{enumerate}
    \item[\textbf{a)}] This compound proposition is satisfiable.
    \item[\textbf{b)}] This compound proposition is satisfiable.
    \item[\textbf{c)}] This compound proposition is satisfiable.
\end{enumerate}

\section*{Exercise 67}
Find the compound proposition \(Q\) constructed in Example 10 for the \(n\)-queens problem, and use it to find all the ways that \(n\) queens can be placed on an \(n \times n\) chessboard, so that no queen can attack another when \(n\) is
\begin{enumerate}
    \item[\textbf{a)}] \(2\)
    \item[\textbf{b)}] \(3\)
    \item[\textbf{c)}] \(4\)
\end{enumerate}

\noindent
\textbf{Solution:}

\printbibliography

\end{document}