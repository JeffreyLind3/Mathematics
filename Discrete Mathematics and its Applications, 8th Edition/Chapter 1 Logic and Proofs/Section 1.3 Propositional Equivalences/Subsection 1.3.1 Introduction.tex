%%%%%%%%%%%%%%%%%%%%%%%%%%%%%%%%%%%%%%%%%%%%%%%%%%%%%%%%%%%%%%%%
%%%%%%%%%%%%%%%%%%%%%%%%%%% Metadata %%%%%%%%%%%%%%%%%%%%%%%%%%%
%%%%%%%%%%%%%%%%%%%%%%%%%%%%%%%%%%%%%%%%%%%%%%%%%%%%%%%%%%%%%%%%
\documentclass{Axon}

\title{Discrete Mathematics and its Applications, 8th Edition - Chapter 1 The Foundations: Logic and Proofs - Section 1.3 Propositional Equivalences - Subsection 1.3.1 Introduction}

\authors{
    \addauthor{Jeffrey G. Lind III}{jeffrey@jeffreylind.dev}
}

\addbibresource{Bibliography.bib}
%%%%%%%%%%%%%%%%%%%%%%%%%%%%%%%%%%%%%%%%%%%%%%%%%%%%%%%%%%%%%%%%
%%%%%%%%%%%%%%%%%%%%%%%%%%%%% Paper %%%%%%%%%%%%%%%%%%%%%%%%%%%%
%%%%%%%%%%%%%%%%%%%%%%%%%%%%%%%%%%%%%%%%%%%%%%%%%%%%%%%%%%%%%%%%
\begin{document}
\maketitle
\makeauthor
%%%%%%%%%%%%%%%%%%%%%%%%%%%%%%%%%%%%%%%%%%%%%%%%%%%%%%%%%%%%%%%%
%%%%%%%%%%%%%%%%%%%%%%%%%%% Abstract %%%%%%%%%%%%%%%%%%%%%%%%%%%
%%%%%%%%%%%%%%%%%%%%%%%%%%%%%%%%%%%%%%%%%%%%%%%%%%%%%%%%%%%%%%%%
\begin{abstract}
Notes on Discrete Mathematics and its Applications, 8th Edition - Chapter 1 The Foundations: Logic and Proofs - Section 1.3 Propositional Equivalences - Subsection 1.3.1 Introduction \cite{Rosen}.
\end{abstract}
%%%%%%%%%%%%%%%%%%%%%%%%%%%%%%%%%%%%%%%%%%%%%%%%%%%%%%%%%%%%%%%%
%%%%%%%%%%%%%%%%%%%%%%%%%%% Section 1 %%%%%%%%%%%%%%%%%%%%%%%%%%
%%%%%%%%%%%%%%%%%%%%%%%%%%%%%%%%%%%%%%%%%%%%%%%%%%%%%%%%%%%%%%%%
\section{Introduction}
An important type of step used in a mathematical argument is the replacement of a statement with another statement with the same truth value. Because of this, methods that produce propositions with the same truth value as a given compound proposition are used extensively in the construction of mathematical arguments. Note that we will use the term "compound proposition" to refer to an expression formed from propositional variables using logical operators, such as \(p \land q\).

We begin our discussion with a classification of compound propositions according to their possible truth values.

\begin{definition}
    A compound proposition that is always true, no matter what the truth values of the propositional variables that occur in it, is called a \textit{tautology}. A compound proposition that is always false is called a \textit{contradiction}. A compound proposition that is neither a tautology nor a contradiction is called a \textit{contingency}.
\end{definition}

Tautologies and contradictions are often important in mathematical reasoning. Example \ref{Example: 1} illustrates these types of compound propositions.

\begin{example}\label{Example: 1}
    We can construct examples of tautologies and contradictions using just one propositional variable. Consider the truth tables of \(p \lor \lnot p\) and \(p \land \lnot p\), shown in Table \ref{Table: 1} Because \(p \lor \lnot p\) is always true, it is a tautology. Because \(p \land \lnot p\) is always false, it is a contradiction.
\end{example}

\begin{table}[h]
    \centering
    \begin{tabular}{c|c|c|c}
        \(p\) & \(\lnot p\) & \(p \lor \lnot p\) & \(p \land \lnot p\) \\
        T     & F           & T                  & F                   \\
        F     & T           & T                  & F                  
    \end{tabular}
    \caption{Examples of a Tautology and a Contradiction.}
    \label{Table: 1}
\end{table}

\printbibliography

\end{document}