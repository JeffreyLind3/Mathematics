%%%%%%%%%%%%%%%%%%%%%%%%%%%%%%%%%%%%%%%%%%%%%%%%%%%%%%%%%%%%%%%%
%%%%%%%%%%%%%%%%%%%%%%%%%%% Metadata %%%%%%%%%%%%%%%%%%%%%%%%%%%
%%%%%%%%%%%%%%%%%%%%%%%%%%%%%%%%%%%%%%%%%%%%%%%%%%%%%%%%%%%%%%%%
\documentclass{Axon}

\title{Discrete Mathematics and its Applications, 8th Edition - Chapter 1 The Foundations: Logic and Proofs - Section 1.1 Propositional Logic - Exercises}

\authors{
    \addauthor{Jeffrey G. Lind III}{jeffrey@jeffreylind.dev}
}

\addbibresource{Bibliography.bib}
%%%%%%%%%%%%%%%%%%%%%%%%%%%%%%%%%%%%%%%%%%%%%%%%%%%%%%%%%%%%%%%%
%%%%%%%%%%%%%%%%%%%%%%%%%%%%% Paper %%%%%%%%%%%%%%%%%%%%%%%%%%%%
%%%%%%%%%%%%%%%%%%%%%%%%%%%%%%%%%%%%%%%%%%%%%%%%%%%%%%%%%%%%%%%%
\begin{document}
\maketitle
\makeauthor
%%%%%%%%%%%%%%%%%%%%%%%%%%%%%%%%%%%%%%%%%%%%%%%%%%%%%%%%%%%%%%%%
%%%%%%%%%%%%%%%%%%%%%%%%%%% Abstract %%%%%%%%%%%%%%%%%%%%%%%%%%%
%%%%%%%%%%%%%%%%%%%%%%%%%%%%%%%%%%%%%%%%%%%%%%%%%%%%%%%%%%%%%%%%
\begin{abstract}
Notes on Discrete Mathematics and its Applications, 8th Edition - Chapter 1 The Foundations: Logic and Proofs - Section 1.1 Propositional Logic - Exercises \cite{Rosen}.
\end{abstract}
%%%%%%%%%%%%%%%%%%%%%%%%%%%%%%%%%%%%%%%%%%%%%%%%%%%%%%%%%%%%%%%%
%%%%%%%%%%%%%%%%%%%%%%%%%%% Section 1 %%%%%%%%%%%%%%%%%%%%%%%%%%
%%%%%%%%%%%%%%%%%%%%%%%%%%%%%%%%%%%%%%%%%%%%%%%%%%%%%%%%%%%%%%%%
\section*{Exercise 1}
Which of these sentences are propositions? What are the truth values of those that are propositions?

\begin{enumerate}
    \item[\textbf{a)}] Boston is the capital of Massachusetts.
    \item[\textbf{b)}] Miami is the capital of Florida.
    \item[\textbf{c)}] \(2 + 3 = 5\).
    \item[\textbf{d)}] \(5 + 7 = 10\).
    \item[\textbf{e)}] \(x + 2 = 11\).
    \item[\textbf{f)}] Answer this question.
\end{enumerate}

\noindent
\textbf{Solution:}
\begin{enumerate}
    \item[\textbf{a)}] This sentence is a proposition with a truth value of T.
    \item[\textbf{b)}] This sentence is a proposition with a truth value of F.
    \item[\textbf{c)}] This sentence is a proposition with a truth value of T.
    \item[\textbf{d)}] This sentence is a proposition with a truth value of F.
    \item[\textbf{e)}] This sentence is not a proposition.
    \item[\textbf{f)}] This sentence is not a proposition.
\end{enumerate}

\section*{Exercise 2}
Which of these are propositions? What are the truth values of those that are propositions?
\begin{enumerate}
    \item[\textbf{a)}] Do not pass go.
    \item[\textbf{b)}] What time is it?
    \item[\textbf{c)}] There are no black flies in Maine.
    \item[\textbf{d)}] \(4 + x = 5\).
    \item[\textbf{e)}] The moon is made of green cheese.
    \item[\textbf{f)}] \(2^n \geq 100\).
\end{enumerate}

\noindent
\textbf{Solution:}
\begin{enumerate}
    \item[\textbf{a)}] This sentence is not a proposition.
    \item[\textbf{b)}] This sentence is not a proposition.
    \item[\textbf{c)}] This sentence is a proposition with a truth value of F.
    \item[\textbf{d)}] This sentence is not a proposition.
    \item[\textbf{e)}] This sentence is a proposition with a truth value of F.
    \item[\textbf{f)}] This sentence is not a proposition.
\end{enumerate}

\section*{Exercise 3}
What is the negation of each of these propositions?
\begin{enumerate}
    \item[\textbf{a)}] Linda is younger than Sanjay.
    \item[\textbf{b)}] Mei makes more money than Isabella.
    \item[\textbf{c)}] Moshe is taller than Monica.
    \item[\textbf{d)}] Abby is richer than Ricardo.
\end{enumerate}

\noindent
\textbf{Solution:}
\begin{enumerate}
    \item[\textbf{a)}] Linda is not younger than Sanjay.
    \item[\textbf{b)}] Mei does not make more money than Isabella.
    \item[\textbf{c)}] Moshe is not taller than Monica.
    \item[\textbf{d)}] Abby is not richer than Ricardo.
\end{enumerate}

\section*{Exercise 4}
What is the negation of each of these propositions?
\begin{enumerate}
    \item[\textbf{a)}] Janice has more Facebook friends than Juan.
    \item[\textbf{b)}] Quincy is smarter than Venkat.
    \item[\textbf{c)}] Zelda drives more miles to school than Paola.
    \item[\textbf{d)}] Briana sleeps longer than Gloria.
\end{enumerate}

\noindent
\textbf{Solution:}
\begin{enumerate}
    \item[\textbf{a)}] Janice does not have more Facebook friends than Juan.
    \item[\textbf{b)}] Quincy is not smarter than Venkat.
    \item[\textbf{c)}] Zelda does not drive more miles to school than Paola.
    \item[\textbf{d)}] Briana does not sleep longer than Gloria.
\end{enumerate}

\section*{Exercise 5}
What is the negation of each of these propositions?
\begin{enumerate}
    \item[\textbf{a)}] Mei has an MP3 player.
    \item[\textbf{b)}] There is no pollution in New Jersey.
    \item[\textbf{c)}] \(2 + 1 = 3\).
    \item[\textbf{d)}] The summer in Maine is hot and sunny.
\end{enumerate}

\noindent
\textbf{Solution:}
\begin{enumerate}
    \item[\textbf{a)}] Mei does not have an MP3 player.
    \item[\textbf{b)}] There is pollution in New Jersey.
    \item[\textbf{c)}] \(2 + 1 \neq 3\).
    \item[\textbf{d)}] The summer in Maine is not hot or it is not sunny.
\end{enumerate}
\section*{Exercise 6}
What is the negation of each of these propositions?
\begin{enumerate}
    \item[\textbf{a)}] Jennifer and Teja are friends.
    \item[\textbf{b)}] There are \(13\) items in a baker's dozen.
    \item[\textbf{c)}] Abby sent more than \(100\) text messages yesterday.
    \item[\textbf{d)}] \(121\) is a perfect square.
\end{enumerate}

\noindent
\textbf{Solution:}
\begin{enumerate}
    \item[\textbf{a)}] Jennifer and Teja are not friends.
    \item[\textbf{b)}] There are not \(13\) items in a baker's dozen.
    \item[\textbf{c)}] Abby did not send more than \(100\) text messages yesterday.
    \item[\textbf{d)}] \(121\) is not a perfect square.
\end{enumerate}

\section*{Exercise 7}
What is the negation of each of these propositions?
\begin{enumerate}
    \item[\textbf{a)}] Steve has more than \(100\) GB free disk space on his laptop.
    \item[\textbf{b)}] Zach blocks e-mails and texts from Jennifer.
    \item[\textbf{c)}] \(7 \cdot 11 \cdot 13 = 999\).
    \item[\textbf{d)}] Diane rode her bicycle \(100\) miles on Sunday.
\end{enumerate}

\noindent
\textbf{Solution:}
\begin{enumerate}
    \item[\textbf{a)}] Steve does not have more than \(100\) GB free disk space on his laptop.
    \item[\textbf{b)}] Zach does not block e-mails or does not block texts from Jennifer.
    \item[\textbf{c)}] \(7 \cdot 11 \cdot 13 \neq 999\).
    \item[\textbf{d)}] Diane did not ride her bicycle \(100\) miles on Sunday.
\end{enumerate}

\section*{Exercise 8}
Suppose that Smartphone A has \(256\) MB RAM and \(32\) GB ROM, and the resolution of its camera is \(8\) MP; Smartphone B has \(288\) MB RAM and \(64\) GB ROM, and the resolution of its camera is \(4\) MP; and Smartphone C has \(128\) MB RAM and \(32\) GB ROM, and the resolution of its camera is \(5\) MP. Determine the truth value of each of these propositions.
\begin{enumerate}
    \item[\textbf{a)}] Smartphone B has the most RAM of these three smartphones.
    \item[\textbf{b)}] Smartphone C has more ROM or a higher resolution camera than Smartphone B.
    \item[\textbf{c)}] Smartphone B has more RAM, more ROM, and a higher resolution camera than Smartphone A.
    \item[\textbf{d)}] If Smartphone B has more RAM and more ROM than Smartphone C, then it also has a higher resolution camera.
    \item[\textbf{e)}] Smartphone A has more RAM than Smartphone B if and only if Smartphone B has more RAM than Smartphone A.
\end{enumerate}

\noindent
\textbf{Solution:}\begin{enumerate}
    \item[\textbf{a)}] This proposition has truth value T.
    \item[\textbf{b)}] This proposition has truth value T.
    \item[\textbf{c)}] This proposition has truth value F.
    \item[\textbf{d)}] This proposition has truth value F.
    \item[\textbf{e)}] This proposition has truth value F.
\end{enumerate}

\section*{Exercise 9}
Suppose that during the most recent fiscal year, the annual revenue of Acme Computer was \(138\) billion dollars and its net profit was \(8\) billion dollars, the annual revenue of Nadir Software was \(87\) billion dollars and its net profit was \(5\) billion dollars, and the annual revenue of Quixote Media was \(111\) billion dollars and its net profit was \(13\) billion dollars. Determine the truth value of each of these propositions for the most recent fiscal year.
\begin{enumerate}
    \item[\textbf{a)}] Quixote Media had the largest annual revenue.
    \item[\textbf{b)}] Nadir Software had the lowest net profit and Acme Computer had the largest annual revenue.
    \item[\textbf{c)}] Acme Computer had the largest net profit or Quixote Media had the largest net profit.
    \item[\textbf{d)}] If Quixote Media had the smallest net profit, then Acme Computer had the largest annual revenue.
    \item[\textbf{e)}] Nadir Software had the smallest net profit if and only if Acme Computer had the largest annual revenue.
\end{enumerate}

\noindent
\textbf{Solution:}\begin{enumerate}
    \item[\textbf{a)}] This proposition has truth value F.
    \item[\textbf{b)}] This proposition has truth value T.
    \item[\textbf{c)}] This proposition has truth value T.
    \item[\textbf{d)}] This proposition has truth value T.
    \item[\textbf{e)}] This proposition has truth value T.
\end{enumerate}

\section*{Exercise 10}
Let \(p\) and \(q\) be the propositions

\begin{equation*}
    p\text{: I bought a lottery ticket this week.}
\end{equation*}

\begin{equation*}
    q\text{: I won the million dollar jackpot.}
\end{equation*}
Express each of these propositions as an English sentence.
\begin{enumerate}
    \item[\textbf{a)}] \(\lnot p\)
    \item[\textbf{b)}] \(p \lor q\)
    \item[\textbf{c)}] \(p \to q\)
    \item[\textbf{d)}] \(p \land q\)
    \item[\textbf{e)}] \(p \leftrightarrow q\)
    \item[\textbf{f)}] \(\lnot p \to \lnot q\)
    \item[\textbf{g)}] \(\lnot p \land \lnot q\)
    \item[\textbf{h)}] \(\lnot p \lor (p \land q)\)
\end{enumerate}

\noindent
\textbf{Solution:}
\begin{enumerate}
    \item[\textbf{a)}] I did not buy a lottery ticket this week.
    \item[\textbf{b)}] I bought a lottery ticket this week or I won the million dollar jackpot.
    \item[\textbf{c)}] If I bought a lottery ticket this week then I won the million dollar jackpot.
    \item[\textbf{d)}] I bought a lottery ticket this week and I won the million dollar jackpot.
    \item[\textbf{e)}] I bought a lottery ticket this week if and only if I won the million dollar jackpot.
    \item[\textbf{f)}] If I did not buy a lottery ticket this week then I did not win the million dollar jackpot.
    \item[\textbf{g)}] I did not buy a lottery ticket this week and I did not win the million dollar jackpot.
    \item[\textbf{h)}] Either I did not buy a lottery ticket this week, or else I did buy one and won the million dollar jackpot.
\end{enumerate}

\section*{Exercise 11}
Let \(p\) and \(q\) be the propositions "Swimming at the New Jersey shore is allowed" and "Sharks have been spotted near the shore," respectively. Express each of these compound propositions as an English sentence.
\begin{enumerate}
    \item[\textbf{a)}] \(\lnot q\)
    \item[\textbf{b)}] \(p \land q\)
    \item[\textbf{c)}] \(\lnot p \lor q\)
    \item[\textbf{d)}] \(p \to \lnot q\)
    \item[\textbf{e)}] \(\lnot q \to p\)
    \item[\textbf{f)}] \(\lnot p \to \lnot q\)
    \item[\textbf{g)}] \(p \leftrightarrow \lnot q\)
    \item[\textbf{h)}] \(\lnot p \land (p \lor \lnot q)\)
\end{enumerate}

\noindent
\textbf{Solution:}
\begin{enumerate}
    \item[\textbf{a)}] Sharks have not been spotted near the shore.
    \item[\textbf{b)}] Swimming at the New Jersey shore is allowed and sharks have been spotted near the shore.
    \item[\textbf{c)}] Swimming at the New Jersey shore is not allowed or sharks have been spotted near the shore.
    \item[\textbf{d)}] If swimming at the New Jersey shore is allowed then sharks have not been spotted near the shore.
    \item[\textbf{e)}] If sharks have not been spotted near the shore then swimming at the New Jersey shore is allowed.
    \item[\textbf{f)}] If swimming at the New Jersey shore is not allowed then sharks have not been spotted near the shore. 
    \item[\textbf{g)}] Swimming at the New Jersey shore is allowed if and only if sharks have not been spotted near the shore.
    \item[\textbf{h)}] Swimming at the New Jersey shore is not allowed and either (in the inclusive sense) swimming at the New Jersey shore is allowed or sharks have not been spotted near the shore.
\end{enumerate}

\section*{Exercise 12}
Let \(p\) and \(q\) be the propositions "The election is decided" and "The votes have been counted," respectively. Express each of these compound propositions as an English sentence.
\begin{enumerate}
    \item[\textbf{a)}] \(\lnot p\)
    \item[\textbf{b)}] \(p \lor q\)
    \item[\textbf{c)}] \(\lnot p \land q\)
    \item[\textbf{d)}] \(q \to p\)
    \item[\textbf{e)}] \(\lnot q \to \lnot p\)
    \item[\textbf{f)}] \(\lnot p \to \lnot q\)
    \item[\textbf{g)}] \(p \leftrightarrow q\)
    \item[\textbf{h)}] \(\lnot q \lor (\lnot p \land q)\)
\end{enumerate}

\noindent
\textbf{Solution:}
\begin{enumerate}
    \item[\textbf{a)}] The election is not decided.
    \item[\textbf{b)}] The election is decided or the votes have been counted.
    \item[\textbf{c)}] The election is not decided and the votes have been counted.
    \item[\textbf{d)}] If the votes have been counted then the election is decided.
    \item[\textbf{e)}] If the votes have not been counted then the election is not decided.
    \item[\textbf{f)}] If the election is not decided then the votes have not been counted.
    \item[\textbf{g)}] The election is decided if and only if the votes have been counted.
    \item[\textbf{h)}] Either the votes have not been counted, or, the election is not decided and the votes have been counted.
\end{enumerate}

\section*{Exercise 13}
Let \(p\) and \(q\) be the propositions
\begin{equation}
    p\text{: It is below freezing.}
\end{equation}
\begin{equation}
    q\text{: It is snowing.}
\end{equation}
Write these propositions using \(p\) and \(q\) and logical connectives (including negations)
\begin{enumerate}
    \item[\textbf{a)}] It is below freezing and snowing.
    \item[\textbf{b)}] It is below freezing but not snowing.
    \item[\textbf{c)}] It is not below freezing and it is not snowing.
    \item[\textbf{d)}] It is either snowing or below freezing (or both).
    \item[\textbf{e)}] If it is below freezing, it is also snowing.
    \item[\textbf{f)}] Either it is below freezing or it is snowing, but it is not snowing if it is below freezing.
    \item[\textbf{g)}] That it is below freezing is necessary and sufficient  for it to be snowing.
\end{enumerate}

\noindent
\textbf{Solution:}
\begin{enumerate}
    \item[\textbf{a)}] \(p \land q\)
    \item[\textbf{b)}] \(p \land \lnot q\)
    \item[\textbf{c)}] \(\lnot p \land \lnot q\)
    \item[\textbf{d)}] \(q \lor p\)
    \item[\textbf{e)}] \(p \to q\)
    \item[\textbf{f)}] \((p \lor q) \land (p \to \lnot q)\)
    \item[\textbf{g)}] \(p \leftrightarrow q\)
\end{enumerate}

\section*{Exercise 14}
Let \(p\), \(q\), and \(r\) be the propositions
\begin{equation}
    p\text{: You have the flu.}
\end{equation}
\begin{equation}
    q\text{: You miss the final examination.}
\end{equation}
\begin{equation}
    r\text{: You pass the course.}
\end{equation}
Express each of these propositions as an English sentence.
\begin{enumerate}
    \item[\textbf{a)}] \(p \to q\)
    \item[\textbf{b)}] \(\lnot q \leftrightarrow r\)
    \item[\textbf{c)}] \(q \to \lnot r\)
    \item[\textbf{d)}] \(p \lor q \lor r\)
    \item[\textbf{e)}] \((p \to \lnot r) \lor (q \to \lnot r)\)
    \item[\textbf{f)}] \((p \land q) \lor (\lnot q \land r)\)
\end{enumerate}

\noindent
\textbf{Solution:}
\begin{enumerate}
    \item[\textbf{a)}] If you have the flu then you miss the final examination.
    \item[\textbf{b)}] You do not miss the final examination if and only if you pass the course.
    \item[\textbf{c)}] If you miss the final examination then you do not pass the course.
    \item[\textbf{d)}] You have the flu or you miss the final examination or you pass the course.
    \item[\textbf{e)}] Either if you have the flu then you do not pass the course, or if you miss the final examination then you do not pass the course.
    \item[\textbf{f)}] Either you have the flu and you miss the final examination, or you do not miss the final examination and you pass the course.
\end{enumerate}

\section*{Exercise 15}
Let \(p\) and \(q\) be the propositions
\begin{equation}
    p\text{: You drive over} \ 65 \ \text{miles per hour.}
\end{equation}
\begin{equation}
    q\text{: You get a speeding ticket.}
\end{equation}
Write these propositions using \(p\) and \(q\) and logical connectives (including negations).
\begin{enumerate}
    \item[\textbf{a)}] You do not drive over \(65\) miles per hour.
    \item[\textbf{b)}] You drive over \(65\) miles per hour, but you do not get a speeding ticket.
    \item[\textbf{c)}] You will get a speeding ticket if you drive over \(65\) miles per hour.
    \item[\textbf{d)}] If you do not drive over \(65\) miles per hour, then you will not get a speeding ticket.
    \item[\textbf{e)}] Driving over \(65\) miles per hour is sufficient for getting a speeding ticket.
    \item[\textbf{f)}] You get a speeding ticket, but you do not drive over \(65\) miles per hour.
    \item[\textbf{g)}] Whenever you get a speeding ticket, you are driving over \(65\) miles per hour.
\end{enumerate}

\noindent
\textbf{Solution:}
\begin{enumerate}
    \item[\textbf{a)}] \(\lnot p\)
    \item[\textbf{b)}] \(p \land \lnot q\)
    \item[\textbf{c)}] \(p \to q\)
    \item[\textbf{d)}] \(\lnot p \to \lnot q\)
    \item[\textbf{e)}] \(p \to q\)
    \item[\textbf{f)}] \(q \land \lnot p\)
    \item[\textbf{g)}] \(q \to p\)
\end{enumerate}

\section*{Exercise 16}
Let \(p\), \(q\), and \(r\) be the propositions
\begin{equation}
    p\text{: You get an A on the final exam.}
\end{equation}
\begin{equation}
    q\text{: You do every exercise in this book.}
\end{equation}
\begin{equation}
    r\text{: You get an A in the class.}
\end{equation}
Write these propositions using \(p\), \(q\), and \(r\) and logical connectives (including negations).
\begin{enumerate}
    \item[\textbf{a)}] You get an A in this class, but you do not do every exercise in this book.
    \item[\textbf{b)}] You get an A on the final, you do every exercise in this book, and you get an A in this class.
    \item[\textbf{c)}] To get an A in this class, it is necessary for you to get an A on the final.
    \item[\textbf{d)}] You get an A on the final, but you don't do every exercise in this book; nevertheless, you get an A in this class.
    \item[\textbf{e)}] Getting an A on the final and doing every exercise in this book is sufficient for getting an A in this class.
    \item[\textbf{f)}] You will get an A in this class if and only if you either do every exercise in this book or you get an A on the final.
\end{enumerate}

\noindent
\textbf{Solution:}
\begin{enumerate}
    \item[\textbf{a)}] \(r \land \lnot q\)
    \item[\textbf{b)}] \(p \land q \land r\)
    \item[\textbf{c)}] \(r \to p\)
    \item[\textbf{d)}] \(p \land \lnot q \land r\)
    \item[\textbf{e)}] \((p \land q) \to r\)
    \item[\textbf{f)}] \(r \leftrightarrow (q \lor p)\)
\end{enumerate}

\section*{Exercise 17}
Let \(p\), \(q\), and \(r\) be the propositions
\begin{equation}
    p\text{: Grizzly bears have been seen in the area.}
\end{equation}
\begin{equation}
    q\text{: Hiking is safe on the trail.}
\end{equation}
\begin{equation}
    r\text{: Berries are ripe along the trail.}
\end{equation}
Write these propositions using \(p\), \(q\), and \(r\) and logical connectives (including negations).
\begin{enumerate}
    \item[\textbf{a)}] Berries are ripe along the trail, but grizzly bears have not been seen in the area.
    \item[\textbf{b)}] Grizzly bears have not been seen in the area and hiking on the trail is safe, but berries are ripe along the trail.
    \item[\textbf{c)}] If berries are ripe along the trail, hiking is safe if and only if grizzly bears have not been seen in the area.
    \item[\textbf{d)}] It is not safe to hike on the trail, but grizzly bears have not been seen in the area and the berries along the trail are ripe.
    \item[\textbf{e)}] For hiking on the trail to be safe, it is necessary but not sufficient that berries not be ripe along the trail and for grizzly bears not to have been seen in the area.
    \item[\textbf{f)}] Hiking is not safe on the trail whenever grizzly bears have been seen in the area and berries are ripe along the trail.
\end{enumerate}

\noindent
\textbf{Solution:}
\begin{enumerate}
    \item[\textbf{a)}] \(r \land \lnot p\)
    \item[\textbf{b)}] \(\lnot p \land q \land r\)
    \item[\textbf{c)}] \(r \to (q \leftrightarrow \lnot p)\)
    \item[\textbf{d)}] \(\lnot q \land \lnot p \land r\)
    \item[\textbf{e)}] \((q \to (\lnot r \land \lnot p)) \land \lnot((\lnot r \land \lnot p) \to q)\)
    \item[\textbf{f)}] \((p \land r) \to \lnot q\)
\end{enumerate}

\section*{Exercise 18}
Determine whether these biconditionals are true or false.
\begin{enumerate}
    \item[\textbf{a)}] \(2 + 2 = 4\) if and only if \(1 + 1 = 2\).
    \item[\textbf{b)}] \(1 + 1 = 2\) if and only if \(2 + 3 = 4\).
    \item[\textbf{c)}] \(1 + 1 = 3\) if and only if monkeys can fly.
    \item[\textbf{d)}] \(0 > 1\) if and only if \(2 > 1\).
\end{enumerate}

\noindent
\textbf{Solution:}
\begin{enumerate}
    \item[\textbf{a)}] This biconditional is true.
    \item[\textbf{b)}] This biconditional is false.
    \item[\textbf{c)}] This biconditional is true.
    \item[\textbf{d)}] This biconditional is false.
\end{enumerate}

\section*{Exercise 19}
Determine whether each of these conditional statements is true or false.
\begin{enumerate}
    \item[\textbf{a)}] If \(1 + 1 = 2\), then \(2 + 2 = 5\).
    \item[\textbf{b)}] If \(1 + 1 = 3\), then \(2 + 2 = 4\).
    \item[\textbf{c)}] If \(1 + 1 = 3\), then \(2 + 2 = 5\).
    \item[\textbf{d)}] If monkeys can fly, then \(1 + 1 = 3\).
\end{enumerate}

\noindent
\textbf{Solution:}
\begin{enumerate}
    \item[\textbf{a)}] This conditional statement is false.
    \item[\textbf{b)}] This conditional statement is true.
    \item[\textbf{c)}] This conditional statement is true.
    \item[\textbf{d)}] This conditional statement is true.
\end{enumerate}

\section*{Exercise 20}
Determine whether each of these conditional statements is true or false.
\begin{enumerate}
    \item[\textbf{a)}] If \(1 + 1 = 3\), then unicorns exist.
    \item[\textbf{b)}] If \(1 + 1 = 3\), then dogs can fly.
    \item[\textbf{c)}] If \(1 + 1 = 2\), then dogs can fly.
    \item[\textbf{d)}] If \(2 + 2 = 4\), then \(1 + 2 = 3\).
\end{enumerate}

\noindent
\textbf{Solution:}
\begin{enumerate}
    \item[\textbf{a)}] This conditional statement is true.
    \item[\textbf{b)}] This conditional statement is true.
    \item[\textbf{c)}] This conditional statement is false.
    \item[\textbf{d)}] This conditional statement is true.
\end{enumerate}

\section*{Exercise 21}
For each of these sentences, determine whether an inclusive or, or an exclusive or, is intended. Explain your answer.
\begin{enumerate}
    \item[\textbf{a)}] Coffee or tea comes with dinner.
    \item[\textbf{b)}] A password must have at least three digits or be at least eight characters long.
    \item[\textbf{c)}] The prerequisite for the course is a course in number theory or a course in cryptography.
    \item[\textbf{d)}] You can pay using U.S. dollars or euros.
\end{enumerate}

\noindent
\textbf{Solution:}
\begin{enumerate}
    \item[\textbf{a)}] In this sentence, an exclusive or is intended, because you can't have both coffee or tea with dinner.
    \item[\textbf{b)}] In this sentence, an inclusive or is intended, because you can fulfill either condition independently, or fulfill both.
    \item[\textbf{c)}] In this sentence, an inclusive or is intended, because you can take either or both of the classes.
    \item[\textbf{d)}] In this sentence, an inclusive or is intended, because you can split the payment.
\end{enumerate}

\section*{Exercise 22}
For each of these sentences, determine whether an inclusive or, or an exclusive or, is intended. Explain your answer.
\begin{enumerate}
    \item[\textbf{a)}] Experience with C++ or Java is required.
    \item[\textbf{b)}] Lunch includes soup or salad.
    \item[\textbf{c)}] To enter the country you need a passport or a voter registration card.
    \item[\textbf{d)}] Publish or perish.
\end{enumerate}

\noindent
\textbf{Solution:}
\begin{enumerate}
    \item[\textbf{a)}] In this sentence, an inclusive or is intended, because you can either have experience with one or both.
    \item[\textbf{b)}] In this sentence, an exclusive or is intended, because you can't have both soup and salad with lunch.
    \item[\textbf{c)}] In this sentence, an inclusive or is intended, because you can either have one or both.
    \item[\textbf{d)}] In this sentence, an exclusive or is intended, because you can't both publish and perish. 
\end{enumerate}

\section*{Exercise 23}
For each of these sentences, state what the sentence means if the logical connective or is an inclusive or (that is, a disjunction) versus an exclusive or. Which of these meanings of or do you think is intended?
\begin{enumerate}
    \item[\textbf{a)}] To take discrete mathematics, you must have taken calculus or a course in computer science.
    \item[\textbf{b)}] When you buy a new car from Acme Motor Company, you get \(\$2000\) back in cash or a \(2\%\) car loan.
    \item[\textbf{c)}] Dinner for two includes two items from column A or three items from column B.
    \item[\textbf{d)}] School is closed if more than two feet of snow falls or if the wind chill is below \(-100\)\textdegree F.
\end{enumerate}

\noindent
\textbf{Solution:}
\begin{enumerate}
    \item[\textbf{a)}] If the logical connective or is an inclusive or, then it means that you can take one of the classes or both. If the logical connective or is an exclusive or, then it means that you can only take one of the classes. The inclusive or is intended.
    \item[\textbf{b)}] If the logical connective or is an inclusive or, then it means you can get both cash back and the car loan. If the logical connective or is an exclusive or, then it means that you can only have the cash back or the car loan.. The exclusive or is intended.
    \item[\textbf{c)}] If the logical connective or is an inclusive or, then it means that you can have both two items from column A and three items from column B. If the logical connective or is an exclusive or, then it means that you can only one have two items from column A or three items from column B. The exclusive or is intended.
    \item[\textbf{d)}] If the logical connective or is an inclusive or, then it means school can be closed if more than two feet of snow falls or if the wind chill is below \(-100\)\textdegree F. If the logical connective or is an exclusive or, then it means that school closes if either there are more than two feet of snow or if the wind chill is below \(-100\)\textdegree F, but now both. The inclusive or is intended.
\end{enumerate}

\section*{Exercise 24}
Write each of these statements in the form "if \(p\), then \(q\)" in English. [\textit{Hint:} Refer to the list of common ways to express conditional statements provided in this section.]
\begin{enumerate}
    \item[\textbf{a)}] It is necessary to wash the boss's car to get promoted.
    \item[\textbf{b)}] Winds from the south imply a spring thaw.
    \item[\textbf{c)}] A sufficient condition for the warranty to be good is that you bought the computer less than a year ago.
    \item[\textbf{d)}] Willy gets caught whenever he cheats.
    \item[\textbf{e)}] You can access the website only if you pay a subscription fee.
    \item[\textbf{f)}] Getting elected follows from knowing the right people.
    \item[\textbf{g)}] Carol gets seasick whenever she is on a boat.
\end{enumerate}

\noindent
\textbf{Solution:}
\begin{enumerate}
    \item[\textbf{a)}] If you are promoted, then you washed the boss's car.
    \item[\textbf{b)}] If there are winds from the south, then there is a spring thaw.
    \item[\textbf{c)}] If you bought the computer less than a year ago, then the warranty is good.
    \item[\textbf{d)}] If Willy cheats, then he gets caught.
    \item[\textbf{e)}] If you can access the website, then you pay a subscription fee.
    \item[\textbf{f)}] If you know the right people, then you get elected.
    \item[\textbf{g)}] If Carol is on a boat, then Carol is seasick.
\end{enumerate}

\section*{Exercise 25}
Write each of these statements in the form "if \(p\), then \(q\)" in English. [\textit{Hint:} Refer to the list of common ways to express conditional statements provided in this section.]
\begin{enumerate}
    \item[\textbf{a)}] It snows whenever the wind blows from the northeast.
    \item[\textbf{b)}] The apple trees will bloom if it stays warm for a week.
    \item[\textbf{c)}] That the Pistons win the championship implies that they beat the Lakers.
    \item[\textbf{d)}] It is necessary to walk eight miles to get to the top of Long's Peak.
    \item[\textbf{e)}] To get tenure as a professor, it is sufficient to be world famous.
    \item[\textbf{f)}] If you drive more than \(400\) miles, you will need to buy gasoline.
    \item[\textbf{g)}] Your guarantee is good only if you bought your CD player less than \(90\) days ago.
    \item[\textbf{h)}] Jan will go swimming unless the water is too cold.
    \item[\textbf{i)}] We will have a future, provided that people believe in science.
\end{enumerate}

\noindent
\textbf{Solution:}
\begin{enumerate}
    \item[\textbf{a)}] If the wind blows from the northeast, then it snows.
    \item[\textbf{b)}] If it stays warm for a week, then the apple trees will bloom.
    \item[\textbf{c)}] If the Pistons win the championship, then they beat the Lakers.
    \item[\textbf{d)}] If you get to the top of Long's Peak, then you walk eight miles.
    \item[\textbf{e)}] If you are world famous, then you get tenure as a professor.
    \item[\textbf{f)}] If you drive more than \(400\) miles, then you will need to buy gasoline.
    \item[\textbf{g)}] If your guarantee is good, then you bought your CD player less than 90 days ago.
    \item[\textbf{h)}] If the water is not too cold, then Jan will go swimming.
    \item[\textbf{i)}] If people believe in science, then we will have a future.
\end{enumerate}

\section*{Exercise 26}
Write each of these statements in the form "if \(p\), then \(q\)" in English. [\textit{Hint:} Refer to the list of common ways to express conditional statements provided in this section.]
\begin{enumerate}
    \item[\textbf{a)}] I will remember to send you the address only if you send me an e-mail message.
    \item[\textbf{b)}] To be a citizen of this country, it is sufficient that you were born in the United States.
    \item[\textbf{c)}] If you keep your textbook, it will be a useful reference in your future courses.
    \item[\textbf{d)}] The Red Wings will win the Stanley Cup if their goalie plays well.
    \item[\textbf{e)}] That you get the job implies that you had the best credentials.
    \item[\textbf{f)}] The beach erodes whenever there is a storm.
    \item[\textbf{g)}] It is necessary to have a valid password to log on to the server.
    \item[\textbf{h)}] You will reach the summit unless you begin your climb too late.
    \item[\textbf{i)}] You will get a free ice cream cone, provided that you are among the first \(100\) customers tomorrow.
\end{enumerate}

\noindent
\textbf{Solution:}
\begin{enumerate}
    \item[\textbf{a)}] If I remember to send you the address, then you sent me an e-mail message.
    \item[\textbf{b)}] If you were born in the United States, then you are a citizen of this country.
    \item[\textbf{c)}] If you keep your textbook, then it will be a useful reference in your future courses.
    \item[\textbf{d)}] If their goalie plays well, then the Red Wings will win the Stanley Cup.
    \item[\textbf{e)}] If you get the job, then you had the best credentials.
    \item[\textbf{f)}] If there is a storm, then the beach erodes.
    \item[\textbf{g)}] If you log on to the server, then you have a valid password.
    \item[\textbf{h)}] If you do not begin your climb too late, then you will reach the summit.
    \item[\textbf{i)}] If you are among the first \(100\) customers tomorrow, then you will get a free ice cream cone.
\end{enumerate}

\section*{Exercise 27}
Write each of these propositions in the form "\(p\) if and only if \(q\)" in English.
\begin{enumerate}
    \item[\textbf{a)}] If it is hot outside you buy an ice cream cone, and if you buy an ice cream cone it is hot outside.
    \item[\textbf{b)}] For you to win the contest it is necessary and sufficient that you have the only winning ticket. 
    \item[\textbf{c)}] You get promoted only if you have connections, and you have connections only if you get promoted.
    \item[\textbf{d)}] If you watch television your mind will decay, and conversely.
    \item[\textbf{e)}] The trains run late on exactly those days when I take it.
\end{enumerate}

\noindent
\textbf{Solution:}
\begin{enumerate}
    \item[\textbf{a)}] It is hot outside if and only if you buy an ice cream cone.
    \item[\textbf{b)}] You win the contest if and only if you have the only winning ticket.
    \item[\textbf{c)}] You get promoted if and only if you have connections.
    \item[\textbf{d)}] You watch television if and only if your mind decays.
    \item[\textbf{e)}] The trains run late if and only if I take it.
\end{enumerate}

\section*{Exercise 28}
Write each of these propositions in the form "\(p\) if and only if \(q\)" in English.
\begin{enumerate}
    \item[\textbf{a)}] For you to get an A in this course, it is necessary and sufficient that you learn how to solve discrete mathematics problems.
    \item[\textbf{b)}] If you read the newspaper every day, you will be informed, and conversely.
    \item[\textbf{c)}] It rains if it is a weekend day, and it is a weekend day if it rains.
    \item[\textbf{d)}] You can see the wizard only if the wizard is not in, and the wizard is not in if you can see him.
    \item[\textbf{e)}] My airplane flight is late exactly when I have to catch a connecting flight. 
\end{enumerate}

\noindent
\textbf{Solution:}
\begin{enumerate}
    \item[\textbf{a)}] You get an A in this course if and only if you learn how to solve discrete mathematics problems.
    \item[\textbf{b)}] You read the newspaper every day if and only if you are informed.
    \item[\textbf{c)}] It rains if and only if it is a weekend day.
    \item[\textbf{d)}] You can see the wizard if and only if the wizard is not in.
    \item[\textbf{e)}] My airplane flight is late if and only if I have to catch a connecting flight.
\end{enumerate}

\section*{Exercise 29}
State the converse, contrapositive, and inverse of each of these conditional statements.
\begin{enumerate}
    \item[\textbf{a)}] If it snows today, I will ski tomorrow.
    \item[\textbf{b)}] I come to class whenever there is going to be a quiz.
    \item[\textbf{c)}] A positive integer is a prime only if it has no divisors other than \(1\) and itself.
\end{enumerate}

\noindent
\textbf{Solution:}
\begin{enumerate}
    \item[\textbf{a)}] Conditional: If it snows today, then I will ski tomorrow. Converse: If I ski tomorrow, then it snows today. Contrapositive: If I do not ski tomorrow, then it doesn't snow today. Inverse: If it does not snow today, then I will not ski tomorrow.
    \item[\textbf{b)}] Conditional: If there is going to be a quiz, then I come to class. Converse: If I come to class, then there is going to be a quiz. Contrapositive: If I don't come to class, then there is not going to be a quiz. Inverse: If there is not going to be a quiz, then I don't come to class.
    \item[\textbf{c)}] Conditional: If a positive integer is prime then it has no divisors other than \(1\) and itself. Converse: If a positive integer has no divisors other than \(1\) and itself, then it is prime. Contrapositive: If a positive integer has divisors other than \(1\) and itself, then the it is not prime. Inverse: If a positive integer is not prime then it has divisors other than \(1\) and itself.
\end{enumerate}

\section*{Exercise 30}
State the converse, contrapositive, and inverse of each of these conditional statements.
\begin{enumerate}
    \item[\textbf{a)}] If it snows tonight, then I will stay at home.
    \item[\textbf{b)}] I go to the beach whenever it is a sunny summer day.
    \item[\textbf{c)}] When I stay up late, it is necessary that I sleep until noon.
\end{enumerate}

\noindent
\textbf{Solution:}
\begin{enumerate}
    \item[\textbf{a)}] Conditional: If it snows tonight, then I will stay at home. Converse: If I stay at home, then it snows tonight. Contrapositive: If I do not stay at home, then it does not snow tonight. Inverse: If it does not snow tonight, then I will not stay at home.
    \item[\textbf{b)}] Conditional: If it is a sunny summer day, then I go to the beach. Converse: If I go to the beach, then it is a sunny summer day. Contrapositive: If I do not go to the beach, then it is not a sunny summer day. Inverse: If it is not a sunny summer day, then I do not go to the beach.
    \item[\textbf{c)}] Conditional: If I stay up late, then I sleep until noon. Converse: If I sleep until noon, then I stay up late. Contrapositive: If I do not sleep until noon, then I did not stay up late. Inverse: If I do not say up late, then I do not sleep until noon.
\end{enumerate}

\section*{Exercise 31}
How many rows appear in a truth table for each of these compound propositions?
\begin{enumerate}
    \item[\textbf{a)}] \(p \to \lnot p\)
    \item[\textbf{b)}] \((p \lor \lnot r) \land (q \lor \lnot s\)
    \item[\textbf{c)}] \(q \lor p \lor \lnot s \lor \lnot r \lor \lnot t \lor u\)
    \item[\textbf{d)}] \((p \land r \land t) \leftrightarrow (q \land t)\)
\end{enumerate}

\noindent
\textbf{Solution:}
\begin{enumerate}
    \item[\textbf{a)}] \(2\)
    \item[\textbf{b)}] \(16\)
    \item[\textbf{c)}] \(64\)
    \item[\textbf{d)}] \(16\)
\end{enumerate}

\section*{Exercise 32}
How many rows appear in a truth table for each of these compound propositions?
\begin{enumerate}
    \item[\textbf{a)}] \((q \to \lnot p) \lor (\lnot p \to \lnot q)\)
    \item[\textbf{b)}] \((p \lor \lnot t) \land (p \lor \lnot s)\)
    \item[\textbf{c)}] \((p \to r) \lor (\lnot s \to \lnot t) \lor (\lnot u \to v)\)
    \item[\textbf{d)}] \((p \land r \land s) \lor (q \land t) \lor (r \land \lnot t)\)
\end{enumerate}

\noindent
\textbf{Solution:}
\begin{enumerate}
    \item[\textbf{a)}] \(4\)
    \item[\textbf{b)}] \(8\)
    \item[\textbf{c)}] \(64\)
    \item[\textbf{d)}] \(32\)
\end{enumerate}

\section*{Exercise 33}
Construct a truth table for each of these compound propositions.
\begin{enumerate}
    \item[\textbf{a)}] \(p \land \lnot p\)
    \item[\textbf{b)}] \(p \lor \lnot p\)
    \item[\textbf{c)}] \((p \lor \lnot q) \to q\)
    \item[\textbf{d)}] \((p \lor q) \to (p \land q)\)
    \item[\textbf{e)}] \((p \to q) \leftrightarrow (\lnot q \to \lnot p)\)
    \item[\textbf{f)}] \((p \to q) \to (q \to p)\)
\end{enumerate}

\noindent
\textbf{Solution:}
\begin{table}[h]
    \centering
    \begin{tabular}{c|c|c}
        \(p\) & \(\lnot p\) & \(p \land \lnot p\) \\
        T     & F           & F                   \\
        F     & T           & F
    \end{tabular}
    \caption{Truth Table for 33(a)}
\end{table}

\begin{table}[h]
    \centering
    \begin{tabular}{c|c|c}
        \(p\) & \(\lnot p\) & \(p \lor \lnot p\) \\
        T     & F           & T                  \\
        F     & T           & T
    \end{tabular}
    \caption{Truth Table for 33(b)}
\end{table}

\begin{table}[h]
    \centering
    \begin{tabular}{c|c|c|c|c}
        \(p\) & \(q\) & \(\lnot q\) & \(p \lor \lnot q\) & \((p \lor \lnot q) \to q\) \\
        T     & T     & F           & T                  & T                          \\
        T     & F     & T           & T                  & F                          \\
        F     & T     & F           & F                  & T                          \\
        F     & F     & T           & T                  & F                        
    \end{tabular}
    \caption{Truth Table for 33(c)}
\end{table}

\begin{table}[ht]
    \centering
    \begin{tabular}{c|c|c|c|c}
        \(p\) & \(q\) & \(p \lor q\) & \(p \land q\) & \((p \lor q) \to (p \land q)\) \\
        T     & T     & T            & T             & T                              \\
        T     & F     & T            & F             & F                              \\
        F     & T     & T            & F             & F                              \\
        F     & F     & F            & F             & T
    \end{tabular}
    \caption{Truth Table for 33(d)}
\end{table}

\begin{table}[ht]
    \centering
    \begin{tabular}{c|c|c|c|c|c|c}
        \(p\) & \(q\) & \(\lnot p\) & \(\lnot q\) & \(p \to q\) & \(\lnot q \to \lnot p\) & \((p \to q) \leftrightarrow (\lnot q \to \lnot p)\) \\
        T     & T     & F           & F           & T           & T                       & T                                                   \\
        T     & F     & F           & T           & F           & F                       & T                                                   \\
        F     & T     & T           & F           & T           & T                       & T                                                   \\
        F     & F     & T           & T           & T           & T                       & T
    \end{tabular}
    \caption{Truth Table for 33(e)}
\end{table}

\begin{table}[ht]
    \centering
    \begin{tabular}{c|c|c|c|c}
        \(p\) & \(q\) & \(p \to q\) & \(q \to p\) & \((p \to q) \to (q \to p)\) \\
        T     & T     & T           & T           & T                           \\
        T     & F     & F           & T           & T                           \\
        F     & T     & T           & F           & F                           \\
        F     & F     & T           & T           & T
    \end{tabular}
    \caption{Truth Table for 33(f)}
\end{table}

\section*{Exercise 34}
Construct a truth table for each of these compound propositions.
\begin{itemize}
    \item[\textbf{a)}] \(p \to \lnot p\)
    \item[\textbf{b)}] \(p \leftrightarrow \lnot p\)
    \item[\textbf{c)}] \(p \oplus (p \lor q)\)
    \item[\textbf{d)}] \((p \land q) \to (p \lor q)\)
    \item[\textbf{e)}] \((q \to \lnot p) \leftrightarrow (p \leftrightarrow q)\)
    \item[\textbf{f)}] \((p \leftrightarrow q) \oplus (p \leftrightarrow \lnot q)\)
\end{itemize}

\noindent
\textbf{Solution:}
\begin{table}[ht]
    \centering
    \begin{tabular}{c|c|c}
        \(p\) & \(\lnot p\) & \(p \to \lnot p\) \\
        T     & F           & F                 \\
        F     & T           & T
    \end{tabular}
    \caption{Truth Table for 34(a)}
\end{table}

\begin{table}[ht]
    \centering
    \begin{tabular}{c|c|c}
        \(p\) & \(\lnot p\) & \(p \leftrightarrow \lnot p\) \\
        T     & F           & F                             \\
        F     & T           & F
    \end{tabular}
    \caption{Truth Table for 34(b)}
\end{table}

\begin{table}[ht]
    \centering
    \begin{tabular}{c|c|c|c}
    \(p\) & \(q\) & \(p \lor q\) & \(p \oplus (p \lor q)\) \\
    T     & T     & T            & F                       \\
    T     & F     & T            & F                       \\
    F     & T     & T            & T                       \\
    F     & F     & F            & F
    \end{tabular}
    \caption{Truth Table for 34(c)}
\end{table}

\begin{table}[ht]
    \centering
    \begin{tabular}{c|c|c|c|c}
        \(p\) & \(q\) & \(p \land q\) & \(p \lor q\) & \((p \land q) \to (p \lor q)\) \\
        T     & T     & T             & T            & T                              \\
        T     & F     & F             & T            & T                              \\
        F     & T     & F             & T            & T                              \\
        F     & F     & F             & F            & T
    \end{tabular}
    \caption{Truth Table for 34(d)}
\end{table}

\begin{table}[ht]
    \centering
    \begin{tabular}{c|c|c|c|c|c}
        \(p\) & \(q\) & \(\lnot p\) & \(q \to \lnot p\) & \(p \leftrightarrow q\) & \((q \to \lnot p) \leftrightarrow (p \leftrightarrow q)\) \\
        T     & T     & F           & F                 & T                       & F                                                         \\
        T     & F     & F           & T                 & F                       & F                                                         \\
        F     & T     & T           & T                 & F                       & F                                                         \\
        F     & F     & T           & T                 & T                       & T
    \end{tabular}
    \caption{Truth Table for 34(e)}
\end{table}

\begin{table}[ht]
    \centering
    \begin{tabular}{c|c|c|c|c|c}
        \(p\) & \(q\) & \(p \leftrightarrow q\) & \(\lnot q\) & \(p \leftrightarrow \lnot q\) & \((p \leftrightarrow q) \oplus (p \leftrightarrow \lnot q)\) \\
        T     & T     & T                       & F           & F                             & T                                                            \\
        T     & F     & F                       & T           & T                             & T                                                            \\
        F     & T     & F                       & F           & T                             & T                                                            \\
        F     & F     & T                       & T           & F                             & T
    \end{tabular}
    \caption{Truth Table for 34(f)}
\end{table}

\section*{Exercise 35}
Construct a truth table for each of these compound propositions.
\begin{enumerate}
    \item[\textbf{a)}] \((p \lor q) \to (p \oplus q)\)
    \item[\textbf{b)}] \((p \oplus q) \to (p \land q)\)
    \item[\textbf{c)}] \((p \lor q) \oplus (p \land q)\)
    \item[\textbf{d)}] \((p \leftrightarrow q) \oplus (\lnot p \leftrightarrow q)\)
    \item[\textbf{e)}] \((p \leftrightarrow q) \oplus (\lnot p \leftrightarrow \lnot r)\)
    \item[\textbf{f)}] \((p \oplus q) \to (p \oplus \lnot q)\)
\end{enumerate}

\noindent
\textbf{Solution:}
\begin{table}[ht]
    \centering
    \begin{tabular}{c|c|c|c|c}
        \(p\) & \(q\) & \(p \lor q\) & \(p \oplus q\) & \((p \lor q) \to (p \oplus q)\) \\
        T     & T     & T            & F              & F                               \\
        T     & F     & T            & T              & T                               \\
        F     & T     & T            & T              & T                               \\
        F     & F     & F            & F              & T
    \end{tabular}
    \caption{Truth Table for 35(a)}
\end{table}

\begin{table}[ht]
    \centering
    \begin{tabular}{c|c|c|c|c}
        \(p\) & \(q\) & \(p \oplus q\) & \(p \land q\) & \((p \oplus q) \to (p \land q)\) \\
        T     & T     & F              & T             & T                                \\
        T     & F     & T              & F             & F                                \\
        F     & T     & T              & F             & F                                \\
        F     & F     & F              & F             & T
    \end{tabular}
    \caption{Truth Table for 35(b)}
\end{table}

\begin{table}[ht]
    \centering
    \begin{tabular}{c|c|c|c|c}
    \(p\) & \(q\) & \(p \lor q\) & \(p \land q\) & \((p \lor q) \oplus (p \land q)\) \\
    T     & T     & T            & T             & F                                 \\
    T     & F     & T            & F             & T                                 \\
    F     & T     & T            & F             & T                                 \\
    F     & F     & F            & F             & F
    \end{tabular}
    \caption{Truth Table for 35(c)}
\end{table}

\begin{table}[ht]
    \centering
    \begin{tabular}{c|c|c|c|c|c}
        \(p\) & \(q\) & \(\lnot p\) & \(p \leftrightarrow q\) & \(\lnot p \leftrightarrow q\) & \((p \leftrightarrow q) \oplus (\lnot p \leftrightarrow q)\) \\
        T     & T     & F           & T                       & F                             & T                                                            \\
        T     & F     & F           & F                       & T                             & T                                                            \\
        F     & T     & T           & F                       & T                             & T                                                            \\
        F     & F     & T           & T                       & F                             & T
    \end{tabular}
    \caption{Truth Table for 35(d)}
\end{table}

\begin{table}[ht]
    \centering
    \begin{tabular}{c|c|c|c|c|c|c|c}
        \(p\) & \(q\) & \(r\) & \(p \leftrightarrow q\) & \(\lnot p\) & \(\lnot r\) & \(\lnot p \leftrightarrow \lnot r\) & \((p \leftrightarrow q) \oplus (\lnot p \leftrightarrow \lnot r)\) \\
        T     & T     & T     & T                       & F           & F           & T                                   & F                                                                  \\
        T     & T     & F     & T                       & F           & T           & F                                   & T                                                                  \\
        T     & F     & T     & F                       & F           & F           & T                                   & T                                                                  \\
        T     & F     & F     & F                       & F           & T           & F                                   & F                                                                  \\
        F     & T     & T     & F                       & T           & F           & F                                   & F                                                                  \\
        F     & T     & F     & F                       & T           & T           & T                                   & T                                                                  \\
        F     & F     & T     & T                       & T           & F           & F                                   & T                                                                  \\
        F     & F     & F     & T                       & T           & T           & T                                   & F
    \end{tabular}
    \caption{Truth Table for 35(e)}
\end{table}

\begin{table}[ht]
    \centering
    \begin{tabular}{c|c|c|c|c|c}
        \(p\) & \(q\) & \(\lnot q\) & \(p \oplus q\) & \(p \oplus \lnot q\) & \((p \oplus q) \to (p \oplus \lnot q)\) \\
        T     & T     & F           & F              & T                    & T                                       \\
        T     & F     & T           & T              & F                    & F                                       \\
        F     & T     & F           & T              & F                    & F                                       \\
        F     & F     & T           & F              & T                    & T
    \end{tabular}
    \caption{Truth Table for 35(f)}
\end{table}

\section*{Exercise 36}
Construct a truth table for each of these compound propositions.
\begin{enumerate}
    \item[\textbf{a)}] \(p \oplus p\)
    \item[\textbf{b)}] \(p \oplus \lnot p\)
    \item[\textbf{c)}] \(p \oplus \lnot q\)
    \item[\textbf{d)}] \(\lnot p \oplus \lnot q\)
    \item[\textbf{e)}] \((p \oplus q) \lor (p \oplus \lnot q)\)
    \item[\textbf{f)}] \((p \oplus q) \land (p \oplus \lnot q)\)
\end{enumerate}

\noindent
\textbf{Solution:}
\begin{table}[ht]
    \centering
    \begin{tabular}{c|c}
        \(p\) & \(p \oplus p\) \\
        T     & F              \\
        F     & F
    \end{tabular}
    \caption{Truth Table for 36(a)}
\end{table}

\begin{table}[ht]
    \centering
    \begin{tabular}{c|c|c}
        \(p\) & \(\lnot p\) & \(p \oplus \lnot p\) \\
        T     & F           & T                    \\
        F     & T           & T
    \end{tabular}
    \caption{Truth Table for 36(b)}
\end{table}

\begin{table}[ht]
    \centering
    \begin{tabular}{c|c|c|c}
        \(p\) & \(q\) & \(\lnot q\) & \(p \oplus \lnot q\) \\
        T     & T     & F           & T                    \\
        T     & F     & T           & F                    \\
        F     & T     & F           & F                    \\
        F     & F     & T           & T
    \end{tabular}
    \caption{Truth Table for 36(c)}
\end{table}

\begin{table}[ht]
    \centering
    \begin{tabular}{c|c|c|c|c}
        \(p\) & \(q\) & \(\lnot p\) & \(\lnot q\) & \(\lnot p \oplus \lnot q\) \\
        T     & T     & F           & F           & F                          \\
        T     & F     & F           & T           & T                          \\
        F     & T     & T           & F           & T                          \\
        F     & F     & T           & T           & F
    \end{tabular}
    \caption{Truth Table for 36(d)}
\end{table}

\begin{table}[ht]
    \centering
    \begin{tabular}{c|c|c|c|c|c}
        \(p\) & \(q\) & \(p \oplus q\) & \(\lnot q\) & \(p \oplus \lnot q\) & \((p \oplus q) \lor (p \oplus \lnot q)\) \\
        T     & T     & F              & F           & T                    & T                                        \\
        T     & F     & T              & T           & F                    & T                                        \\
        F     & T     & T              & F           & F                    & T                                        \\
        F     & F     & F              & T           & T                    & T
    \end{tabular}
    \caption{Truth Table for 36(e)}
\end{table}

\begin{table}[ht]
    \centering
    \begin{tabular}{c|c|c|c|c|c}
        \(p\) & \(q\) & \(p \oplus q\) & \(\lnot q\) & \(p \oplus \lnot q\) & \((p \oplus q) \land (p \oplus \lnot q)\) \\
        T     & T     & F              & F           & T                    & F                                         \\
        T     & F     & T              & T           & F                    & F                                         \\
        F     & T     & T              & F           & F                    & F                                         \\
        F     & F     & F              & T           & T                    & F 
    \end{tabular}
    \caption{Truth Table for 36(f)}
\end{table}

\section*{Exercise 37}
Construct a truth table for each of these compound propositions.
\begin{enumerate}
    \item[\textbf{a)}] \(p \to \lnot q\)
    \item[\textbf{b)}] \(\lnot p \leftrightarrow q\)
    \item[\textbf{c)}] \((p \to q) \lor (\lnot p \to q)\)
    \item[\textbf{d)}] \((p \to q) \land (\lnot p \to q)\)
    \item[\textbf{e)}] \((p \leftrightarrow q) \lor (\lnot p \leftrightarrow q)\)
    \item[\textbf{f)}] \((\lnot p \leftrightarrow \lnot q) \leftrightarrow (p \leftrightarrow q)\)
\end{enumerate}

\noindent
\textbf{Solution:}
\begin{table}[ht]
    \centering
    \begin{tabular}{c|c|c|c}
        \(p\) & \(q\) & \(\lnot q\) & \(p \to \lnot q\) \\
        T     & T     & F           & F                 \\
        T     & F     & T           & T                 \\
        F     & T     & F           & T                 \\
        F     & F     & T           & T
    \end{tabular}
    \caption{Truth Table for 37(a)}
\end{table}

\begin{table}[ht]
    \centering
    \begin{tabular}{c|c|c|c}
        \(p\) & \(q\) & \(\lnot p\) & \(\lnot p \leftrightarrow q\) \\
        T     & T     & F           & F                             \\
        T     & F     & F           & T                             \\
        F     & T     & T           & T                             \\
        F     & F     & T           & F
    \end{tabular}
    \caption{Truth Table for 37(b)}
\end{table}

\begin{table}[ht]
    \centering
    \begin{tabular}{c|c|c|c|c|c}
        \(p\) & \(q\) & \(\lnot p\) & \(p \to q\) & \(\lnot p \to q\) & \((p \to q) \lor (\lnot p \to q)\) \\
        T     & T     & F           & T           & T                 & T                                  \\
        T     & F     & F           & F           & T                 & T                                  \\
        F     & T     & T           & T           & T                 & T                                  \\
        F     & F     & T           & T           & F                 & T
    \end{tabular}
    \caption{Truth Table for 37(c)}
\end{table}

\begin{table}[ht]
    \centering
    \begin{tabular}{c|c|c|c|c|c}
        \(p\) & \(q\) & \(\lnot p\) & \(p \to q\) & \(\lnot p \to q\) & \((p \to q) \land (\lnot p \to q)\) \\
        T     & T     & F           & T           & T                 & T                                   \\
        T     & F     & F           & F           & T                 & F                                   \\
        F     & T     & T           & T           & T                 & T                                   \\
        F     & F     & T           & T           & F                 & F
        \end{tabular}
    \caption{Truth Table for 37(d)}
\end{table}

\begin{table}[ht]
    \centering
    \begin{tabular}{c|c|c|c|c|c}
        \(p\) & \(q\) & \(\lnot p\) & \(p \leftrightarrow q\) & \(\lnot p \leftrightarrow q\) & \((p \leftrightarrow q) \lor (\lnot p \leftrightarrow q)\) \\
        T     & T     & F           & T                       & F                             & T                                                          \\
        T     & F     & F           & F                       & T                             & T                                                          \\
        F     & T     & T           & F                       & T                             & T                                                          \\
        F     & F     & T           & T                       & F                             & T
    \end{tabular}
    \caption{Truth Table for 37(e)}
\end{table}

\begin{table}[ht]
    \centering
    \begin{tabular}{c|c|c|c|c|c|c}
        \(p\) & \(q\) & \(\lnot p\) & \(\lnot q\) & \(\lnot p \leftrightarrow \lnot q\) & \(p \leftrightarrow q\) & \((\lnot p \leftrightarrow \lnot q) \leftrightarrow (p \leftrightarrow q)\) \\
        T     & T     & F           & F           & T                                   & T                       & T                                                                           \\
        T     & F     & F           & T           & F                                   & F                       & T                                                                           \\
        F     & T     & T           & F           & F                                   & F                       & T                                                                           \\
        F     & F     & T           & T           & T                                   & T                       & T
    \end{tabular}
    \caption{Truth Table for 37(f)}
\end{table}

\section*{Exercise 38}
Construct a truth table for each of these compound propositions.
\begin{enumerate}
    \item[\textbf{a)}] \((p \lor q) \lor r\)
    \item[\textbf{b)}] \((p \lor q) \land r\)
    \item[\textbf{c)}] \((p \land q) \lor r\)
    \item[\textbf{d)}] \((p \land q) \land r\)
    \item[\textbf{e)}] \((p \lor q) \land \lnot r\)
    \item[\textbf{f)}] \((p \land q) \lor \lnot r\)
\end{enumerate}

\noindent
\textbf{Solution:}
\begin{table}[ht]
    \centering
    \begin{tabular}{c|c|c|c|c}
        \(p\) & \(q\) & \(r\) & \(p \lor q\) & \((p \lor q) \lor r\) \\
        T     & T     & T     & T            & T                     \\
        T     & T     & F     & T            & T                     \\
        T     & F     & T     & T            & T                     \\
        T     & F     & F     & T            & T                     \\
        F     & T     & T     & T            & T                     \\
        F     & T     & F     & T            & T                     \\
        F     & F     & T     & F            & T                     \\
        F     & F     & F     & F            & F                    
    \end{tabular}
    \caption{Truth Table for 38(a)}
\end{table}

\begin{table}[ht]
    \centering
    \begin{tabular}{c|c|c|c|c}
        \(p\) & \(q\) & \(r\) & \(p \lor q\) & \((p \lor q) \land r\) \\
        T     & T     & T     & T            & T                      \\
        T     & T     & F     & T            & F                      \\
        T     & F     & T     & T            & T                      \\
        T     & F     & F     & T            & F                      \\
        F     & T     & T     & T            & T                      \\
        F     & T     & F     & T            & F                      \\
        F     & F     & T     & F            & F                      \\
        F     & F     & F     & F            & F                     
    \end{tabular}
    \caption{Truth Table for 38(b)}
\end{table}

\begin{table}[ht]
    \centering
    \begin{tabular}{c|c|c|c|c}
        \(p\) & \(q\) & \(r\) & \(p \land q\) & \((p \land q) \lor r\) \\
        T     & T     & T     & T             & T                      \\
        T     & T     & F     & T             & T                      \\
        T     & F     & T     & F             & T                      \\
        T     & F     & F     & F             & F                      \\
        F     & T     & T     & F             & T                      \\
        F     & T     & F     & F             & F                      \\
        F     & F     & T     & F             & T                      \\
        F     & F     & F     & F             & F
    \end{tabular}
    \caption{Truth Table for 38(c)}
\end{table}

\begin{table}[ht]
    \centering
    \begin{tabular}{c|c|c|c|c}
        \(p\) & \(q\) & \(r\) & \(p \land q\) & \((p \land q) \land r\) \\
        T     & T     & T     & T             & T                       \\
        T     & T     & F     & T             & F                       \\
        T     & F     & T     & F             & F                       \\
        T     & F     & F     & F             & F                       \\
        F     & T     & T     & F             & F                       \\
        F     & T     & F     & F             & F                       \\
        F     & F     & T     & F             & F                       \\
        F     & F     & F     & F             & F                      
    \end{tabular}
    \caption{Truth Table for 38(d)}
\end{table}

\begin{table}[ht]
    \centering
    \begin{tabular}{c|c|c|c|c|c}
        \(p\) & \(q\) & \(r\) & \(\lnot r\) & \(p \lor q\) & \((p \lor q) \land \lnot r\) \\
        T     & T     & T     & F           & T            & F                            \\
        T     & T     & F     & T           & T            & T                            \\
        T     & F     & T     & F           & T            & F                            \\
        T     & F     & F     & T           & T            & T                            \\
        F     & T     & T     & F           & T            & F                            \\
        F     & T     & F     & T           & T            & T                            \\
        F     & F     & T     & F           & F            & F                            \\
        F     & F     & F     & T           & F            & F
    \end{tabular}
    \caption{Truth Table for 38(e)}
\end{table}

\begin{table}[ht]
    \centering
    \begin{tabular}{c|c|c|c|c|c}
        \(p\) & \(q\) & \(r\) & \(\lnot r\) & \(p \land q\) & \((p \land q) \lor \lnot r\) \\
        T     & T     & T     & F           & T             & T                            \\
        T     & T     & F     & T           & T             & T                            \\
        T     & F     & T     & F           & F             & F                            \\
        T     & F     & F     & T           & F             & T                            \\
        F     & T     & T     & F           & F             & F                            \\
        F     & T     & F     & T           & F             & T                            \\
        F     & F     & T     & F           & F             & F                            \\
        F     & F     & F     & T           & F             & T                            
    \end{tabular}
    \caption{Truth Table for 38(f)}
\end{table}

\section*{Exercise 39}
Construct a truth table for each of these compound propositions.
\begin{enumerate}
    \item[\textbf{a)}] \(p \to (\lnot q \lor r)\)
    \item[\textbf{b)}] \(\lnot p \to (q \to r)\)
    \item[\textbf{c)}] \((p \to q) \lor (\lnot p \to r)\)
    \item[\textbf{d)}] \((p \to q) \land (\lnot p \to r)\)
    \item[\textbf{e)}] \((p \leftrightarrow q) \lor (\lnot q \leftrightarrow r)\)
    \item[\textbf{f)}] \((\lnot p \leftrightarrow \lnot q) \leftrightarrow (q \leftrightarrow r)\)
\end{enumerate}

\noindent
\textbf{Section:}
\begin{table}[ht]
    \centering
    \begin{tabular}{c|c|c|c|c|c}
    \(p\) & \(q\) & \(r\) & \(\lnot q\) & \(\lnot q \lor r\) & \(p \to (\lnot q \lor r)\) \\
    T     & T     & T     & F           & T                  & T                          \\
    T     & T     & F     & F           & F                  & F                          \\
    T     & F     & T     & T           & T                  & T                          \\
    T     & F     & F     & T           & T                  & T                          \\
    F     & T     & T     & F           & T                  & T                          \\
    F     & T     & F     & F           & F                  & T                          \\
    F     & F     & T     & T           & T                  & T                          \\
    F     & F     & F     & T           & T                  & T
    \end{tabular}
    \caption{Truth Table for 39(a)}
\end{table}

\begin{table}[ht]
    \centering
    \begin{tabular}{c|c|c|c|c|c}
    \(p\) & \(q\) & \(r\) & \(\lnot p\) & \(q \to r\) & \(\lnot p \to (q \to r)\) \\
    T     & T     & T     & F           & T           & T                         \\
    T     & T     & F     & F           & F           & T                         \\
    T     & F     & T     & F           & T           & T                         \\
    T     & F     & F     & F           & T           & T                         \\
    F     & T     & T     & T           & T           & T                         \\
    F     & T     & F     & T           & F           & F                         \\
    F     & F     & T     & T           & T           & T                         \\
    F     & F     & F     & T           & T           & T
    \end{tabular}
    \caption{Truth Table for 39(b)}
\end{table}

\begin{table}[ht]
    \centering
    \begin{tabular}{c|c|c|c|c|c|c}
    \(p\) & \(q\) & \(r\) & \(\lnot p\) & \(p \to q\) & \(\lnot p \to r\) & \((p \to q) \lor (\lnot p \to r)\) \\
    T     & T     & T     & F           & T           & T                 & T                                  \\
    T     & T     & F     & F           & T           & T                 & T                                  \\
    T     & F     & T     & F           & F           & T                 & T                                  \\
    T     & F     & F     & F           & F           & T                 & T                                  \\
    F     & T     & T     & T           & T           & T                 & T                                  \\
    F     & T     & F     & T           & T           & F                 & T                                  \\
    F     & F     & T     & T           & T           & T                 & T                                  \\
    F     & F     & F     & T           & T           & F                 & T                                  
    \end{tabular}
    \caption{Truth Table for 39(c)}
\end{table}

\begin{table}[ht]
    \centering
    \begin{tabular}{c|c|c|c|c|c|c}
    \(p\) & \(q\) & \(r\) & \(\lnot p\) & \(p \to q\) & \(\lnot p \to r\) & \((p \to q) \land (\lnot p \to r)\) \\
    T     & T     & T     & F           & T           & T                 & T                                   \\
    T     & T     & F     & F           & T           & T                 & T                                   \\
    T     & F     & T     & F           & F           & T                 & F                                   \\
    T     & F     & F     & F           & F           & T                 & F                                   \\
    F     & T     & T     & T           & T           & T                 & T                                   \\
    F     & T     & F     & T           & T           & F                 & F                                   \\
    F     & F     & T     & T           & T           & T                 & T                                   \\
    F     & F     & F     & T           & T           & F                 & F                                   
    \end{tabular}
    \caption{Truth Table for 39(d)}
\end{table}

\begin{table}[ht]
    \centering
    \begin{tabular}{c|c|c|c|c|c|c}
    \(p\) & \(q\) & \(r\) & \(\lnot q\) & \(\lnot q \leftrightarrow r\) & \(p \leftrightarrow q\) & \((p \leftrightarrow q) \lor (\lnot q \leftrightarrow r)\) \\
    T     & T     & T     & F           & F                             & T                       & T                                                          \\
    T     & T     & F     & F           & T                             & T                       & T                                                          \\
    T     & F     & T     & T           & T                             & F                       & T                                                          \\
    T     & F     & F     & T           & F                             & F                       & F                                                          \\
    F     & T     & T     & F           & F                             & F                       & F                                                          \\
    F     & T     & F     & F           & T                             & F                       & T                                                          \\
    F     & F     & T     & T           & T                             & T                       & T                                                          \\
    F     & F     & F     & T           & F                             & T                       & T
    \end{tabular}
    \caption{Truth Table for 39(e)}
\end{table}

\begin{table}[ht]
    \centering
    \begin{tabular}{c|c|c|c|c|c|c|c}
    \(p\) & \(q\) & \(r\) & \(\lnot p\) & \(\lnot q\) & \(\lnot p \leftrightarrow \lnot q\) & \(q \leftrightarrow r\) & \((\lnot p \leftrightarrow \lnot q) \leftrightarrow (q \leftrightarrow r)\) \\
    T     & T     & T     & F           & F           & T                                   & T                       & T                                                                           \\
    T     & T     & F     & F           & F           & T                                   & F                       & F                                                                           \\
    T     & F     & T     & F           & T           & F                                   & F                       & T                                                                           \\
    T     & F     & F     & F           & T           & F                                   & T                       & F                                                                           \\
    F     & T     & T     & T           & F           & F                                   & T                       & F                                                                           \\
    F     & T     & F     & T           & F           & F                                   & F                       & T                                                                           \\
    F     & F     & T     & T           & T           & T                                   & F                       & F                                                                           \\
    F     & F     & F     & T           & T           & T                                   & T                       & T
    \end{tabular}
    \caption{Truth Table for 39(f)}
\end{table}

\section*{Exercise 40}
Construct a truth table for \(((p \to q) \to r) \to s\).

\noindent
\textbf{Solution:}
\begin{table}[ht]
    \centering
    \begin{tabular}{c|c|c|c|c|c|c}
    \(p\) & \(q\) & \(r\) & \(s\) & \(p \to q\) & \((p \to q) \to r)\) & \(((p \to q) \to r) \to s\) \\
    T     & T     & T     & T     & T           & T                    & T                           \\
    T     & T     & T     & F     & T           & T                    & F                           \\
    T     & T     & F     & T     & T           & F                    & T                           \\
    T     & T     & F     & F     & T           & F                    & T                           \\
    T     & F     & T     & T     & F           & T                    & T                           \\
    T     & F     & T     & F     & F           & T                    & F                           \\
    T     & F     & F     & T     & F           & T                    & T                           \\
    T     & F     & F     & F     & F           & T                    & F                           \\
    F     & T     & T     & T     & T           & T                    & T                           \\
    F     & T     & T     & F     & T           & T                    & F                           \\
    F     & T     & F     & T     & T           & F                    & T                           \\
    F     & T     & F     & F     & T           & F                    & T                           \\
    F     & F     & T     & T     & T           & T                    & T                           \\
    F     & F     & T     & F     & T           & T                    & F                           \\
    F     & F     & F     & T     & T           & F                    & T                           \\
    F     & F     & F     & F     & T           & F                    & T
    \end{tabular}
    \caption{Truth Table for 40}
\end{table}

\section*{Exercise 41}
Construct a truth table for \((p \leftrightarrow q) \leftrightarrow (r \leftrightarrow s)\).

\noindent
\textbf{Solution:}
\noindent
\textbf{Solution:}
\begin{table}[ht]
    \centering
    \begin{tabular}{c|c|c|c|c|c|c}
    \(p\) & \(q\) & \(r\) & \(s\) & \(p \leftrightarrow q\) & \(r \leftrightarrow s\) & \((p \leftrightarrow q) \leftrightarrow (r \leftrightarrow s)\) \\
    T     & T     & T     & T     & T                       & T                       & T                                                               \\
    T     & T     & T     & F     & T                       & F                       & F                                                               \\
    T     & T     & F     & T     & T                       & F                       & F                                                               \\
    T     & T     & F     & F     & T                       & T                       & T                                                               \\
    T     & F     & T     & T     & F                       & T                       & F                                                               \\
    T     & F     & T     & F     & F                       & F                       & T                                                               \\
    T     & F     & F     & T     & F                       & F                       & T                                                               \\
    T     & F     & F     & F     & F                       & T                       & F                                                               \\
    F     & T     & T     & T     & F                       & T                       & F                                                               \\
    F     & T     & T     & F     & F                       & F                       & T                                                               \\
    F     & T     & F     & T     & F                       & F                       & T                                                               \\
    F     & T     & F     & F     & F                       & T                       & F                                                               \\
    F     & F     & T     & T     & T                       & T                       & T                                                               \\
    F     & F     & T     & F     & T                       & F                       & F                                                               \\
    F     & F     & F     & T     & T                       & F                       & F                                                               \\
    F     & F     & F     & F     & T                       & T                       & T
    \end{tabular}
    \caption{Truth Table for 40}
\end{table}

\section*{Exercise 42}
Explain, without using a truth table, why \((p \lor \lnot q) \land (q \lor \lnot r) \land (r \lor \lnot p)\) is true when \(p\), \(q\), and \(r\) have the same truth value and it is false otherwise.

\noindent
\textbf{Solution:}
Suppose \(p\), \(q\), and \(r\) have the same truth value. Then every OR expression will come out to true since each contains both a propositional variable and a negated propositional variable (where the negated variable has the opposite value of the propositional variable). Finally, a conjunction of true propositional variables is true.

Suppose \(p\), \(q\), and \(r\) don't have the same truth value. For the entire compound proposition to be true, each OR must evaluate to true. Due top each variable and its negation being present, if one OR is true, then there must be one that is not true.

\section*{Exercise 43}
Explain, without using a truth table, why \((p \lor q \lor r) \land (\lnot p \lor \lnot q \lor \lnot r)\) is true when at least one of \(p\), \(q\), and \(r\) is true and at least one is false, but is false when all three variables have the same truth value.

Suppose at least one of \(p\), \(q\), and \(r\) is true and at least one is false. Then \(p \lor q \lor r\) is true, and \(\lnot p \lor \lnot q \lor \lnot r\) is true, thus the entire compound proposition is true.

Suppose that all three variables have the same truth value. Then one of the two compound disjunctions must be false, and thus the conjunction is false.

\section*{Exercise 44}
If \(p_1, p_2, \ldots, p_n\) are \(n\) propositions, explain why
\begin{equation*}
    \bigwedge_{i = 1}^{n - 1} \bigwedge_{j = i + 1}^n (\lnot p_i \lor \lnot p_j)
\end{equation*}
is true if and only if at most one of \(p_1, p_2, \ldots, p_n\) is true.

\noindent
\textbf{Solution:}
This compound conjunction relies on each OR compound proposition to be true, or else the whole "chain" of conjunction breaks. If more than one of \(p_1, p_2, \ldots, p_n\) is true, then there will exist an OR compound proposition that is false due to there being two true propositional variables being negated and then OR-ed.

\section*{Exercise 45}
Use Exercise 44 to construct a compound proposition that is true if and only if exactly one of the propositions \(p_1, p_2, \ldots, p_n\) is true. [\textit{Hint}: Combine the compound proposition in Exercise 44 and a compound proposition that is true if and only if at least one of \(p_1, p_2, \ldots, p_n\) is true.]

\noindent
\textbf{Solution:}
Consider the compound proposition
\begin{equation*}
    p_1 \lor p_2 \lor \cdots \lor p_n = \bigvee_{i = 1}^n p_i
\end{equation*}
This compound proposition is true if and only if one or more of the propositions \(p_1, p_2, \ldots, p_n\) is true. Therefore,
\begin{equation*}
    \left(\bigvee_{i = 1}^n p_i\right) \land \left(\bigwedge_{i = 1}^{n - 1} \bigwedge_{j = i + 1}^n (\lnot p_i \lor \lnot p_j)\right)
\end{equation*}
is a compound proposition that is true if and only if exactly one of the propositions \(p_1, p_2, \ldots, p_n\) is true.

\section*{Exercise 46}
What is the value of \(x\) after each of these statements is encountered in a computer program, if \(x = 1\) before the statement is reached?
\begin{enumerate}
    \item[\textbf{a)}] \textbf{if} \(x + 2 = 3\) \textbf{then} \(x \coloneq x + 1\)
    \item[\textbf{b)}] \textbf{if} \((x + 1 = 3) \ OR \ (2x + 2 = 3)\) \textbf{then} \(x \coloneq x + 1\)
    \item[\textbf{c)}] \textbf{if} \((2x + 3 = 5) \ AND \ (3x + 4 = 7)\) \textbf{then} \(x \coloneq x + 1\)
    \item[\textbf{d)}] \textbf{if} \((x + 1 = 2) \ XOR \ (x + 2 = 3)\) \textbf{then} \(x \coloneq x + 1\)
    \item[\textbf{e)}] \textbf{if} \(x < 2\) \textbf{then} \(x \coloneq x + 1\)
\end{enumerate}

\noindent
\textbf{Solution:}
\begin{enumerate}
    \item[\textbf{a)}] \(x = 2\)
    \item[\textbf{b)}] \(x = 1\)
    \item[\textbf{c)}] \(x = 2\)
    \item[\textbf{d)}] \(x = 1\)
    \item[\textbf{e)}] \(x = 2\)
\end{enumerate}

\section*{Exercise 47}
Find the bitwise \(OR\), bitwise \(AND\), and bitwise \(XOR\) of each of these pairs of bit strings.
\begin{enumerate}
    \item[\textbf{a)}] \(101 \ 1110, \ 010 \ 0001\)
    \item[\textbf{b)}] \(1111 \ 0000, \ 1010 \ 1010\)
    \item[\textbf{c)}] \(00 \ 0111 \ 0001, \ 10 \ 0100 \ 1000\)
    \item[\textbf{d)}] \(11 \ 1111 \ 1111, \ 00 \ 0000 \ 0000\)
\end{enumerate}

\noindent
\textbf{Solution:}
\begin{enumerate}
    \item[\textbf{a)}] \(OR: 111 \ 1111\); \(AND: 000 \ 0000\); \(XOR: 111 \ 1111\);
    \item[\textbf{b)}] \(OR: 1111 \ 1010\); \(AND: 1010 \ 0000\); \(XOR: 0101 \ 1010\);
    \item[\textbf{c)}] \(OR: 10 \ 0111 \ 1001\); \(AND: 00 \ 0100 \ 0000\); \(XOR: 10 \ 0011 \  1001\);
    \item[\textbf{d)}] \(OR: 11 \ 1111 \ 1111\); \(AND: 00 \ 0000 \ 0000\); \(XOR: 11 \ 1111 \ 1111\);
\end{enumerate}

\section*{Exercise 48}
Evaluate each of these expressions.
\begin{enumerate}
    \item[\textbf{a)}] \(1 \ 1000 \land (0 \ 1011 \lor 1 \ 1011)\)
    \item[\textbf{b)}] \((0 \ 1111 \ \land \ 1 \ 0101) \lor 0 \ 1000\)
    \item[\textbf{c)}] \((0 \ 1010 \oplus 1 \ 1011) \oplus 0 \ 1000\)
    \item[\textbf{d)}] \((1 \ 1011 \lor 0 \ 1010) \land (1 \ 0001 \lor 1 \ 1011)\)
\end{enumerate}

\noindent
\textbf{Solution:}
\begin{enumerate}
    \item[\textbf{a)}] \(1 \ 1000 \land (0 \ 1011 \lor 1 \ 1011) = 1 \ 1000 \land 1 \ 1011 = 1 \ 1000\)
    \item[\textbf{b)}] \((0 \ 1111 \ \land \ 1 \ 0101) \lor 0 \ 1000 = 0 \ 0101 \lor 0 \ 1000 = 0 \ 1101\)
    \item[\textbf{c)}] \((0 \ 1010 \oplus 1 \ 1011) \oplus 0 \ 1000 = 1 \ 0001 \oplus 0 \ 1000 = 1 \ 1001\)
    \item[\textbf{d)}] \((1 \ 1011 \lor 0 \ 1010) \land (1 \ 0001 \lor 1 \ 1011) = 1 \ 1011 \land 1 \ 1011 = 1 \ 1011\)
\end{enumerate}

\textbf{Fuzzy logic} is used in artificial intelligence. In fuzzy logic, a proposition has a truth value that is a number between \(0\) and \(1\), inclusive. A proposition with a truth value of \(0\) is false and one with a truth value of \(1\) is true. Truth values that are between \(0\) and \(1\) indicate varying degrees of truth. For instance, the truth value \(0.8\) can be assigned to the statement 
Fred is happy," because Fred is happy most of the time, and the truth value \(0.4\) can be assigned to the statement "John is happy," because John is happy slightly less than half the time. Use these truth values to solve Exercises 49-51.

\section*{Exercise 49}
The truth value of the negation of a proposition in fuzzy logic is \(1\) minus the truth value of the proposition. What are the truth values of the statements "Fred is not happy" and "John is not happy"?

\noindent
\textbf{Section:}
If the truth value of the statement "John is happy" is \(0.4\) and the truth value of the statement "Fred is happy" is \(0.8\), then the truth values of the statements "Fred is not happy" and "John is not happy" are \(0.2\) and \(0.6\) respectively. 

\section*{Exercise 50}
The truth value of the conjunction of two propositions in fuzzy logic is the minimum of the truth values of the two propositions. What are the truth values of the statements "Fred and John are happy" and "Neither Fred nor John is happy"?

\noindent
\textbf{Section:}
If the truth value of the statement "John is happy" is \(0.4\) and the truth value of the statement "Fred is happy" is \(0.8\), then the truth value of the statement "Fred and John are happy" is \(0.4\). If the truth value of the statement "John is not happy" is \(0.6\) and the truth value of the statement "Fred is not happy" is \(0.2\), then the truth value of the statement "Neither Fred nor John is happy" is \(0.2\).

\section*{Exercise 51}
The truth value of the disjunction of two propositions in fuzzy logic is the maximum of the truth values of the two propositions. What are the truth values of the statements "Fred is happy, or John is happy" and "Fred is not happy, or John is not happy"?

\noindent
\textbf{Solution:}
The truth value of the statements "Fred is happy, or John is happy" and "Fred is not happy, or John is not happy" are \(0.8\) and \(0.6\), respectively.

\section*{Exercise 52}
Is the assertion "This statement is false" a proposition?

\noindent
\textbf{Solution:}
A proposition is defined as a declarative sentence that is either true or false, but not both. The sentence "This statement is false" is inherently paradoxical, and thus takes on no declarative truth. Therefore, it is not a proposition.

\section*{Exercise 53}
The \(n\)th statement in a list of \(100\) statements is "Exactly \(n\) of the statements in this list are false."
\begin{enumerate}
    \item[\textbf{a)}] What conclusions can you draw from these statements?
    \item[\textbf{b)}] Answer part (a) if the \(n\)th statement is "At least \(n\) of the statements in this list are false."
    \item[\textbf{c)}] Answer part (b) assuming that the list contains \(99\) statements.
\end{enumerate}

\noindent
\textbf{Solution:}
\begin{itemize}
    \item[\textbf{a)}] The only possibility is for the \(99\)th statement to be true, and the rest to be false.
    \item[\textbf{b)}] Statements \(1\) through \(50\) are true, and statements \(51\) and \(100\) are false.
    \item[\textbf{c)}] This is an impossible situation. It is a paradox.
\end{itemize}

\section*{Exercise 54}
An ancient Sicilian legend says that the barber in a remote town who can be reached only by traveling a dangerous mountain road shaves those people, and only those people, who do not shave themselves. Can there be such a barber?

\noindent
\textbf{Solution:}
Suppose that such a barber exists and the barber shaves himself. This is a paradox, as he only shaves those who do not shave themselves. Suppose that the barber exist and the barber does not shave himself. Then, the barber would shave himself, a paradox. Therefore this barber can not exist.

\printbibliography

\end{document}