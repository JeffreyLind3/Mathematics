%%%%%%%%%%%%%%%%%%%%%%%%%%%%%%%%%%%%%%%%%%%%%%%%%%%%%%%%%%%%%%%%
%%%%%%%%%%%%%%%%%%%%%%%%%%% Metadata %%%%%%%%%%%%%%%%%%%%%%%%%%%
%%%%%%%%%%%%%%%%%%%%%%%%%%%%%%%%%%%%%%%%%%%%%%%%%%%%%%%%%%%%%%%%
\documentclass{Axon}

\title{Discrete Mathematics and its Applications, 8th Edition - Chapter 1 The Foundations: Logic and Proofs - Introduction}

\authors{
    \addauthor{Jeffrey G. Lind III}{jeffrey@jeffreylind.dev}
}

\addbibresource{Bibliography.bib}
%%%%%%%%%%%%%%%%%%%%%%%%%%%%%%%%%%%%%%%%%%%%%%%%%%%%%%%%%%%%%%%%
%%%%%%%%%%%%%%%%%%%%%%%%%%%%% Paper %%%%%%%%%%%%%%%%%%%%%%%%%%%%
%%%%%%%%%%%%%%%%%%%%%%%%%%%%%%%%%%%%%%%%%%%%%%%%%%%%%%%%%%%%%%%%
\begin{document}
\maketitle
\makeauthor
%%%%%%%%%%%%%%%%%%%%%%%%%%%%%%%%%%%%%%%%%%%%%%%%%%%%%%%%%%%%%%%%
%%%%%%%%%%%%%%%%%%%%%%%%%%% Abstract %%%%%%%%%%%%%%%%%%%%%%%%%%%
%%%%%%%%%%%%%%%%%%%%%%%%%%%%%%%%%%%%%%%%%%%%%%%%%%%%%%%%%%%%%%%%
\begin{abstract}
Notes on Discrete Mathematics and its Applications, 8th Edition - Chapter 1 The Foundations: Logic and Proofs - Introduction \cite{Rosen}.
\end{abstract}
%%%%%%%%%%%%%%%%%%%%%%%%%%%%%%%%%%%%%%%%%%%%%%%%%%%%%%%%%%%%%%%%
%%%%%%%%%%%%%%%%%%%%%%%%%%% Section 1 %%%%%%%%%%%%%%%%%%%%%%%%%%
%%%%%%%%%%%%%%%%%%%%%%%%%%%%%%%%%%%%%%%%%%%%%%%%%%%%%%%%%%%%%%%%
\section{Introduction}
The rules of logic specify the meaning of mathematical statements. For instance, these rules help us understand and reason with statements such as "There exists an integer that is not the sum of two squares" and "For every positive integer \(n\), the sum of the positive integers not exceeding \(n\) is \(\frac{n(n+1)}{2}\)." Logic is the basis of all mathematical reasoning, and of all automated reasoning. It has practical applications to the design of computing machines, to the specification of systems, to artificial intelligence, to computer programming, to programming languages, and to other areas of computer science, as well as to many other fields of study.

To understand mathematics, we must understand what makes up a correct mathematical argument, that is, a proof. Once we prove a mathematical statement is true, we call it a theorem. A collection of theorems on a topic organize what we know about this topic. To learn a mathematical topic, a person needs to actively construct mathematical arguments on this topic, and not just read exposition. Moreover, knowing the proof of a theorem often makes it possible to modify the result to fit new situations.

Everyone knows that proofs are important throughout mathematics, but many people find it surprising how important proofs are in computer science. In fact, proofs are used to verify that computer programs produce the correct output for all possible input values, to show that algorithms always produce the correct result, to establish the security of a system, and to create artificial intelligence. Furthermore, automated reasoning systems have been created to allow computers to construct their own proofs.

In this chapter, we will explain what makes up a correct mathematical argument and introduce tools to construct these arguments. We will develop an arsenal of different proof methods that will enable us to prove many different types of results. After introducing many different methods of proof, we will introduce several strategies for constructing proofs. We will introduce the notion of a conjecture and explain the process of developing mathematics by studying conjectures.
In this chapter, we will explain what makes up a correct mathematical argument and introduce tools to construct these arguments. We will develop an arsenal of different proof methods that will enable us to prove many different types of results. After introducing many different methods of proof, we will introduce several strategies for constructing proofs. We will introduce the notion of a conjecture and explain the process of developing mathematics by studying conjectures.

\printbibliography

\end{document}